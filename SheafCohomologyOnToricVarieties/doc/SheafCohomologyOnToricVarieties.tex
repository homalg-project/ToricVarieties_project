% generated by GAPDoc2LaTeX from XML source (Frank Luebeck)
\documentclass[a4paper,11pt]{report}

\usepackage[top=37mm,bottom=37mm,left=27mm,right=27mm]{geometry}
\sloppy
\pagestyle{myheadings}
\usepackage{amssymb}
\usepackage[utf8]{inputenc}
\usepackage{makeidx}
\makeindex
\usepackage{color}
\definecolor{FireBrick}{rgb}{0.5812,0.0074,0.0083}
\definecolor{RoyalBlue}{rgb}{0.0236,0.0894,0.6179}
\definecolor{RoyalGreen}{rgb}{0.0236,0.6179,0.0894}
\definecolor{RoyalRed}{rgb}{0.6179,0.0236,0.0894}
\definecolor{LightBlue}{rgb}{0.8544,0.9511,1.0000}
\definecolor{Black}{rgb}{0.0,0.0,0.0}

\definecolor{linkColor}{rgb}{0.0,0.0,0.554}
\definecolor{citeColor}{rgb}{0.0,0.0,0.554}
\definecolor{fileColor}{rgb}{0.0,0.0,0.554}
\definecolor{urlColor}{rgb}{0.0,0.0,0.554}
\definecolor{promptColor}{rgb}{0.0,0.0,0.589}
\definecolor{brkpromptColor}{rgb}{0.589,0.0,0.0}
\definecolor{gapinputColor}{rgb}{0.589,0.0,0.0}
\definecolor{gapoutputColor}{rgb}{0.0,0.0,0.0}

%%  for a long time these were red and blue by default,
%%  now black, but keep variables to overwrite
\definecolor{FuncColor}{rgb}{0.0,0.0,0.0}
%% strange name because of pdflatex bug:
\definecolor{Chapter }{rgb}{0.0,0.0,0.0}
\definecolor{DarkOlive}{rgb}{0.1047,0.2412,0.0064}


\usepackage{fancyvrb}

\usepackage{mathptmx,helvet}
\usepackage[T1]{fontenc}
\usepackage{textcomp}


\usepackage[
            pdftex=true,
            bookmarks=true,        
            a4paper=true,
            pdftitle={Written with GAPDoc},
            pdfcreator={LaTeX with hyperref package / GAPDoc},
            colorlinks=true,
            backref=page,
            breaklinks=true,
            linkcolor=linkColor,
            citecolor=citeColor,
            filecolor=fileColor,
            urlcolor=urlColor,
            pdfpagemode={UseNone}, 
           ]{hyperref}

\newcommand{\maintitlesize}{\fontsize{50}{55}\selectfont}

% write page numbers to a .pnr log file for online help
\newwrite\pagenrlog
\immediate\openout\pagenrlog =\jobname.pnr
\immediate\write\pagenrlog{PAGENRS := [}
\newcommand{\logpage}[1]{\protect\write\pagenrlog{#1, \thepage,}}
%% were never documented, give conflicts with some additional packages

\newcommand{\GAP}{\textsf{GAP}}

%% nicer description environments, allows long labels
\usepackage{enumitem}
\setdescription{style=nextline}

%% depth of toc
\setcounter{tocdepth}{1}





%% command for ColorPrompt style examples
\newcommand{\gapprompt}[1]{\color{promptColor}{\bfseries #1}}
\newcommand{\gapbrkprompt}[1]{\color{brkpromptColor}{\bfseries #1}}
\newcommand{\gapinput}[1]{\color{gapinputColor}{#1}}


\begin{document}

\logpage{[ 0, 0, 0 ]}
\begin{titlepage}
\mbox{}\vfill

\begin{center}{\maintitlesize \textbf{ SheafCohomologyOnToricVarieties \mbox{}}}\\
\vfill

\hypersetup{pdftitle= SheafCohomologyOnToricVarieties }
\markright{\scriptsize \mbox{}\hfill  SheafCohomologyOnToricVarieties  \hfill\mbox{}}
{\Huge \textbf{ A package to compute sheaf cohomology on toric varieties \mbox{}}}\\
\vfill

{\Huge  2019.01.11 \mbox{}}\\[1cm]
{ 11 January 2019 \mbox{}}\\[1cm]
\mbox{}\\[2cm]
{\Large \textbf{ Martin Bies\\
    \mbox{}}}\\
\hypersetup{pdfauthor= Martin Bies\\
    }
\end{center}\vfill

\mbox{}\\
{\mbox{}\\
\small \noindent \textbf{ Martin Bies\\
    }  Email: \href{mailto://martin.bies@alumni.uni-heidelberg.de} {\texttt{martin.bies@alumni.uni-heidelberg.de}}\\
  Homepage: \href{https://www.ulb.ac.be/sciences/ptm/pmif/people.html} {\texttt{https://www.ulb.ac.be/sciences/ptm/pmif/people.html}}\\
  Address: \begin{minipage}[t]{8cm}\noindent
 Physique Th{\a'e}orique et Math{\a'e}matique \\
 Universit{\a'e} Libre de Bruxelles \\
 Campus Plaine - CP 231 \\
 Building NO - Level 6 - Office O.6.111 \\
 1050 Brussels \\
 Belgium\\
 \end{minipage}
}\\
\end{titlepage}

\newpage\setcounter{page}{2}
\newpage

\def\contentsname{Contents\logpage{[ 0, 0, 1 ]}}

\tableofcontents
\newpage

     
\chapter{\textcolor{Chapter }{Introduction}}\label{Chapter_Introduction}
\logpage{[ 1, 0, 0 ]}
\hyperdef{L}{X7DFB63A97E67C0A1}{}
{
  

 
\section{\textcolor{Chapter }{What is the goal of the SheafCohomologyOnToricVarieties package?}}\label{Chapter_Introduction_Section_What_is_the_goal_of_the_SheafCohomologyOnToricVarieties_package}
\logpage{[ 1, 1, 0 ]}
\hyperdef{L}{X81D885427CCE7373}{}
{
  

 \emph{SheafCohomologyOnToricVarieties} provides data structures to compute sheaf cohomology on such varieties. The
ultimate goal is to provide high-performance-algorithms for its computation.
To this, our main theorem for the computation of sheaf cohomology is based on
an idea of Gregory G. Smith (see math/0305214 and DOI: 10.4171/OWR/2013/25),
which we combine with the powerful cohomCalg algorithm. Information on the
latter can be found at https://arxiv.org/abs/1003.5217v3 and references
therein. 

 }

 }

   
\chapter{\textcolor{Chapter }{Tools for FPGradedModules}}\label{Chapter_Tools_for_FPGradedModules}
\logpage{[ 2, 0, 0 ]}
\hyperdef{L}{X7D7ADE9084A9A8F0}{}
{
  
\section{\textcolor{Chapter }{Minimal free resolutions}}\label{Chapter_Tools_for_FPGradedModules_Section_Minimal_free_resolutions}
\logpage{[ 2, 1, 0 ]}
\hyperdef{L}{X82A7ED648724EF28}{}
{
  

\subsection{\textcolor{Chapter }{LeftIdealForCAP (for IsList, IsHomalgGradedRing)}}
\logpage{[ 2, 1, 1 ]}\nobreak
\hyperdef{L}{X8433F09B80165E70}{}
{\noindent\textcolor{FuncColor}{$\triangleright$\enspace\texttt{LeftIdealForCAP({\mdseries\slshape L, R})\index{LeftIdealForCAP@\texttt{LeftIdealForCAP}!for IsList, IsHomalgGradedRing}
\label{LeftIdealForCAP:for IsList, IsHomalgGradedRing}
}\hfill{\scriptsize (operation)}}\\
\textbf{\indent Returns:\ }
a f.p. module presentation 



 The argument is a list L of generators of an ideal and a homalg graded ring R.
This method then construct the left ideal in this ring generated by these
generators. }

 

\subsection{\textcolor{Chapter }{RightIdealForCAP (for IsList, IsHomalgGradedRing)}}
\logpage{[ 2, 1, 2 ]}\nobreak
\hyperdef{L}{X83B3D8E67B92F735}{}
{\noindent\textcolor{FuncColor}{$\triangleright$\enspace\texttt{RightIdealForCAP({\mdseries\slshape L, R})\index{RightIdealForCAP@\texttt{RightIdealForCAP}!for IsList, IsHomalgGradedRing}
\label{RightIdealForCAP:for IsList, IsHomalgGradedRing}
}\hfill{\scriptsize (operation)}}\\
\textbf{\indent Returns:\ }
a f.p. module presentation 



 The argument is a list L of generators of an ideal and a homalg graded ring R.
This method then construct the right ideal in this ring generated by these
generators. }

 

\subsection{\textcolor{Chapter }{MinimalFreeResolutionForCAP (for IsFpGradedLeftOrRightModulesObject)}}
\logpage{[ 2, 1, 3 ]}\nobreak
\hyperdef{L}{X86451DED791CA6FA}{}
{\noindent\textcolor{FuncColor}{$\triangleright$\enspace\texttt{MinimalFreeResolutionForCAP({\mdseries\slshape M})\index{MinimalFreeResolutionForCAP@\texttt{MinimalFreeResolutionForCAP}!for IsFpGradedLeftOrRightModulesObject}
\label{MinimalFreeResolutionForCAP:for IsFpGradedLeftOrRightModulesObject}
}\hfill{\scriptsize (attribute)}}\\
\textbf{\indent Returns:\ }
a complex of projective graded module morphisms 



 The argument is a graded left or right module presentation \mbox{\texttt{\mdseries\slshape M}}. We then compute a minimal free resolution of \mbox{\texttt{\mdseries\slshape M}}. }

 }

 
\section{\textcolor{Chapter }{Betti tables}}\label{Chapter_Tools_for_FPGradedModules_Section_Betti_tables}
\logpage{[ 2, 2, 0 ]}
\hyperdef{L}{X7D3F8F6880101A0B}{}
{
  

\subsection{\textcolor{Chapter }{BettiTableForCAP (for IsFpGradedLeftOrRightModulesObject)}}
\logpage{[ 2, 2, 1 ]}\nobreak
\hyperdef{L}{X7914E27A7999D6B8}{}
{\noindent\textcolor{FuncColor}{$\triangleright$\enspace\texttt{BettiTableForCAP({\mdseries\slshape M})\index{BettiTableForCAP@\texttt{BettiTableForCAP}!for IsFpGradedLeftOrRightModulesObject}
\label{BettiTableForCAP:for IsFpGradedLeftOrRightModulesObject}
}\hfill{\scriptsize (attribute)}}\\
\textbf{\indent Returns:\ }
a list of lists 



 The argument is a graded left or right module presentation \mbox{\texttt{\mdseries\slshape M}}. We then compute the Betti table of \mbox{\texttt{\mdseries\slshape M}}. }

 }

 
\section{\textcolor{Chapter }{Example: Minimal free resolution and Betti table}}\label{Chapter_Tools_for_FPGradedModules_Section_Example_Minimal_free_resolution_and_Betti_table}
\logpage{[ 2, 3, 0 ]}
\hyperdef{L}{X857F74B17988DA68}{}
{
  
\begin{Verbatim}[commandchars=@|E,fontsize=\small,frame=single,label=Example]
  @gapprompt|gap>E @gapinput|P2 := ProjectiveSpace( 2 );E
  <A projective toric variety of dimension 2>
  @gapprompt|gap>E @gapinput|IR := IrrelevantLeftIdealForCAP( P2 );;E
  @gapprompt|gap>E @gapinput|IsWellDefined( IR );E
  true
  @gapprompt|gap>E @gapinput|resolution := MinimalFreeResolutionForCAP( IR );E
  <An object in Complex category of Category of graded
  rows over Q[x_1,x_2,x_3] (with weights [ 1, 1, 1 ])>
  @gapprompt|gap>E @gapinput|differential_function :=E
  @gapprompt|>E @gapinput|                    UnderlyingZFunctorCell( resolution )!.differential_func;E
  function( i ) ... end
  @gapprompt|gap>E @gapinput|IsWellDefined( differential_function( -1 ) );E
  true
  @gapprompt|gap>E @gapinput|IsWellDefined( differential_function( -2 ) );E
  true
  @gapprompt|gap>E @gapinput|IsWellDefined( differential_function( -3 ) );E
  true
  @gapprompt|gap>E @gapinput|BT := BettiTableForCAP( IR );E
  [ [ -1, -1, -1 ], [ -2, -2, -2 ], [ -3 ] ]
\end{Verbatim}
 }

 }

   
\chapter{\textcolor{Chapter }{Additional methods and properties for toric varieties}}\label{Chapter_Additional_methods_and_properties_for_toric_varieties}
\logpage{[ 3, 0, 0 ]}
\hyperdef{L}{X7A66C08A8698D99B}{}
{
  
\section{\textcolor{Chapter }{Input check for cohomology computations}}\label{Chapter_Additional_methods_and_properties_for_toric_varieties_Section_Input_check_for_cohomology_computations}
\logpage{[ 3, 1, 0 ]}
\hyperdef{L}{X7FD321947CF8A4A6}{}
{
  

\subsection{\textcolor{Chapter }{IsValidInputForCohomologyComputations (for IsToricVariety)}}
\logpage{[ 3, 1, 1 ]}\nobreak
\hyperdef{L}{X8528A8937EB0BB37}{}
{\noindent\textcolor{FuncColor}{$\triangleright$\enspace\texttt{IsValidInputForCohomologyComputations({\mdseries\slshape V})\index{IsValidInputForCohomologyComputations@\texttt{IsValid}\-\texttt{Input}\-\texttt{For}\-\texttt{Cohomology}\-\texttt{Computations}!for IsToricVariety}
\label{IsValidInputForCohomologyComputations:for IsToricVariety}
}\hfill{\scriptsize (property)}}\\
\textbf{\indent Returns:\ }
\texttt{true} or \texttt{false} 



 Returns if the given variety V is a valid input for cohomology computations.
If the variable
SHEAF{\textunderscore}COHOMOLOGY{\textunderscore}ON{\textunderscore}TORIC{\textunderscore}VARIETIES{\textunderscore}INTERNAL{\textunderscore}LAZY
is set to false (default), then we just check if the variety is smooth,
complete. In case of success we return true and false otherwise. If however
SHEAF{\textunderscore}COHOMOLOGY{\textunderscore}ON{\textunderscore}TORIC{\textunderscore}VARIETIES{\textunderscore}INTERNAL{\textunderscore}LAZY
is set to true, then we will check if the variety is smooth, complete or
simplicial, projective. In case of success we return true and false other. }

 }

 
\section{\textcolor{Chapter }{Stanley-Reisner and irrelevant ideal via FpGradedModules}}\label{Chapter_Additional_methods_and_properties_for_toric_varieties_Section_Stanley-Reisner_and_irrelevant_ideal_via_FpGradedModules}
\logpage{[ 3, 2, 0 ]}
\hyperdef{L}{X865CD75C7B927C6F}{}
{
  

\subsection{\textcolor{Chapter }{GeneratorsOfIrrelevantIdeal (for IsToricVariety)}}
\logpage{[ 3, 2, 1 ]}\nobreak
\hyperdef{L}{X7B22B95F83A49431}{}
{\noindent\textcolor{FuncColor}{$\triangleright$\enspace\texttt{GeneratorsOfIrrelevantIdeal({\mdseries\slshape vari})\index{GeneratorsOfIrrelevantIdeal@\texttt{GeneratorsOfIrrelevantIdeal}!for IsToricVariety}
\label{GeneratorsOfIrrelevantIdeal:for IsToricVariety}
}\hfill{\scriptsize (attribute)}}\\
\textbf{\indent Returns:\ }
a list 



 Returns the lift of generators of the irrelevant ideal of the variety \mbox{\texttt{\mdseries\slshape vari}}. }

 

\subsection{\textcolor{Chapter }{IrrelevantLeftIdealForCAP (for IsToricVariety)}}
\logpage{[ 3, 2, 2 ]}\nobreak
\hyperdef{L}{X79ACE3F77FE30422}{}
{\noindent\textcolor{FuncColor}{$\triangleright$\enspace\texttt{IrrelevantLeftIdealForCAP({\mdseries\slshape vari})\index{IrrelevantLeftIdealForCAP@\texttt{IrrelevantLeftIdealForCAP}!for IsToricVariety}
\label{IrrelevantLeftIdealForCAP:for IsToricVariety}
}\hfill{\scriptsize (attribute)}}\\
\textbf{\indent Returns:\ }
a graded left ideal for CAP 



 Returns the irrelevant left ideal of the Cox ring of the variety \mbox{\texttt{\mdseries\slshape vari}}, using the language of CAP. }

 

\subsection{\textcolor{Chapter }{IrrelevantRightIdealForCAP (for IsToricVariety)}}
\logpage{[ 3, 2, 3 ]}\nobreak
\hyperdef{L}{X86F9A54682C69DBD}{}
{\noindent\textcolor{FuncColor}{$\triangleright$\enspace\texttt{IrrelevantRightIdealForCAP({\mdseries\slshape vari})\index{IrrelevantRightIdealForCAP@\texttt{IrrelevantRightIdealForCAP}!for IsToricVariety}
\label{IrrelevantRightIdealForCAP:for IsToricVariety}
}\hfill{\scriptsize (attribute)}}\\
\textbf{\indent Returns:\ }
a graded right ideal for CAP 



 Returns the irrelevant right ideal of the Cox ring of the variety \mbox{\texttt{\mdseries\slshape vari}}, using the language of CAP. }

 

\subsection{\textcolor{Chapter }{GeneratorsOfSRIdeal (for IsToricVariety)}}
\logpage{[ 3, 2, 4 ]}\nobreak
\hyperdef{L}{X78362D807A6E7334}{}
{\noindent\textcolor{FuncColor}{$\triangleright$\enspace\texttt{GeneratorsOfSRIdeal({\mdseries\slshape vari})\index{GeneratorsOfSRIdeal@\texttt{GeneratorsOfSRIdeal}!for IsToricVariety}
\label{GeneratorsOfSRIdeal:for IsToricVariety}
}\hfill{\scriptsize (attribute)}}\\
\textbf{\indent Returns:\ }
a list 



 Returns the lift of generators of the Stanley-Reisner-ideal of the variety \mbox{\texttt{\mdseries\slshape vari}}. }

 

\subsection{\textcolor{Chapter }{SRLeftIdealForCAP (for IsToricVariety)}}
\logpage{[ 3, 2, 5 ]}\nobreak
\hyperdef{L}{X80CA39B87FE8012B}{}
{\noindent\textcolor{FuncColor}{$\triangleright$\enspace\texttt{SRLeftIdealForCAP({\mdseries\slshape vari})\index{SRLeftIdealForCAP@\texttt{SRLeftIdealForCAP}!for IsToricVariety}
\label{SRLeftIdealForCAP:for IsToricVariety}
}\hfill{\scriptsize (attribute)}}\\
\textbf{\indent Returns:\ }
a graded left ideal for CAP 



 Returns the Stanley-Rei{\ss}ner left ideal of the Cox ring of the variety \mbox{\texttt{\mdseries\slshape vari}}, using the langauge of CAP. }

 

\subsection{\textcolor{Chapter }{SRRightIdealForCAP (for IsToricVariety)}}
\logpage{[ 3, 2, 6 ]}\nobreak
\hyperdef{L}{X7D5DC0FC86BF1246}{}
{\noindent\textcolor{FuncColor}{$\triangleright$\enspace\texttt{SRRightIdealForCAP({\mdseries\slshape vari})\index{SRRightIdealForCAP@\texttt{SRRightIdealForCAP}!for IsToricVariety}
\label{SRRightIdealForCAP:for IsToricVariety}
}\hfill{\scriptsize (attribute)}}\\
\textbf{\indent Returns:\ }
a graded right ideal for CAP 



 Returns the Stanley-Rei{\ss}ner right ideal of the Cox ring of the variety \mbox{\texttt{\mdseries\slshape vari}}, using the langauge of CAP. }

 

\subsection{\textcolor{Chapter }{FpGradedLeftModules (for IsToricVariety)}}
\logpage{[ 3, 2, 7 ]}\nobreak
\hyperdef{L}{X835E617380BF3A64}{}
{\noindent\textcolor{FuncColor}{$\triangleright$\enspace\texttt{FpGradedLeftModules({\mdseries\slshape variety})\index{FpGradedLeftModules@\texttt{FpGradedLeftModules}!for IsToricVariety}
\label{FpGradedLeftModules:for IsToricVariety}
}\hfill{\scriptsize (attribute)}}\\
\textbf{\indent Returns:\ }
a CapCategory 



 Given a toric variety \mbox{\texttt{\mdseries\slshape variety}} one can consider the Cox ring $S$ of this variety, which is graded over the class group of \mbox{\texttt{\mdseries\slshape variety}}. Subsequently one can consider the category of f.p. graded left $S$-modules. This attribute captures the corresponding CapCategory. }

 

\subsection{\textcolor{Chapter }{FpGradedRightModules (for IsToricVariety)}}
\logpage{[ 3, 2, 8 ]}\nobreak
\hyperdef{L}{X823D261B85C1284D}{}
{\noindent\textcolor{FuncColor}{$\triangleright$\enspace\texttt{FpGradedRightModules({\mdseries\slshape variety})\index{FpGradedRightModules@\texttt{FpGradedRightModules}!for IsToricVariety}
\label{FpGradedRightModules:for IsToricVariety}
}\hfill{\scriptsize (attribute)}}\\
\textbf{\indent Returns:\ }
a CapCategory 



 Given a toric variety \mbox{\texttt{\mdseries\slshape variety}} one can consider the Cox ring $S$ of this variety, which is graded over the class group of \mbox{\texttt{\mdseries\slshape variety}}. Subsequently one can consider the category of f.p. graded right $S$-modules. This attribute captures the corresponding CapCategory. }

 }

 
\section{\textcolor{Chapter }{Monoms of given degree in the Cox ring}}\label{Chapter_Additional_methods_and_properties_for_toric_varieties_Section_Monoms_of_given_degree_in_the_Cox_ring}
\logpage{[ 3, 3, 0 ]}
\hyperdef{L}{X87B26F0680686270}{}
{
  

\subsection{\textcolor{Chapter }{Exponents (for IsToricVariety, IsList)}}
\logpage{[ 3, 3, 1 ]}\nobreak
\hyperdef{L}{X868B1B6681613E2A}{}
{\noindent\textcolor{FuncColor}{$\triangleright$\enspace\texttt{Exponents({\mdseries\slshape vari, degree})\index{Exponents@\texttt{Exponents}!for IsToricVariety, IsList}
\label{Exponents:for IsToricVariety, IsList}
}\hfill{\scriptsize (operation)}}\\
\textbf{\indent Returns:\ }
a list of lists of integers 



 Given a smooth and complete toric variety and a list of integers (= degree)
corresponding to an element of the class group of the variety, this method
return a list of integer valued lists. These lists correspond to the exponents
of the monomials of degree in the Cox ring of this toric variety. }

 

\subsection{\textcolor{Chapter }{MonomsOfCoxRingOfDegreeByNormaliz (for IsToricVariety, IsList)}}
\logpage{[ 3, 3, 2 ]}\nobreak
\hyperdef{L}{X87C90D267B9462DF}{}
{\noindent\textcolor{FuncColor}{$\triangleright$\enspace\texttt{MonomsOfCoxRingOfDegreeByNormaliz({\mdseries\slshape vari, degree})\index{MonomsOfCoxRingOfDegreeByNormaliz@\texttt{MonomsOfCoxRingOfDegreeByNormaliz}!for IsToricVariety, IsList}
\label{MonomsOfCoxRingOfDegreeByNormaliz:for IsToricVariety, IsList}
}\hfill{\scriptsize (operation)}}\\
\textbf{\indent Returns:\ }
a list 



 Given a smooth and complete toric variety and a list of integers (= degree)
corresponding to an element of the class group of the variety, this method
returns the list of all monomials in the Cox ring of the given degree. This
method uses NormalizInterface. }

 

\subsection{\textcolor{Chapter }{MonomsOfCoxRingOfDegreeByNormalizAsColumnMatrices (for IsToricVariety, IsList, IsPosInt, IsPosInt)}}
\logpage{[ 3, 3, 3 ]}\nobreak
\hyperdef{L}{X7F90985F79409480}{}
{\noindent\textcolor{FuncColor}{$\triangleright$\enspace\texttt{MonomsOfCoxRingOfDegreeByNormalizAsColumnMatrices({\mdseries\slshape vari, degree, i, l})\index{MonomsOfCoxRingOfDegreeByNormalizAsColumnMatrices@\texttt{Monoms}\-\texttt{Of}\-\texttt{Cox}\-\texttt{Ring}\-\texttt{Of}\-\texttt{Degree}\-\texttt{By}\-\texttt{Normaliz}\-\texttt{As}\-\texttt{Column}\-\texttt{Matrices}!for IsToricVariety, IsList, IsPosInt, IsPosInt}
\label{MonomsOfCoxRingOfDegreeByNormalizAsColumnMatrices:for IsToricVariety, IsList, IsPosInt, IsPosInt}
}\hfill{\scriptsize (operation)}}\\
\textbf{\indent Returns:\ }
a list of matrices 



 Given a smooth and complete toric variety, a list of integers (= degree)
corresponding to an element of the class group of the variety and two
non-negative integers i and l, this method returns a list of column matrices.
The columns are of length l and have at position i the monoms of the Coxring
of degree 'degree'. }

 }

 
\section{\textcolor{Chapter }{Example: Stanley-Reisner ideal for CAP}}\label{Chapter_Additional_methods_and_properties_for_toric_varieties_Section_Example_Stanley-Reisner_ideal_for_CAP}
\logpage{[ 3, 4, 0 ]}
\hyperdef{L}{X7D8E966785BA84AA}{}
{
  
\begin{Verbatim}[commandchars=!@|,fontsize=\small,frame=single,label=Example]
  !gapprompt@gap>| !gapinput@P2 := ProjectiveSpace( 2 );|
  <A projective toric variety of dimension 2>
  !gapprompt@gap>| !gapinput@SR1 := SRLeftIdealForCAP( P2 );;|
  !gapprompt@gap>| !gapinput@IsWellDefined( SR1 );|
  true
  !gapprompt@gap>| !gapinput@SR2 := SRRightIdealForCAP( P2 );;|
  !gapprompt@gap>| !gapinput@IsWellDefined( SR2 );|
  true
\end{Verbatim}
 }

 
\section{\textcolor{Chapter }{Example: Irrelevant ideal for CAP}}\label{Chapter_Additional_methods_and_properties_for_toric_varieties_Section_Example_Irrelevant_ideal_for_CAP}
\logpage{[ 3, 5, 0 ]}
\hyperdef{L}{X7897E1B678B1BD24}{}
{
  
\begin{Verbatim}[commandchars=!@|,fontsize=\small,frame=single,label=Example]
  !gapprompt@gap>| !gapinput@P2 := ProjectiveSpace( 2 );|
  <A projective toric variety of dimension 2>
  !gapprompt@gap>| !gapinput@IR1 := IrrelevantLeftIdealForCAP( P2 );;|
  !gapprompt@gap>| !gapinput@IsWellDefined( IR1 );|
  true
  !gapprompt@gap>| !gapinput@IR2 := IrrelevantRightIdealForCAP( P2 );;|
  !gapprompt@gap>| !gapinput@IsWellDefined( IR2 );|
  true
\end{Verbatim}
 }

 
\section{\textcolor{Chapter }{Example: Monomials of given degree}}\label{Chapter_Additional_methods_and_properties_for_toric_varieties_Section_Example_Monomials_of_given_degree}
\logpage{[ 3, 6, 0 ]}
\hyperdef{L}{X86BBA5BF8112A591}{}
{
  
\begin{Verbatim}[commandchars=!@|,fontsize=\small,frame=single,label=Example]
  !gapprompt@gap>| !gapinput@P1 := ProjectiveSpace( 1 );|
  <A projective toric variety of dimension 1>
  !gapprompt@gap>| !gapinput@var := P1*P1;|
  <A projective toric variety of dimension 2
  which is a product of 2 toric varieties>
  !gapprompt@gap>| !gapinput@Exponents( var, [ 1,1 ] );|
  [ [ 1, 1, 0, 0 ], [ 1, 0, 1, 0 ],
  [ 0, 1, 0, 1 ], [ 0, 0, 1, 1 ] ]
  !gapprompt@gap>| !gapinput@MonomsOfCoxRingOfDegreeByNormaliz( var, [1,2] );|
  [ x_1^2*x_2, x_1^2*x_3, x_1*x_2*x_4,
  x_1*x_3*x_4, x_2*x_4^2, x_3*x_4^2 ]
  !gapprompt@gap>| !gapinput@MonomsOfCoxRingOfDegreeByNormaliz( var, [-1,-1] );|
  []
  !gapprompt@gap>| !gapinput@l := MonomsOfCoxRingOfDegreeByNormalizAsColumnMatrices|
  !gapprompt@>| !gapinput@     ( var, [1,2], 2, 3 );;|
  !gapprompt@gap>| !gapinput@Display( l[ 1 ] );|
  0,
  x_1^2*x_2,
  0
  (over a graded ring)
\end{Verbatim}
 }

 }

   
\chapter{\textcolor{Chapter }{Cohomology of coherent sheaves from resolution}}\label{Chapter_Cohomology_of_coherent_sheaves_from_resolution}
\logpage{[ 4, 0, 0 ]}
\hyperdef{L}{X8223BB1C7A235BD3}{}
{
  
\section{\textcolor{Chapter }{Deductions On Sheaf Cohomology From Cohomology Of projective modules in a
minimal free resolution}}\label{Chapter_Cohomology_of_coherent_sheaves_from_resolution_Section_Deductions_On_Sheaf_Cohomology_From_Cohomology_Of_projective_modules_in_a_minimal_free_resolution}
\logpage{[ 4, 1, 0 ]}
\hyperdef{L}{X80480F098054EA3B}{}
{
  

\subsection{\textcolor{Chapter }{CohomologiesList (for IsToricVariety, IsFpGradedLeftOrRightModulesObject)}}
\logpage{[ 4, 1, 1 ]}\nobreak
\hyperdef{L}{X7EE2EBD87859666A}{}
{\noindent\textcolor{FuncColor}{$\triangleright$\enspace\texttt{CohomologiesList({\mdseries\slshape vari, M})\index{CohomologiesList@\texttt{CohomologiesList}!for IsToricVariety, IsFpGradedLeftOrRightModulesObject}
\label{CohomologiesList:for IsToricVariety, IsFpGradedLeftOrRightModulesObject}
}\hfill{\scriptsize (operation)}}\\
\textbf{\indent Returns:\ }
a list of lists of integers 



 Given a smooth and projective toric variety $vari$ with Coxring $S$ and a f. p. graded S-modules $M$, this method computes a minimal free resolution of $M$ and then the dimension of the cohomology classes of the projective modules in
this minimal free resolution. }

 

\subsection{\textcolor{Chapter }{DeductionOfSheafCohomologyFromResolution (for IsToricVariety, IsFpGradedLeftOrRightModulesObject, IsBool)}}
\logpage{[ 4, 1, 2 ]}\nobreak
\hyperdef{L}{X8318355A7A5ED933}{}
{\noindent\textcolor{FuncColor}{$\triangleright$\enspace\texttt{DeductionOfSheafCohomologyFromResolution({\mdseries\slshape vari, M})\index{DeductionOfSheafCohomologyFromResolution@\texttt{Deduction}\-\texttt{Of}\-\texttt{Sheaf}\-\texttt{Cohomology}\-\texttt{From}\-\texttt{Resolution}!for IsToricVariety, IsFpGradedLeftOrRightModulesObject, IsBool}
\label{DeductionOfSheafCohomologyFromResolution:for IsToricVariety, IsFpGradedLeftOrRightModulesObject, IsBool}
}\hfill{\scriptsize (operation)}}\\
\textbf{\indent Returns:\ }
a list 



 Given a smooth and projective toric variety $vari$ with Coxring $S$ and a f. p. graded S-modules $M$, this method computes a minimal free resolution of $M$ and then the dimension of the cohomology classes of the projective modules in
this minimal free resolution. From this information we draw conclusions on the
sheaf cohomologies of the sheaf $\tilde{M}$. }

 }

 
\section{\textcolor{Chapter }{Example: Pullback line bundle}}\label{Chapter_Cohomology_of_coherent_sheaves_from_resolution_Section_Example_Pullback_line_bundle}
\logpage{[ 4, 2, 0 ]}
\hyperdef{L}{X84092AAF7F8D1BAB}{}
{
  
\begin{Verbatim}[commandchars=!@|,fontsize=\small,frame=single,label=Example]
  !gapprompt@gap>| !gapinput@var := ProjectiveSpace( 2 );|
  <A projective toric variety of dimension 2>
  !gapprompt@gap>| !gapinput@cox_ring := CoxRing( var );|
  Q[x_1,x_2,x_3]
  (weights: [ 1, 1, 1 ])
  !gapprompt@gap>| !gapinput@vars := IndeterminatesOfPolynomialRing( cox_ring );|
  [ x_1, x_2, x_3 ]
  !gapprompt@gap>| !gapinput@range := GradedRow( [[[2],1]], cox_ring );|
  <A graded row of rank 1>
  !gapprompt@gap>| !gapinput@source := GradedRow( [[[1],1]], cox_ring );|
  <A graded row of rank 1>
  !gapprompt@gap>| !gapinput@matrix := HomalgMatrix( [[vars[1]] ], cox_ring );|
  <A 1 x 1 matrix over a graded ring>
  !gapprompt@gap>| !gapinput@mor := GradedRowOrColumnMorphism( source, matrix, range );|
  <A morphism in Category of graded rows over
  Q[x_1,x_2,x_3]
  (with weights [ 1,1,1 ])>
  !gapprompt@gap>| !gapinput@IsWellDefined( mor );|
  true
  !gapprompt@gap>| !gapinput@pullback_line_bundle := FreydCategoryObject( mor );|
  <An object in Category of f.p. graded left modules over
  Q[x_1,x_2,x_3] (with weights
  [ 1, 1, 1 ])>
  !gapprompt@gap>| !gapinput@coh := DeductionOfSheafCohomologyFromResolution( var, pullback_line_bundle );|
  [ 3, 0, 0 ]
\end{Verbatim}
 
\begin{Verbatim}[commandchars=!@|,fontsize=\small,frame=single,label=Example]
  !gapprompt@gap>| !gapinput@P1 := ProjectiveSpace( 1 );|
  <A projective toric variety of dimension 1>
  !gapprompt@gap>| !gapinput@P2 := ProjectiveSpace( 2 );|
  <A projective toric variety of dimension 2>
  !gapprompt@gap>| !gapinput@var2 := P1 * P1 * P2;|
  <A projective toric variety of dimension 4 
  which is a product of 3 toric varieties>
  !gapprompt@gap>| !gapinput@cox_ring2 := CoxRing( var2 );|
  Q[x_1,x_2,x_3,x_4,x_5,x_6,x_7]
  (weights: [ ( 0, 0, 1 ), ( 0, 1, 0 ), ( 1, 0, 0 ),
  ( 1, 0, 0 ), ( 1, 0, 0 ), ( 0, 1, 0 ), ( 0, 0, 1 ) ])
  !gapprompt@gap>| !gapinput@vars2 := IndeterminatesOfPolynomialRing( cox_ring2 );|
  [ x_1, x_2, x_3, x_4, x_5, x_6, x_7 ]
  !gapprompt@gap>| !gapinput@range2 := GradedRow( [[[1,1,2],1]], cox_ring2 );|
  <A graded row of rank 1>
  !gapprompt@gap>| !gapinput@source2 := GradedRow( [[[0,1,2],2]], cox_ring2 );|
  <A graded row of rank 2>
  !gapprompt@gap>| !gapinput@matrix2 := HomalgMatrix( [[vars2[3]],[vars2[4]]], cox_ring2 );|
  <A 2 x 1 matrix over a graded ring>
  !gapprompt@gap>| !gapinput@mor2 := GradedRowOrColumnMorphism( source2, matrix2, range2 );|
  <A morphism in Category of graded rows over
  Q[x_1,x_2,x_3,x_4,x_5,x_6,x_7]
  (with weights [ [ 0, 0, 1 ], [ 0, 1, 0 ], [ 1, 0, 0 ],
  [ 1, 0, 0 ], [ 1, 0, 0 ], [ 0, 1, 0 ], [ 0, 0, 1 ] ])>
  !gapprompt@gap>| !gapinput@IsWellDefined( mor2 );|
  true
  !gapprompt@gap>| !gapinput@pullback_line_bundle2 := FreydCategoryObject( mor2 );|
  <An object in Category of f.p. graded left modules over
  Q[x_1,x_2,x_3,x_4,x_5,x_6,x_7] (with weights
  [ [ 0, 0, 1 ], [ 0, 1, 0 ], [ 1, 0, 0 ], [ 1, 0, 0 ],
  [ 1, 0, 0 ], [ 0, 1, 0 ], [ 0, 0, 1 ] ])>
  !gapprompt@gap>| !gapinput@coh2 := DeductionOfSheafCohomologyFromResolution( var2, pullback_line_bundle2 );|
  [ 6, 0, 0, 0, 0 ]
\end{Verbatim}
 
\begin{Verbatim}[commandchars=!@|,fontsize=\small,frame=single,label=Example]
  !gapprompt@gap>| !gapinput@P2 := ProjectiveSpace( 2 );|
  <A projective toric variety of dimension 2>
  !gapprompt@gap>| !gapinput@var3 := P2 * P2;|
  <A projective toric variety of dimension 4 
  which is a product of 2 toric varieties>
  !gapprompt@gap>| !gapinput@cox_ring3 := CoxRing( var3 );|
  Q[x_1,x_2,x_3,x_4,x_5,x_6]
  (weights: [ ( 0, 1 ), ( 1, 0 ), ( 1, 0 ), 
  ( 1, 0 ), ( 0, 1 ), ( 0, 1 ) ])
  !gapprompt@gap>| !gapinput@range3 := GradedRow( [[[1,1],4]], cox_ring3 );|
  <A graded row of rank 4>
  !gapprompt@gap>| !gapinput@source3 := ZeroObject( CapCategory( range3 ) );|
  <A graded row of rank 0>
  !gapprompt@gap>| !gapinput@matrix3 := HomalgZeroMatrix( 0, 4, cox_ring3 );|
  <An unevaluated 0 x 4 zero matrix over a graded ring>
  !gapprompt@gap>| !gapinput@mor3 := GradedRowOrColumnMorphism( source3, matrix3, range3 );|
  <A morphism in Category of graded rows over
  Q[x_1,x_2,x_3,x_4,x_5,x_6] (with weights
  [ [ 0, 1 ], [ 1, 0 ], [ 1, 0 ], [ 1, 0 ], [ 0, 1 ], [ 0, 1 ] ])>
  !gapprompt@gap>| !gapinput@line_bundle3 := FreydCategoryObject( mor3 );|
  <An object in Category of f.p. graded left modules over 
  Q[x_1,x_2,x_3,x_4,x_5,x_6] (with weights 
  [ [ 0, 1 ], [ 1, 0 ], [ 1, 0 ], [ 1, 0 ], [0, 1 ], [ 0, 1 ] ])>
  !gapprompt@gap>| !gapinput@IsWellDefined( line_bundle3 );|
  true
  !gapprompt@gap>| !gapinput@coh3 := DeductionOfSheafCohomologyFromResolution( var3, line_bundle3 );|
  [ 36, 0, 0, 0, 0 ]
\end{Verbatim}
 }

 }

   
\chapter{\textcolor{Chapter }{Wrapper for generators of semigroups and hyperplane constraints of cones}}\label{Chapter_Wrapper_for_generators_of_semigroups_and_hyperplane_constraints_of_cones}
\logpage{[ 5, 0, 0 ]}
\hyperdef{L}{X80F4CB6E7CC3A8F6}{}
{
  
\section{\textcolor{Chapter }{GAP Categories}}\label{Chapter_Wrapper_for_generators_of_semigroups_and_hyperplane_constraints_of_cones_Section_GAP_Categories}
\logpage{[ 5, 1, 0 ]}
\hyperdef{L}{X7D03633A7D98026B}{}
{
  

\subsection{\textcolor{Chapter }{IsSemigroupForPresentationsByProjectiveGradedModules (for IsObject)}}
\logpage{[ 5, 1, 1 ]}\nobreak
\hyperdef{L}{X7F620E4585CD2BCA}{}
{\noindent\textcolor{FuncColor}{$\triangleright$\enspace\texttt{IsSemigroupForPresentationsByProjectiveGradedModules({\mdseries\slshape object})\index{IsSemigroupForPresentationsByProjectiveGradedModules@\texttt{IsSemigroup}\-\texttt{For}\-\texttt{Presentations}\-\texttt{By}\-\texttt{Projective}\-\texttt{Graded}\-\texttt{Modules}!for IsObject}
\label{IsSemigroupForPresentationsByProjectiveGradedModules:for IsObject}
}\hfill{\scriptsize (filter)}}\\
\textbf{\indent Returns:\ }
\texttt{true} or \texttt{false} 



 The GAP category of lists of integer-valued lists, which encode the generators
of subsemigroups of $Z^n$. }

 

\subsection{\textcolor{Chapter }{IsAffineSemigroupForPresentationsByProjectiveGradedModules (for IsObject)}}
\logpage{[ 5, 1, 2 ]}\nobreak
\hyperdef{L}{X7847C2907F558C0B}{}
{\noindent\textcolor{FuncColor}{$\triangleright$\enspace\texttt{IsAffineSemigroupForPresentationsByProjectiveGradedModules({\mdseries\slshape object})\index{IsAffineSemigroupForPresentationsByProjectiveGradedModules@\texttt{IsAffine}\-\texttt{Semigroup}\-\texttt{For}\-\texttt{Presentations}\-\texttt{By}\-\texttt{Projective}\-\texttt{Graded}\-\texttt{Modules}!for IsObject}
\label{IsAffineSemigroupForPresentationsByProjectiveGradedModules:for IsObject}
}\hfill{\scriptsize (filter)}}\\
\textbf{\indent Returns:\ }
\texttt{true} or \texttt{false} 



 The GAP category of affine semigroups $H$ in $\mathbb{Z}^n$. That means that there is a semigroup $G \subseteq \mathbb{Z}^n$ and $p \in \mathbb{Z}^n$ such that $H = p + G$. }

 }

 
\section{\textcolor{Chapter }{Constructors}}\label{Chapter_Wrapper_for_generators_of_semigroups_and_hyperplane_constraints_of_cones_Section_Constructors}
\logpage{[ 5, 2, 0 ]}
\hyperdef{L}{X86EC0F0A78ECBC10}{}
{
  

\subsection{\textcolor{Chapter }{SemigroupForPresentationsByProjectiveGradedModules (for IsList, IsInt)}}
\logpage{[ 5, 2, 1 ]}\nobreak
\hyperdef{L}{X8559679E79F4E234}{}
{\noindent\textcolor{FuncColor}{$\triangleright$\enspace\texttt{SemigroupForPresentationsByProjectiveGradedModules({\mdseries\slshape L})\index{SemigroupForPresentationsByProjectiveGradedModules@\texttt{Semigroup}\-\texttt{For}\-\texttt{Presentations}\-\texttt{By}\-\texttt{Projective}\-\texttt{Graded}\-\texttt{Modules}!for IsList, IsInt}
\label{SemigroupForPresentationsByProjectiveGradedModules:for IsList, IsInt}
}\hfill{\scriptsize (operation)}}\\
\textbf{\indent Returns:\ }
a SemigroupGeneratorList 



 The argument is a list $L$ and a non-negative integer $d$. We then check if this list could be the list of generators of a subsemigroup
of $Z^d$. If so, we create the corresponding SemigroupGeneratorList. }

 

\subsection{\textcolor{Chapter }{SemigroupForPresentationsByProjectiveGradedModules (for IsList)}}
\logpage{[ 5, 2, 2 ]}\nobreak
\hyperdef{L}{X805B7049869F75B5}{}
{\noindent\textcolor{FuncColor}{$\triangleright$\enspace\texttt{SemigroupForPresentationsByProjectiveGradedModules({\mdseries\slshape arg})\index{SemigroupForPresentationsByProjectiveGradedModules@\texttt{Semigroup}\-\texttt{For}\-\texttt{Presentations}\-\texttt{By}\-\texttt{Projective}\-\texttt{Graded}\-\texttt{Modules}!for IsList}
\label{SemigroupForPresentationsByProjectiveGradedModules:for IsList}
}\hfill{\scriptsize (operation)}}\\


 

 }

 

\subsection{\textcolor{Chapter }{AffineSemigroupForPresentationsByProjectiveGradedModules (for IsSemigroupForPresentationsByProjectiveGradedModules, IsList)}}
\logpage{[ 5, 2, 3 ]}\nobreak
\hyperdef{L}{X861AD2C37B7CA62F}{}
{\noindent\textcolor{FuncColor}{$\triangleright$\enspace\texttt{AffineSemigroupForPresentationsByProjectiveGradedModules({\mdseries\slshape L, p})\index{AffineSemigroupForPresentationsByProjectiveGradedModules@\texttt{Affine}\-\texttt{Semigroup}\-\texttt{For}\-\texttt{Presentations}\-\texttt{By}\-\texttt{Projective}\-\texttt{Graded}\-\texttt{Modules}!for IsSemigroupForPresentationsByProjectiveGradedModules, IsList}
\label{AffineSemigroupForPresentationsByProjectiveGradedModules:for IsSemigroupForPresentationsByProjectiveGradedModules, IsList}
}\hfill{\scriptsize (operation)}}\\
\textbf{\indent Returns:\ }
an AffineSemigroup 



 The argument is a SemigroupForPresentationsByProjectiveGradedModules $S$ and a point $p \in \mathbb{Z}^n$ encoded as list of integers. We then compute the affine semigroup $p + S$. Alternatively to $S$ we allow the use of either a list of generators (or a list of generators
together with the embedding dimension). }

 

\subsection{\textcolor{Chapter }{AffineSemigroupForPresentationsByProjectiveGradedModules (for IsList, IsList)}}
\logpage{[ 5, 2, 4 ]}\nobreak
\hyperdef{L}{X82C8862A7F92FC2D}{}
{\noindent\textcolor{FuncColor}{$\triangleright$\enspace\texttt{AffineSemigroupForPresentationsByProjectiveGradedModules({\mdseries\slshape arg1, arg2})\index{AffineSemigroupForPresentationsByProjectiveGradedModules@\texttt{Affine}\-\texttt{Semigroup}\-\texttt{For}\-\texttt{Presentations}\-\texttt{By}\-\texttt{Projective}\-\texttt{Graded}\-\texttt{Modules}!for IsList, IsList}
\label{AffineSemigroupForPresentationsByProjectiveGradedModules:for IsList, IsList}
}\hfill{\scriptsize (operation)}}\\


 

 }

 

\subsection{\textcolor{Chapter }{AffineSemigroupForPresentationsByProjectiveGradedModules (for IsList, IsInt, IsList)}}
\logpage{[ 5, 2, 5 ]}\nobreak
\hyperdef{L}{X7CED2E788760B25F}{}
{\noindent\textcolor{FuncColor}{$\triangleright$\enspace\texttt{AffineSemigroupForPresentationsByProjectiveGradedModules({\mdseries\slshape arg1, arg2, arg3})\index{AffineSemigroupForPresentationsByProjectiveGradedModules@\texttt{Affine}\-\texttt{Semigroup}\-\texttt{For}\-\texttt{Presentations}\-\texttt{By}\-\texttt{Projective}\-\texttt{Graded}\-\texttt{Modules}!for IsList, IsInt, IsList}
\label{AffineSemigroupForPresentationsByProjectiveGradedModules:for IsList, IsInt, IsList}
}\hfill{\scriptsize (operation)}}\\


 

 }

 }

 
\section{\textcolor{Chapter }{Attributes}}\label{Chapter_Wrapper_for_generators_of_semigroups_and_hyperplane_constraints_of_cones_Section_Attributes}
\logpage{[ 5, 3, 0 ]}
\hyperdef{L}{X7C701DBF7BAE649A}{}
{
  

\subsection{\textcolor{Chapter }{GeneratorList (for IsSemigroupForPresentationsByProjectiveGradedModules)}}
\logpage{[ 5, 3, 1 ]}\nobreak
\hyperdef{L}{X84CF285781ADAEA9}{}
{\noindent\textcolor{FuncColor}{$\triangleright$\enspace\texttt{GeneratorList({\mdseries\slshape L})\index{GeneratorList@\texttt{GeneratorList}!for IsSemigroupForPresentationsByProjectiveGradedModules}
\label{GeneratorList:for IsSemigroupForPresentationsByProjectiveGradedModules}
}\hfill{\scriptsize (attribute)}}\\
\textbf{\indent Returns:\ }
a list 



 The argument is a SemigroupForPresentationsByProjectiveGradedModules $L$. We then return the list of its generators. }

 

\subsection{\textcolor{Chapter }{EmbeddingDimension (for IsSemigroupForPresentationsByProjectiveGradedModules)}}
\logpage{[ 5, 3, 2 ]}\nobreak
\hyperdef{L}{X828C7AA47AAA81AB}{}
{\noindent\textcolor{FuncColor}{$\triangleright$\enspace\texttt{EmbeddingDimension({\mdseries\slshape L})\index{EmbeddingDimension@\texttt{EmbeddingDimension}!for IsSemigroupForPresentationsByProjectiveGradedModules}
\label{EmbeddingDimension:for IsSemigroupForPresentationsByProjectiveGradedModules}
}\hfill{\scriptsize (attribute)}}\\
\textbf{\indent Returns:\ }
a non-negative integer 



 The argument is a SemigroupForPresentationsByProjectiveGradedModules $L$. We then return the embedding dimension of this semigroup. }

 

\subsection{\textcolor{Chapter }{ConeHPresentationList (for IsSemigroupForPresentationsByProjectiveGradedModules)}}
\logpage{[ 5, 3, 3 ]}\nobreak
\hyperdef{L}{X78195B828461E21F}{}
{\noindent\textcolor{FuncColor}{$\triangleright$\enspace\texttt{ConeHPresentationList({\mdseries\slshape L})\index{ConeHPresentationList@\texttt{ConeHPresentationList}!for IsSemigroupForPresentationsByProjectiveGradedModules}
\label{ConeHPresentationList:for IsSemigroupForPresentationsByProjectiveGradedModules}
}\hfill{\scriptsize (attribute)}}\\
\textbf{\indent Returns:\ }
a list or fail 



 The argument is a SemigroupForPresentationsByProjectiveGradedModules $L$. If the associated semigroup is a cone semigroup, then (during construction)
an H-presentation of that cone was computed. We return the list of the
corresponding H-constraints. This conversion uses Normaliz and can fail if the
cone if not full-dimensional. In case that such a conversion error occured,
the attribute is set to the value 'fail'. }

 

\subsection{\textcolor{Chapter }{Offset (for IsAffineSemigroupForPresentationsByProjectiveGradedModules)}}
\logpage{[ 5, 3, 4 ]}\nobreak
\hyperdef{L}{X7F7B530D87E7B9A6}{}
{\noindent\textcolor{FuncColor}{$\triangleright$\enspace\texttt{Offset({\mdseries\slshape S})\index{Offset@\texttt{Offset}!for IsAffineSemigroupForPresentationsByProjectiveGradedModules}
\label{Offset:for IsAffineSemigroupForPresentationsByProjectiveGradedModules}
}\hfill{\scriptsize (attribute)}}\\
\textbf{\indent Returns:\ }
a list of integers 



 The argument is an AffineSemigroupForPresentationsByProjectiveGradedModules $S$. This one is given as $S = p + H$ for a point $p \in \mathbb{Z}^n$ and a semigroup $H \subseteq \mathbb{Z}^n$. We then return the offset $p$. }

 

\subsection{\textcolor{Chapter }{UnderlyingSemigroup (for IsAffineSemigroupForPresentationsByProjectiveGradedModules)}}
\logpage{[ 5, 3, 5 ]}\nobreak
\hyperdef{L}{X80E1CD2D83A05640}{}
{\noindent\textcolor{FuncColor}{$\triangleright$\enspace\texttt{UnderlyingSemigroup({\mdseries\slshape S})\index{UnderlyingSemigroup@\texttt{UnderlyingSemigroup}!for IsAffineSemigroupForPresentationsByProjectiveGradedModules}
\label{UnderlyingSemigroup:for IsAffineSemigroupForPresentationsByProjectiveGradedModules}
}\hfill{\scriptsize (attribute)}}\\
\textbf{\indent Returns:\ }
a SemigroupGeneratorList 



 The argument is an IsAffineSemigroupForPresentationsByProjectiveGradedModules $S$. This one is given as $S = p + H$ for a point $p \in \mathbb{Z}^n$ and a semigroup $H \subseteq \mathbb{Z}^n$. We then return the SemigroupGeneratorList of $H$. }

 

\subsection{\textcolor{Chapter }{EmbeddingDimension (for IsAffineSemigroupForPresentationsByProjectiveGradedModules)}}
\logpage{[ 5, 3, 6 ]}\nobreak
\hyperdef{L}{X81E1AA2E7B2817CA}{}
{\noindent\textcolor{FuncColor}{$\triangleright$\enspace\texttt{EmbeddingDimension({\mdseries\slshape S})\index{EmbeddingDimension@\texttt{EmbeddingDimension}!for IsAffineSemigroupForPresentationsByProjectiveGradedModules}
\label{EmbeddingDimension:for IsAffineSemigroupForPresentationsByProjectiveGradedModules}
}\hfill{\scriptsize (attribute)}}\\
\textbf{\indent Returns:\ }
a non-negative integer 



 The argument is an IsAffineSemigroupForPresentationsByProjectiveGradedModules $S$. We then return the embedding dimension of this affine semigroup. }

 }

 
\section{\textcolor{Chapter }{Property}}\label{Chapter_Wrapper_for_generators_of_semigroups_and_hyperplane_constraints_of_cones_Section_Property}
\logpage{[ 5, 4, 0 ]}
\hyperdef{L}{X854AB144863C3BB4}{}
{
  

\subsection{\textcolor{Chapter }{IsTrivial (for IsSemigroupForPresentationsByProjectiveGradedModules)}}
\logpage{[ 5, 4, 1 ]}\nobreak
\hyperdef{L}{X87ED07E181BDB1D1}{}
{\noindent\textcolor{FuncColor}{$\triangleright$\enspace\texttt{IsTrivial({\mdseries\slshape L})\index{IsTrivial@\texttt{IsTrivial}!for IsSemigroupForPresentationsByProjectiveGradedModules}
\label{IsTrivial:for IsSemigroupForPresentationsByProjectiveGradedModules}
}\hfill{\scriptsize (property)}}\\
\textbf{\indent Returns:\ }
true or false 



 The argument is a SemigroupForPresentationsByProjectiveGradedModules $L$. This property returns 'true' if this semigroup is trivial and 'false'
otherwise. }

 

\subsection{\textcolor{Chapter }{IsSemigroupOfCone (for IsSemigroupForPresentationsByProjectiveGradedModules)}}
\logpage{[ 5, 4, 2 ]}\nobreak
\hyperdef{L}{X8461ED527CF1D822}{}
{\noindent\textcolor{FuncColor}{$\triangleright$\enspace\texttt{IsSemigroupOfCone({\mdseries\slshape L})\index{IsSemigroupOfCone@\texttt{IsSemigroupOfCone}!for IsSemigroupForPresentationsByProjectiveGradedModules}
\label{IsSemigroupOfCone:for IsSemigroupForPresentationsByProjectiveGradedModules}
}\hfill{\scriptsize (property)}}\\
\textbf{\indent Returns:\ }
true, false 



 The argument is a SemigroupForPresentationsByProjectiveGradedModules $L$. We return if this is the semigroup of a cone. }

 

\subsection{\textcolor{Chapter }{IsTrivial (for IsAffineSemigroupForPresentationsByProjectiveGradedModules)}}
\logpage{[ 5, 4, 3 ]}\nobreak
\hyperdef{L}{X786D6EFA81A30549}{}
{\noindent\textcolor{FuncColor}{$\triangleright$\enspace\texttt{IsTrivial({\mdseries\slshape L})\index{IsTrivial@\texttt{IsTrivial}!for IsAffineSemigroupForPresentationsByProjectiveGradedModules}
\label{IsTrivial:for IsAffineSemigroupForPresentationsByProjectiveGradedModules}
}\hfill{\scriptsize (property)}}\\
\textbf{\indent Returns:\ }
true or false 



 The argument is an AffineSemigroupForPresentationsByProjectiveGradedModules.
This property returns 'true' if the underlying semigroup is trivial and
otherwise 'false'. }

 

\subsection{\textcolor{Chapter }{IsAffineSemigroupOfCone (for IsAffineSemigroupForPresentationsByProjectiveGradedModules)}}
\logpage{[ 5, 4, 4 ]}\nobreak
\hyperdef{L}{X7DCACDFE80D574CD}{}
{\noindent\textcolor{FuncColor}{$\triangleright$\enspace\texttt{IsAffineSemigroupOfCone({\mdseries\slshape H})\index{IsAffineSemigroupOfCone@\texttt{IsAffineSemigroupOfCone}!for IsAffineSemigroupForPresentationsByProjectiveGradedModules}
\label{IsAffineSemigroupOfCone:for IsAffineSemigroupForPresentationsByProjectiveGradedModules}
}\hfill{\scriptsize (property)}}\\
\textbf{\indent Returns:\ }
true, false or fail 



 The argument is an IsAffineSemigroupForPresentationsByProjectiveGradedModules $H$. We return if this is an AffineConeSemigroup. If Normaliz cannot decide this
'fail' is returned. }

 }

 
\section{\textcolor{Chapter }{Operations}}\label{Chapter_Wrapper_for_generators_of_semigroups_and_hyperplane_constraints_of_cones_Section_Operations}
\logpage{[ 5, 5, 0 ]}
\hyperdef{L}{X7DE8E16C7C2D387B}{}
{
  

\subsection{\textcolor{Chapter }{DecideIfIsConeSemigroupGeneratorList (for IsList)}}
\logpage{[ 5, 5, 1 ]}\nobreak
\hyperdef{L}{X7AFD481281187217}{}
{\noindent\textcolor{FuncColor}{$\triangleright$\enspace\texttt{DecideIfIsConeSemigroupGeneratorList({\mdseries\slshape L})\index{DecideIfIsConeSemigroupGeneratorList@\texttt{Decide}\-\texttt{If}\-\texttt{Is}\-\texttt{Cone}\-\texttt{Semigroup}\-\texttt{Generator}\-\texttt{List}!for IsList}
\label{DecideIfIsConeSemigroupGeneratorList:for IsList}
}\hfill{\scriptsize (operation)}}\\
\textbf{\indent Returns:\ }
true, false or fail 



 The argument is a list $L$ of generators of a semigroup in $\mathbb{Z}^n$. We then check if this is the semigroup of a cone. In this case we return
'true', otherwise 'false'. If the operation fails due to shortcommings in
Normaliz we return 'fail'. }

 }

 
\section{\textcolor{Chapter }{Check if point is contained in (affine) cone or (affine ) semigroup}}\label{Chapter_Wrapper_for_generators_of_semigroups_and_hyperplane_constraints_of_cones_Section_Check_if_point_is_contained_in_affine_cone_or_affine__semigroup}
\logpage{[ 5, 6, 0 ]}
\hyperdef{L}{X783CFB8C84B9BD6A}{}
{
  

\subsection{\textcolor{Chapter }{PointContainedInSemigroup (for IsSemigroupForPresentationsByProjectiveGradedModules, IsList)}}
\logpage{[ 5, 6, 1 ]}\nobreak
\hyperdef{L}{X82A214FE7B77E980}{}
{\noindent\textcolor{FuncColor}{$\triangleright$\enspace\texttt{PointContainedInSemigroup({\mdseries\slshape S, p})\index{PointContainedInSemigroup@\texttt{PointContainedInSemigroup}!for IsSemigroupForPresentationsByProjectiveGradedModules, IsList}
\label{PointContainedInSemigroup:for IsSemigroupForPresentationsByProjectiveGradedModules, IsList}
}\hfill{\scriptsize (operation)}}\\
\textbf{\indent Returns:\ }
true or false 



 The argument is a SemigroupForPresentationsByProjectiveGradedModules $S$ of $\mathbb{Z}^n$, and an integral point $p$ in this lattice. This operation then verifies if the point $p$ is contained in $S$ or not. }

 

\subsection{\textcolor{Chapter }{PointContainedInAffineSemigroup (for IsAffineSemigroupForPresentationsByProjectiveGradedModules, IsList)}}
\logpage{[ 5, 6, 2 ]}\nobreak
\hyperdef{L}{X87B1A82385155514}{}
{\noindent\textcolor{FuncColor}{$\triangleright$\enspace\texttt{PointContainedInAffineSemigroup({\mdseries\slshape H, p})\index{PointContainedInAffineSemigroup@\texttt{PointContainedInAffineSemigroup}!for IsAffineSemigroupForPresentationsByProjectiveGradedModules, IsList}
\label{PointContainedInAffineSemigroup:for IsAffineSemigroupForPresentationsByProjectiveGradedModules, IsList}
}\hfill{\scriptsize (operation)}}\\
\textbf{\indent Returns:\ }
true or false 



 The argument is an IsAffineSemigroupForPresentationsByProjectiveGradedModules $H$ and a point $p$. The second argument This method then checks if $p$ lies in $H$. }

 }

 
\section{\textcolor{Chapter }{Examples}}\label{Chapter_Wrapper_for_generators_of_semigroups_and_hyperplane_constraints_of_cones_Section_Examples}
\logpage{[ 5, 7, 0 ]}
\hyperdef{L}{X7A489A5D79DA9E5C}{}
{
  The following commands are used to handle generators of semigroups in $\mathbb{Z}^n$, generators of cones in $\mathbb{Z}^n$ as well as hyperplane constraints that define cones in $\mathbb{Z}^n$. Here are some examples: 
\begin{Verbatim}[commandchars=!@|,fontsize=\small,frame=single,label=Example]
  !gapprompt@gap>| !gapinput@semigroup1 := SemigroupForPresentationsByProjectiveGradedModules(|
  !gapprompt@>| !gapinput@              [[ 1,0 ], [ 1,1 ]] );|
  <A cone-semigroup in Z^2 formed as the span of 2 generators>
  !gapprompt@gap>| !gapinput@IsSemigroupForPresentationsByProjectiveGradedModules( semigroup1 );|
  true
  !gapprompt@gap>| !gapinput@GeneratorList( semigroup1 );|
  [ [ 1, 0 ], [ 1, 1 ] ]
  !gapprompt@gap>| !gapinput@semigroup2 := SemigroupForPresentationsByProjectiveGradedModules(|
  !gapprompt@>| !gapinput@              [[ 2,0 ], [ 1,1 ]] );|
  <A non-cone semigroup in Z^2 formed as the span of 2 generators>
  !gapprompt@gap>| !gapinput@IsSemigroupForPresentationsByProjectiveGradedModules( semigroup2 );|
  true
  !gapprompt@gap>| !gapinput@GeneratorList( semigroup2 );|
  [ [ 2, 0 ], [ 1, 1 ] ]
\end{Verbatim}
 We can check if a semigroup in $\mathbb{Z}^n$ is the semigroup of a cone. In case we can look at an H-presentation of this
cone. 
\begin{Verbatim}[commandchars=!@|,fontsize=\small,frame=single,label=Example]
  !gapprompt@gap>| !gapinput@IsSemigroupOfCone( semigroup1 );|
  true
  !gapprompt@gap>| !gapinput@ConeHPresentationList( semigroup1 );|
  [ [ 0, 1 ], [ 1, -1 ] ]
  !gapprompt@gap>| !gapinput@Display( ConeHPresentationList( semigroup1 ) );|
  [ [   0,  1 ],
    [   1, -1 ] ]
  !gapprompt@gap>| !gapinput@IsSemigroupOfCone( semigroup2 );|
  false
  !gapprompt@gap>| !gapinput@HasConeHPresentationList( semigroup2 );|
  false
\end{Verbatim}
 We can check membership of points in semigroups. 
\begin{Verbatim}[commandchars=!@|,fontsize=\small,frame=single,label=Example]
  !gapprompt@gap>| !gapinput@PointContainedInSemigroup( semigroup2, [ 1,0 ] );|
  false
  !gapprompt@gap>| !gapinput@PointContainedInSemigroup( semigroup2, [ 2,0 ] );|
  true
\end{Verbatim}
 Given a semigroup $S \subseteq \mathbb{Z}^n$ and a point $p \in \mathbb{Z}^n$ we can consider 
\[ H := p + S = \left\{ p + x \; , \; x \in S \right\}. \]
 We term this an affine semigroup. Given that $S = C \cap \mathbb{Z}^n$ for a cone $C \subseteq \mathbb{Z}^n$, we use the term affine cone{\textunderscore}semigroup. The constructors are
as follows: 
\begin{Verbatim}[commandchars=!@|,fontsize=\small,frame=single,label=Example]
  !gapprompt@gap>| !gapinput@affine_semigroup1 := AffineSemigroupForPresentationsByProjectiveGradedModules(|
  !gapprompt@>| !gapinput@                     semigroup1, [ -1, -1 ] );|
  <A non-trivial affine cone-semigroup in Z^2>
  !gapprompt@gap>| !gapinput@affine_semigroup2 := AffineSemigroupForPresentationsByProjectiveGradedModules(|
  !gapprompt@>| !gapinput@                     semigroup2, [ 2, 2 ] );|
  <A non-trivial affine non-cone semigroup in Z^2>
\end{Verbatim}
 We can access the properties of these affine semigroups as follows. 
\begin{Verbatim}[commandchars=!@|,fontsize=\small,frame=single,label=Example]
  !gapprompt@gap>| !gapinput@IsAffineSemigroupOfCone( affine_semigroup2 );|
  false
  !gapprompt@gap>| !gapinput@UnderlyingSemigroup( affine_semigroup2 );|
  <A non-cone semigroup in Z^2 formed as the span of 2 generators>
  !gapprompt@gap>| !gapinput@Display( UnderlyingSemigroup( affine_semigroup2 ) );|
  A non-cone semigroup in Z^2 formed as the span of 2 generators -
  generators are as follows:
  [ [  2,  0 ],
    [  1,  1 ] ]
  !gapprompt@gap>| !gapinput@IsAffineSemigroupOfCone( affine_semigroup1 );|
  true
  !gapprompt@gap>| !gapinput@Offset( affine_semigroup2 );|
  [ 2, 2 ]
  !gapprompt@gap>| !gapinput@ConeHPresentationList( UnderlyingSemigroup( affine_semigroup1 ) );|
  [ [ 0, 1 ], [ 1, -1 ] ]
\end{Verbatim}
 Of course we can also decide membership in affine
(cone{\textunderscore})semigroups. 
\begin{Verbatim}[commandchars=!@|,fontsize=\small,frame=single,label=Example]
  !gapprompt@gap>| !gapinput@Display( affine_semigroup1 );|
  A non-trivial affine cone-semigroup in Z^2
  Offset: [ -1, -1 ]
  Hilbert basis: [ [ 1, 0 ], [ 1, 1 ] ]
  !gapprompt@gap>| !gapinput@PointContainedInAffineSemigroup( affine_semigroup1, [ -2,-2 ] );|
  false
  !gapprompt@gap>| !gapinput@PointContainedInAffineSemigroup( affine_semigroup1, [ 3,1 ] );|
  true
  !gapprompt@gap>| !gapinput@Display( affine_semigroup2 );|
  A non-trivial affine non-cone semigroup in Z^2
  Offset: [ 2, 2 ]
  Semigroup generators: [ [ 2, 0 ], [ 1, 1 ] ]
  !gapprompt@gap>| !gapinput@PointContainedInAffineSemigroup( affine_semigroup2, [ 3,2 ] );|
  false
  !gapprompt@gap>| !gapinput@PointContainedInAffineSemigroup( affine_semigroup2, [ 3,3 ] );|
  true
\end{Verbatim}
 }

 }

   
\chapter{\textcolor{Chapter }{Vanishing sets on toric varieties}}\label{Chapter_Vanishing_sets_on_toric_varieties}
\logpage{[ 6, 0, 0 ]}
\hyperdef{L}{X8201C4637F14E99F}{}
{
  
\section{\textcolor{Chapter }{GAP category for vanishing sets}}\label{Chapter_Vanishing_sets_on_toric_varieties_Section_GAP_category_for_vanishing_sets}
\logpage{[ 6, 1, 0 ]}
\hyperdef{L}{X819265EC7FB8AF8A}{}
{
  

\subsection{\textcolor{Chapter }{IsVanishingSet (for IsObject)}}
\logpage{[ 6, 1, 1 ]}\nobreak
\hyperdef{L}{X7E1CB2C07F4E4A77}{}
{\noindent\textcolor{FuncColor}{$\triangleright$\enspace\texttt{IsVanishingSet({\mdseries\slshape arg})\index{IsVanishingSet@\texttt{IsVanishingSet}!for IsObject}
\label{IsVanishingSet:for IsObject}
}\hfill{\scriptsize (filter)}}\\
\textbf{\indent Returns:\ }
\texttt{true} or \texttt{false} 



 The GAP category of vanishing sets formed from affine semigroups. }

 }

 
\section{\textcolor{Chapter }{Constructors}}\label{Chapter_Vanishing_sets_on_toric_varieties_Section_Constructors}
\logpage{[ 6, 2, 0 ]}
\hyperdef{L}{X86EC0F0A78ECBC10}{}
{
  

\subsection{\textcolor{Chapter }{VanishingSet (for IsToricVariety, IsList, IsInt)}}
\logpage{[ 6, 2, 1 ]}\nobreak
\hyperdef{L}{X7B537B427EE545B5}{}
{\noindent\textcolor{FuncColor}{$\triangleright$\enspace\texttt{VanishingSet({\mdseries\slshape variety, L, d, i, or, s})\index{VanishingSet@\texttt{VanishingSet}!for IsToricVariety, IsList, IsInt}
\label{VanishingSet:for IsToricVariety, IsList, IsInt}
}\hfill{\scriptsize (operation)}}\\
\textbf{\indent Returns:\ }
a vanishing set 



 The argument is a toric variety, a list $L$ of AffineSemigroups and the cohomological index $i$. Alternatively a string $s$ can be used instead of $d$ to inform the user for which cohomology classes this set identifies the
'vanishing twists'. }

 

\subsection{\textcolor{Chapter }{VanishingSet (for IsToricVariety, IsList, IsString)}}
\logpage{[ 6, 2, 2 ]}\nobreak
\hyperdef{L}{X7C2813B97E1B2309}{}
{\noindent\textcolor{FuncColor}{$\triangleright$\enspace\texttt{VanishingSet({\mdseries\slshape arg1, arg2, arg3})\index{VanishingSet@\texttt{VanishingSet}!for IsToricVariety, IsList, IsString}
\label{VanishingSet:for IsToricVariety, IsList, IsString}
}\hfill{\scriptsize (operation)}}\\


 

 }

 }

 
\section{\textcolor{Chapter }{Attributes}}\label{Chapter_Vanishing_sets_on_toric_varieties_Section_Attributes}
\logpage{[ 6, 3, 0 ]}
\hyperdef{L}{X7C701DBF7BAE649A}{}
{
  

\subsection{\textcolor{Chapter }{ListOfUnderlyingAffineSemigroups (for IsVanishingSet)}}
\logpage{[ 6, 3, 1 ]}\nobreak
\hyperdef{L}{X849687377EF1A2BE}{}
{\noindent\textcolor{FuncColor}{$\triangleright$\enspace\texttt{ListOfUnderlyingAffineSemigroups({\mdseries\slshape V})\index{ListOfUnderlyingAffineSemigroups@\texttt{ListOfUnderlyingAffineSemigroups}!for IsVanishingSet}
\label{ListOfUnderlyingAffineSemigroups:for IsVanishingSet}
}\hfill{\scriptsize (attribute)}}\\
\textbf{\indent Returns:\ }
a list of affine semigroups 



 The argument is a vanishingSet $V$. We then return the underlying list of semigroups that form this vanishing
set. }

 

\subsection{\textcolor{Chapter }{EmbeddingDimension (for IsVanishingSet)}}
\logpage{[ 6, 3, 2 ]}\nobreak
\hyperdef{L}{X87BD035D80CE31DF}{}
{\noindent\textcolor{FuncColor}{$\triangleright$\enspace\texttt{EmbeddingDimension({\mdseries\slshape V})\index{EmbeddingDimension@\texttt{EmbeddingDimension}!for IsVanishingSet}
\label{EmbeddingDimension:for IsVanishingSet}
}\hfill{\scriptsize (attribute)}}\\
\textbf{\indent Returns:\ }
a non-negative integer 



 The argument is a vanishingSet $V$. We then return the embedding dimension of this vanishing set. }

 

\subsection{\textcolor{Chapter }{CohomologicalIndex (for IsVanishingSet)}}
\logpage{[ 6, 3, 3 ]}\nobreak
\hyperdef{L}{X860662B581F0FAC3}{}
{\noindent\textcolor{FuncColor}{$\triangleright$\enspace\texttt{CohomologicalIndex({\mdseries\slshape V})\index{CohomologicalIndex@\texttt{CohomologicalIndex}!for IsVanishingSet}
\label{CohomologicalIndex:for IsVanishingSet}
}\hfill{\scriptsize (attribute)}}\\
\textbf{\indent Returns:\ }
an integer between $0$ and $dim \left( X_\Sigma \right)$ 



 The argument is a vanishingSet $V$. This vanishing set identifies those $D \in Pic \left( X_\Sigma \right)$ such that $H^i \left( X_\Sigma, \mathcal{O} \left( D \right) \right) = 0$. We return the integer $i$. }

 

\subsection{\textcolor{Chapter }{CohomologicalSpecification (for IsVanishingSet)}}
\logpage{[ 6, 3, 4 ]}\nobreak
\hyperdef{L}{X8769C3577B4C84CE}{}
{\noindent\textcolor{FuncColor}{$\triangleright$\enspace\texttt{CohomologicalSpecification({\mdseries\slshape V})\index{CohomologicalSpecification@\texttt{CohomologicalSpecification}!for IsVanishingSet}
\label{CohomologicalSpecification:for IsVanishingSet}
}\hfill{\scriptsize (attribute)}}\\
\textbf{\indent Returns:\ }
a string 



 The argument is a vanishingSet $V$. This could for example identify those $D \in Pic \left( X_\Sigma \right)$ such that $H^i \left( X_\Sigma, \mathcal{O} \left( D \right) \right) = 0$ for all $i > 0$. If such a specification is known, it will be returned by this method. }

 

\subsection{\textcolor{Chapter }{AmbientToricVariety (for IsVanishingSet)}}
\logpage{[ 6, 3, 5 ]}\nobreak
\hyperdef{L}{X809900B980D8E310}{}
{\noindent\textcolor{FuncColor}{$\triangleright$\enspace\texttt{AmbientToricVariety({\mdseries\slshape V})\index{AmbientToricVariety@\texttt{AmbientToricVariety}!for IsVanishingSet}
\label{AmbientToricVariety:for IsVanishingSet}
}\hfill{\scriptsize (attribute)}}\\


 The argument is a vanishingSet $V$. We return the toric variety to which this vanishing set belongs. }

 }

 
\section{\textcolor{Chapter }{Property}}\label{Chapter_Vanishing_sets_on_toric_varieties_Section_Property}
\logpage{[ 6, 4, 0 ]}
\hyperdef{L}{X854AB144863C3BB4}{}
{
  

\subsection{\textcolor{Chapter }{IsFull (for IsVanishingSet)}}
\logpage{[ 6, 4, 1 ]}\nobreak
\hyperdef{L}{X84BA529978699063}{}
{\noindent\textcolor{FuncColor}{$\triangleright$\enspace\texttt{IsFull({\mdseries\slshape V})\index{IsFull@\texttt{IsFull}!for IsVanishingSet}
\label{IsFull:for IsVanishingSet}
}\hfill{\scriptsize (property)}}\\
\textbf{\indent Returns:\ }
\texttt{true} or \texttt{false} 



 The argument is a VanishingSet $V$. We then check if this vanishing set is empty. }

 }

 
\section{\textcolor{Chapter }{Improved vanishing sets via cohomCalg}}\label{Chapter_Vanishing_sets_on_toric_varieties_Section_Improved_vanishing_sets_via_cohomCalg}
\logpage{[ 6, 5, 0 ]}
\hyperdef{L}{X7FA1F61E7A6C2B44}{}
{
  

 

\subsection{\textcolor{Chapter }{TurnDenominatorIntoShiftedSemigroup (for IsToricVariety, IsString)}}
\logpage{[ 6, 5, 1 ]}\nobreak
\hyperdef{L}{X818861BB7D57AEF2}{}
{\noindent\textcolor{FuncColor}{$\triangleright$\enspace\texttt{TurnDenominatorIntoShiftedSemigroup({\mdseries\slshape arg1, arg2})\index{TurnDenominatorIntoShiftedSemigroup@\texttt{TurnDenominatorIntoShiftedSemigroup}!for IsToricVariety, IsString}
\label{TurnDenominatorIntoShiftedSemigroup:for IsToricVariety, IsString}
}\hfill{\scriptsize (operation)}}\\


 

 }

 

 

\subsection{\textcolor{Chapter }{VanishingSets (for IsToricVariety)}}
\logpage{[ 6, 5, 2 ]}\nobreak
\hyperdef{L}{X81D251E67C3C1FBB}{}
{\noindent\textcolor{FuncColor}{$\triangleright$\enspace\texttt{VanishingSets({\mdseries\slshape arg})\index{VanishingSets@\texttt{VanishingSets}!for IsToricVariety}
\label{VanishingSets:for IsToricVariety}
}\hfill{\scriptsize (attribute)}}\\


 

 }

 

 

\subsection{\textcolor{Chapter }{ComputeVanishingSets (for IsToricVariety, IsBool)}}
\logpage{[ 6, 5, 3 ]}\nobreak
\hyperdef{L}{X872B02807E62E84A}{}
{\noindent\textcolor{FuncColor}{$\triangleright$\enspace\texttt{ComputeVanishingSets({\mdseries\slshape arg1, arg2})\index{ComputeVanishingSets@\texttt{ComputeVanishingSets}!for IsToricVariety, IsBool}
\label{ComputeVanishingSets:for IsToricVariety, IsBool}
}\hfill{\scriptsize (operation)}}\\


 

 }

 

 

\subsection{\textcolor{Chapter }{PointContainedInVanishingSet (for IsVanishingSet, IsList)}}
\logpage{[ 6, 5, 4 ]}\nobreak
\hyperdef{L}{X82088A1D83343227}{}
{\noindent\textcolor{FuncColor}{$\triangleright$\enspace\texttt{PointContainedInVanishingSet({\mdseries\slshape arg1, arg2})\index{PointContainedInVanishingSet@\texttt{PointContainedInVanishingSet}!for IsVanishingSet, IsList}
\label{PointContainedInVanishingSet:for IsVanishingSet, IsList}
}\hfill{\scriptsize (operation)}}\\


 

 }

 }

 
\section{\textcolor{Chapter }{Examples}}\label{Chapter_Vanishing_sets_on_toric_varieties_Section_Examples}
\logpage{[ 6, 6, 0 ]}
\hyperdef{L}{X7A489A5D79DA9E5C}{}
{
  
\begin{Verbatim}[commandchars=!@B,fontsize=\small,frame=single,label=Example]
  !gapprompt@gap>B !gapinput@F1 := Fan( [[1],[-1]],[[1],[2]] );B
  <A fan in |R^1>
  !gapprompt@gap>B !gapinput@P1 := ToricVariety( F1 );B
  <A toric variety of dimension 1>
  !gapprompt@gap>B !gapinput@v1 := VanishingSets( P1 );B
  rec( 0 := <A non-full vanishing set in Z^1 for cohomological index 0>,
  1 := <A non-full vanishing set in Z^1 for cohomological index 1> )
  !gapprompt@gap>B !gapinput@Display( v1.0 );B
  A non-full vanishing set in Z^1 for cohomological index 0 formed from
  the points NOT contained in the following affine semigroup:
  
  A non-trivial affine cone-semigroup in Z^1
  Offset: [ 0 ]
  Hilbert basis: [ [ 1 ] ]
  !gapprompt@gap>B !gapinput@Display( v1.1 );B
  A non-full vanishing set in Z^1 for cohomological index 1 formed from
  the points NOT contained in the following affine semigroup:
  
  A non-trivial affine cone-semigroup in Z^1
  Offset: [ -2 ]
  Hilbert basis: [ [ -1 ] ]
  !gapprompt@gap>B !gapinput@P1xP1 := P1*P1;B
  <A projective smooth toric variety of dimension
  2 which is a product of 2 toric varieties>
  !gapprompt@gap>B !gapinput@v2 := VanishingSets( P1xP1 );B
  rec( 0 := <A non-full vanishing set in Z^2 for cohomological index 0>,
       1 := <A non-full vanishing set in Z^2 for cohomological index 1>,
       2 := <A non-full vanishing set in Z^2 for cohomological index 2> )
  !gapprompt@gap>B !gapinput@Display( v2.0 );B
  A non-full vanishing set in Z^2 for cohomological index 0 formed from
  the points NOT contained in the following affine semigroup:
  
  A non-trivial affine cone-semigroup in Z^2
  Offset: [ 0, 0 ]
  Hilbert basis: [ [ 0, 1 ], [ 1, 0 ] ]
  !gapprompt@gap>B !gapinput@Display( v2.1 );B
  A non-full vanishing set in Z^2 for cohomological index 1 formed from
  the points NOT contained in the following 2 affine semigroups:
  
  Affine semigroup 1: 
  A non-trivial affine cone-semigroup in Z^2
  Offset: [ 0, -2 ]
  Hilbert basis: [ [ 0, -1 ], [ 1, 0 ] ]
  
  Affine semigroup 2: 
  A non-trivial affine cone-semigroup in Z^2
  Offset: [ -2, 0 ]
  Hilbert basis: [ [ 0, 1 ], [ -1, 0 ] ]
  !gapprompt@gap>B !gapinput@Display( v2.2 );B
  A non-full vanishing set in Z^2 for cohomological index 2 formed from 
  the points NOT contained in the following affine semigroup: 
  
  A non-trivial affine cone-semigroup in Z^2
  Offset: [ -2, -2 ]
  Hilbert basis: [ [ 0, -1 ], [ -1, 0 ] ]
  !gapprompt@gap>B !gapinput@P2 := ProjectiveSpace( 2 );B
  <A projective toric variety of dimension 2>
  !gapprompt@gap>B !gapinput@v3 := VanishingSets( P2 );B
  rec( 0 := <A non-full vanishing set in Z^1 for cohomological index 0>,
       1 := <A full vanishing set in Z^1 for cohomological index 1>,
       2 := <A non-full vanishing set in Z^1 for cohomological index 2> )
  !gapprompt@gap>B !gapinput@P2xP1xP1 := P2*P1*P1;B
  <A projective smooth toric variety of dimension
  4 which is a product of 3 toric varieties>
  !gapprompt@gap>B !gapinput@v4 := VanishingSets( P2xP1xP1 );B
  rec( 0 := <A non-full vanishing set in Z^3 for cohomological index 0>,
       1 := <A non-full vanishing set in Z^3 for cohomological index 1>,
       2 := <A non-full vanishing set in Z^3 for cohomological index 2>,
       3 := <A non-full vanishing set in Z^3 for cohomological index 3>,
       4 := <A non-full vanishing set in Z^3 for cohomological index 4> )
  !gapprompt@gap>B !gapinput@P := Polytope( [[ -2,2],[1,2],[2,1],[2,-2],[-2,-2]] );B
  <A polytope in |R^2>
  !gapprompt@gap>B !gapinput@T := ToricVariety( P );B
  <A projective toric variety of dimension 2>
  !gapprompt@gap>B !gapinput@v5 := VanishingSets( T );B
  rec( 0 := <A non-full vanishing set in Z^3 for cohomological index 0>,
       1 := <A non-full vanishing set in Z^3 for cohomological index 1>,
       2 := <A non-full vanishing set in Z^3 for cohomological index 2> )
  !gapprompt@gap>B !gapinput@Display( v5.2 );B
  A non-full vanishing set in Z^3 for cohomological index 2 formed from 
  the points NOT contained in the following affine semigroup: 
  
  A non-trivial affine cone-semigroup in Z^3
  Offset: [ -1, -2, -1 ]
  Hilbert basis: [ [ 1, 0, -1 ], [ -1, 0, 0 ], [ -1, -1, 1 ], [ 0, -1, 0 ],
  [ 0, 0, -1 ] ]
  !gapprompt@gap>B !gapinput@H7 := Fan( [[0,1],[1,0],[0,-1],[-1,7]], [[1,2],[2,3],[3,4],[4,1]] );B
  <A fan in |R^2>
  !gapprompt@gap>B !gapinput@H7 := ToricVariety( H7 );B
  <A toric variety of dimension 2>
  !gapprompt@gap>B !gapinput@v6 := VanishingSets( H7 );B
  rec( 0 := <A non-full vanishing set in Z^2 for cohomological index 0>,
       1 := <A non-full vanishing set in Z^2 for cohomological index 1>,
       2 := <A non-full vanishing set in Z^2 for cohomological index 2> )
  !gapprompt@gap>B !gapinput@H5 := Fan( [[-1,5],[0,1],[1,0],[0,-1]] ,[[1,2],[2,3],[3,4],[4,1]] );B
  <A fan in |R^2>
  !gapprompt@gap>B !gapinput@H5 := ToricVariety( H5 );B
  <A toric variety of dimension 2>
  !gapprompt@gap>B !gapinput@v7 := VanishingSets( H5 );B
  rec( 0 := <A non-full vanishing set in Z^2 for cohomological index 0>,
       1 := <A non-full vanishing set in Z^2 for cohomological index 1>,
       2 := <A non-full vanishing set in Z^2 for cohomological index 2> )
  !gapprompt@gap>B !gapinput@PointContainedInVanishingSet( v1.0, [ 1 ] );B
  false
  !gapprompt@gap>B !gapinput@PointContainedInVanishingSet( v1.0, [ 0 ] );B
  false
  !gapprompt@gap>B !gapinput@PointContainedInVanishingSet( v1.0, [ -1 ] );B
  true
  !gapprompt@gap>B !gapinput@PointContainedInVanishingSet( v1.0, [ -2 ] );B
  true
  !gapprompt@gap>B !gapinput@rays := [ [1,0,0], [-1,0,0], [0,1,0], [0,-1,0], [0,0,1], [0,0,-1],B
  !gapprompt@>B !gapinput@          [2,1,1], [1,2,1], [1,1,2], [1,1,1] ];B
  [ [ 1, 0, 0 ], [ -1, 0, 0 ], [ 0, 1, 0 ], [ 0, -1, 0 ], [ 0, 0, 1 ], [ 0, 0, -1 ], 
  [ 2, 1, 1 ], [ 1, 2, 1 ], [ 1, 1, 2 ], [ 1, 1, 1 ] ]
  !gapprompt@gap>B !gapinput@cones := [ [1,3,6], [1,4,6], [1,4,5], [2,3,6], [2,4,6], [2,3,5], [2,4,5],B
  !gapprompt@>B !gapinput@           [1,5,9], [3,5,8], [1,3,7], [1,7,9], [5,8,9], [3,7,8],B
  !gapprompt@>B !gapinput@           [7,9,10], [8,9,10], [7,8,10] ];B
  [ [ 1, 3, 6 ], [ 1, 4, 6 ], [ 1, 4, 5 ], [ 2, 3, 6 ], [ 2, 4, 6 ], [ 2, 3, 5 ],
    [ 2, 4, 5 ], [ 1, 5, 9 ], [ 3, 5, 8 ], [ 1, 3, 7 ], [ 1, 7, 9 ], [ 5, 8, 9 ], 
    [ 3, 7, 8 ], [ 7, 9, 10 ], [ 8, 9, 10 ], [ 7, 8, 10 ] ]
  !gapprompt@gap>B !gapinput@F := Fan( rays, cones );B
  <A fan in |R^3>
  !gapprompt@gap>B !gapinput@T := ToricVariety( F );B
  <A toric variety of dimension 3>
  !gapprompt@gap>B !gapinput@[ IsSmooth( T ), IsComplete( T ), IsProjective( T ) ];B
  [ true, true, false ]
  !gapprompt@gap>B !gapinput@SRIdeal( T );B
  <A graded torsion-free (left) ideal given by 23 generators>
  !gapprompt@gap>B !gapinput@v8 := VanishingSets( T );B
  rec( 0 := <A non-full vanishing set in Z^7 for cohomological index 0>,
       1 := <A non-full vanishing set in Z^7 for cohomological index 1>,
       2 := <A non-full vanishing set in Z^7 for cohomological index 2>,
       3 := <A non-full vanishing set in Z^7 for cohomological index 3> )
  !gapprompt@gap>B !gapinput@Display( v8.3 );B
  A non-full vanishing set in Z^7 for cohomological index 3 formed from \
  the points NOT contained in the following affine semigroup: 
  
  A non-trivial affine cone-semigroup in Z^7
  Offset: [ -2, -2, -2, -2, -1, -3, -3 ]
  Hilbert basis: [ [ 0, 0, -1, -1, -1, -1, -2 ], [ 0, -1, 0, -1, -1, -2,\
   -1 ], [ -1, 0, 0, 0, 0, 0, 0 ], [ -1, 0, 0, 1, 2, 1, 1 ], [ 0, -1, 0,\
   0, 0, 0, 0 ], [ 0, 0, -1, 0, 0, 0, 0 ], [ 0, 0, 0, -1, 0, 0, 0 ], [ 0\
  , 0, 0, 0, -1, 0, 0 ], [ 0, 0, 0, 0, 0, -1, 0 ], [ 0, 0, 0, 0, 0, 0, -\
  1 ] ]
\end{Verbatim}
 }

 }

   
\chapter{\textcolor{Chapter }{Irreducible, complete, torus-invariant curves and proper 1-cycles in a toric
variety}}\label{Chapter_Irreducible_complete_torus-invariant_curves_and_proper_1-cycles_in_a_toric_variety}
\logpage{[ 7, 0, 0 ]}
\hyperdef{L}{X7CCC40FA815A55EA}{}
{
  
\section{\textcolor{Chapter }{GAP category of irreducible, complete, torus-invariant curves (= ICT curves)}}\label{Chapter_Irreducible_complete_torus-invariant_curves_and_proper_1-cycles_in_a_toric_variety_Section_GAP_category_of_irreducible_complete_torus-invariant_curves__ICT_curves}
\logpage{[ 7, 1, 0 ]}
\hyperdef{L}{X830F58607B4D4C12}{}
{
  

\subsection{\textcolor{Chapter }{IsICTCurve (for IsObject)}}
\logpage{[ 7, 1, 1 ]}\nobreak
\hyperdef{L}{X850B21658166706F}{}
{\noindent\textcolor{FuncColor}{$\triangleright$\enspace\texttt{IsICTCurve({\mdseries\slshape object})\index{IsICTCurve@\texttt{IsICTCurve}!for IsObject}
\label{IsICTCurve:for IsObject}
}\hfill{\scriptsize (filter)}}\\
\textbf{\indent Returns:\ }
true or false 



 The GAP category for irreducible, complete, torus-invariant curves }

 }

 
\section{\textcolor{Chapter }{Constructors for ICT Curves}}\label{Chapter_Irreducible_complete_torus-invariant_curves_and_proper_1-cycles_in_a_toric_variety_Section_Constructors_for_ICT_Curves}
\logpage{[ 7, 2, 0 ]}
\hyperdef{L}{X8553801780AB31D2}{}
{
  

\subsection{\textcolor{Chapter }{ICTCurve (for IsToricVariety, IsInt, IsInt)}}
\logpage{[ 7, 2, 1 ]}\nobreak
\hyperdef{L}{X7B4CD31E8173B4BC}{}
{\noindent\textcolor{FuncColor}{$\triangleright$\enspace\texttt{ICTCurve({\mdseries\slshape X{\textunderscore}Sigma, i, j})\index{ICTCurve@\texttt{ICTCurve}!for IsToricVariety, IsInt, IsInt}
\label{ICTCurve:for IsToricVariety, IsInt, IsInt}
}\hfill{\scriptsize (operation)}}\\
\textbf{\indent Returns:\ }
an ICT curve 



 The arguments are a smooth and complete toric variety $X_\Sigma$ and two non-negative and distinct integers $i,j$. We then consider the i-th and j-th maximal cones $\sigma_i$ and $\sigma_j$. ! If $\tau := \sigma_i \cap \sigma_j$ satisfies $dim \left( \tau \right) = dim \left( \sigma_1 \right) - 1$, then $\tau$ corresponds to an ICT-curve. We then construct this very ICT-curve. }

 }

 
\section{\textcolor{Chapter }{Attributes for ICT curves}}\label{Chapter_Irreducible_complete_torus-invariant_curves_and_proper_1-cycles_in_a_toric_variety_Section_Attributes_for_ICT_curves}
\logpage{[ 7, 3, 0 ]}
\hyperdef{L}{X7B35A89482BC8E89}{}
{
  

\subsection{\textcolor{Chapter }{AmbientToricVariety (for IsICTCurve)}}
\logpage{[ 7, 3, 1 ]}\nobreak
\hyperdef{L}{X820B4CFF7B90E4A1}{}
{\noindent\textcolor{FuncColor}{$\triangleright$\enspace\texttt{AmbientToricVariety({\mdseries\slshape C})\index{AmbientToricVariety@\texttt{AmbientToricVariety}!for IsICTCurve}
\label{AmbientToricVariety:for IsICTCurve}
}\hfill{\scriptsize (attribute)}}\\
\textbf{\indent Returns:\ }
a toric variety 



 The argument is an ICT curve $C$. The output is the toric variety, in which this curve $C$ lies. }

 

\subsection{\textcolor{Chapter }{IntersectedMaximalCones (for IsICTCurve)}}
\logpage{[ 7, 3, 2 ]}\nobreak
\hyperdef{L}{X7931E28B7BBAE126}{}
{\noindent\textcolor{FuncColor}{$\triangleright$\enspace\texttt{IntersectedMaximalCones({\mdseries\slshape C})\index{IntersectedMaximalCones@\texttt{IntersectedMaximalCones}!for IsICTCurve}
\label{IntersectedMaximalCones:for IsICTCurve}
}\hfill{\scriptsize (attribute)}}\\
\textbf{\indent Returns:\ }
a list of two positive and distinct integers 



 The argument is an ICT curve $C$. The output are two integers, which indicate which maximal rays were
intersected to form the cone $\tau$ associated to this curve $C$. }

 

\subsection{\textcolor{Chapter }{RayGenerators (for IsICTCurve)}}
\logpage{[ 7, 3, 3 ]}\nobreak
\hyperdef{L}{X7F73DF2F846C208E}{}
{\noindent\textcolor{FuncColor}{$\triangleright$\enspace\texttt{RayGenerators({\mdseries\slshape C})\index{RayGenerators@\texttt{RayGenerators}!for IsICTCurve}
\label{RayGenerators:for IsICTCurve}
}\hfill{\scriptsize (attribute)}}\\
\textbf{\indent Returns:\ }
a list of lists of integers 



 The argument is an ICT curve $C$. The output is the list of ray-generators for the cone tau }

 

\subsection{\textcolor{Chapter }{DefiningVariables (for IsICTCurve)}}
\logpage{[ 7, 3, 4 ]}\nobreak
\hyperdef{L}{X80EE138281563E7D}{}
{\noindent\textcolor{FuncColor}{$\triangleright$\enspace\texttt{DefiningVariables({\mdseries\slshape C})\index{DefiningVariables@\texttt{DefiningVariables}!for IsICTCurve}
\label{DefiningVariables:for IsICTCurve}
}\hfill{\scriptsize (attribute)}}\\
\textbf{\indent Returns:\ }
a list 



 The argument is an ICT curve $C$. The output is the list of variables whose simultaneous vanishing locus cuts
out this curve. }

 

\subsection{\textcolor{Chapter }{LeftStructureSheaf (for IsICTCurve)}}
\logpage{[ 7, 3, 5 ]}\nobreak
\hyperdef{L}{X7C6D5FC08563A307}{}
{\noindent\textcolor{FuncColor}{$\triangleright$\enspace\texttt{LeftStructureSheaf({\mdseries\slshape C})\index{LeftStructureSheaf@\texttt{LeftStructureSheaf}!for IsICTCurve}
\label{LeftStructureSheaf:for IsICTCurve}
}\hfill{\scriptsize (attribute)}}\\
\textbf{\indent Returns:\ }
a f.p. graded left S-module 



 The argument is an ICT curve $C$. The output is the f.p. graded left S-module which sheafifes to the structure
sheaf of this curve $C$. }

 

\subsection{\textcolor{Chapter }{RightStructureSheaf (for IsICTCurve)}}
\logpage{[ 7, 3, 6 ]}\nobreak
\hyperdef{L}{X818EE26F7E0CB119}{}
{\noindent\textcolor{FuncColor}{$\triangleright$\enspace\texttt{RightStructureSheaf({\mdseries\slshape C})\index{RightStructureSheaf@\texttt{RightStructureSheaf}!for IsICTCurve}
\label{RightStructureSheaf:for IsICTCurve}
}\hfill{\scriptsize (attribute)}}\\
\textbf{\indent Returns:\ }
a f.p. graded right S-module 



 The argument is an ICT curve $C$. The output is the f.p. graded right S-module which sheafifes to the
structure sheaf of this curve $C$. }

 

\subsection{\textcolor{Chapter }{IntersectionU (for IsICTCurve)}}
\logpage{[ 7, 3, 7 ]}\nobreak
\hyperdef{L}{X7C5D67897A9879FB}{}
{\noindent\textcolor{FuncColor}{$\triangleright$\enspace\texttt{IntersectionU({\mdseries\slshape C})\index{IntersectionU@\texttt{IntersectionU}!for IsICTCurve}
\label{IntersectionU:for IsICTCurve}
}\hfill{\scriptsize (attribute)}}\\
\textbf{\indent Returns:\ }
a list of integers 



 The argument is an ICT curve $C$. The output is the integral vector $u$ used to compute intersection products with Cartier divisors. }

 

\subsection{\textcolor{Chapter }{IntersectionList (for IsICTCurve)}}
\logpage{[ 7, 3, 8 ]}\nobreak
\hyperdef{L}{X870E47497F6FA828}{}
{\noindent\textcolor{FuncColor}{$\triangleright$\enspace\texttt{IntersectionList({\mdseries\slshape C})\index{IntersectionList@\texttt{IntersectionList}!for IsICTCurve}
\label{IntersectionList:for IsICTCurve}
}\hfill{\scriptsize (attribute)}}\\
\textbf{\indent Returns:\ }
a list of integers 



 The argument is an ICT curve $C$. The output is a list with the intersection numbers of a canonical base of
the class group. This basis is to take $\left( e_1, \dots, e_k \right)$ with $e_i = \left( 0, \dots, 0, 1, 0, \dots, 0 \right) \in Cl \left( X_\Sigma
\right)$. }

 }

 
\section{\textcolor{Chapter }{Operations with ICTCurves}}\label{Chapter_Irreducible_complete_torus-invariant_curves_and_proper_1-cycles_in_a_toric_variety_Section_Operations_with_ICTCurves}
\logpage{[ 7, 4, 0 ]}
\hyperdef{L}{X85D64A7F820930F4}{}
{
  

\subsection{\textcolor{Chapter }{ICTCurves (for IsToricVariety)}}
\logpage{[ 7, 4, 1 ]}\nobreak
\hyperdef{L}{X7E4E0C127CC4BAA8}{}
{\noindent\textcolor{FuncColor}{$\triangleright$\enspace\texttt{ICTCurves({\mdseries\slshape X{\textunderscore}Sigma})\index{ICTCurves@\texttt{ICTCurves}!for IsToricVariety}
\label{ICTCurves:for IsToricVariety}
}\hfill{\scriptsize (attribute)}}\\
\textbf{\indent Returns:\ }
a list of ICT-curves. 



 For a smooth and complete toric variety $X_\Sigma$, this method computes a list of all ICT-curves in $X_\Sigma$. Note that those curves can be numerically equivalent. }

 

\subsection{\textcolor{Chapter }{IntersectionProduct (for IsICTCurve, IsToricDivisor)}}
\logpage{[ 7, 4, 2 ]}\nobreak
\hyperdef{L}{X86F12D36867861E5}{}
{\noindent\textcolor{FuncColor}{$\triangleright$\enspace\texttt{IntersectionProduct({\mdseries\slshape C, D})\index{IntersectionProduct@\texttt{IntersectionProduct}!for IsICTCurve, IsToricDivisor}
\label{IntersectionProduct:for IsICTCurve, IsToricDivisor}
}\hfill{\scriptsize (operation)}}\\
\textbf{\indent Returns:\ }
an integer 



 Given an ICT-curve $C$ and a divisor $D$ in a smooth and complete toric variety $X_\Sigma$, this method computes their intersection product. }

 

\subsection{\textcolor{Chapter }{IntersectionProduct (for IsToricDivisor, IsICTCurve)}}
\logpage{[ 7, 4, 3 ]}\nobreak
\hyperdef{L}{X82B2C573868E2017}{}
{\noindent\textcolor{FuncColor}{$\triangleright$\enspace\texttt{IntersectionProduct({\mdseries\slshape arg1, arg2})\index{IntersectionProduct@\texttt{IntersectionProduct}!for IsToricDivisor, IsICTCurve}
\label{IntersectionProduct:for IsToricDivisor, IsICTCurve}
}\hfill{\scriptsize (operation)}}\\


 

 }

 }

 
\section{\textcolor{Chapter }{GAP category for proper 1-cycles}}\label{Chapter_Irreducible_complete_torus-invariant_curves_and_proper_1-cycles_in_a_toric_variety_Section_GAP_category_for_proper_1-cycles}
\logpage{[ 7, 5, 0 ]}
\hyperdef{L}{X7AFF052E7BA9D0F3}{}
{
  

\subsection{\textcolor{Chapter }{IsProper1Cycle (for IsObject)}}
\logpage{[ 7, 5, 1 ]}\nobreak
\hyperdef{L}{X7A25DF897F2CD605}{}
{\noindent\textcolor{FuncColor}{$\triangleright$\enspace\texttt{IsProper1Cycle({\mdseries\slshape object})\index{IsProper1Cycle@\texttt{IsProper1Cycle}!for IsObject}
\label{IsProper1Cycle:for IsObject}
}\hfill{\scriptsize (filter)}}\\
\textbf{\indent Returns:\ }
true or false 



 The GAP category for proper 1-cycles }

 }

 
\section{\textcolor{Chapter }{Constructor For Proper 1-Cycles}}\label{Chapter_Irreducible_complete_torus-invariant_curves_and_proper_1-cycles_in_a_toric_variety_Section_Constructor_For_Proper_1-Cycles}
\logpage{[ 7, 6, 0 ]}
\hyperdef{L}{X7FE0B6E9806599A1}{}
{
  

\subsection{\textcolor{Chapter }{GeneratorsOfProper1Cycles (for IsToricVariety)}}
\logpage{[ 7, 6, 1 ]}\nobreak
\hyperdef{L}{X7EED11157DAD88FA}{}
{\noindent\textcolor{FuncColor}{$\triangleright$\enspace\texttt{GeneratorsOfProper1Cycles({\mdseries\slshape X{\textunderscore}Sigma})\index{GeneratorsOfProper1Cycles@\texttt{GeneratorsOfProper1Cycles}!for IsToricVariety}
\label{GeneratorsOfProper1Cycles:for IsToricVariety}
}\hfill{\scriptsize (attribute)}}\\
\textbf{\indent Returns:\ }
a list of ICT-curves. 



 For a smooth and complete toric variety $X_\Sigma$, this method computes a list of all ICT-curves which are not numerically
equivalent. We use this list of ICT-curves as a basis of proper 1-cycles on $X_\Sigma$ in the constructor below, when computing the intersection form and the
Nef-cone. }

 

\subsection{\textcolor{Chapter }{Proper1Cycle (for IsToricVariety, IsList)}}
\logpage{[ 7, 6, 2 ]}\nobreak
\hyperdef{L}{X83F7464C820146BA}{}
{\noindent\textcolor{FuncColor}{$\triangleright$\enspace\texttt{Proper1Cycle({\mdseries\slshape X{\textunderscore}Sigma, list})\index{Proper1Cycle@\texttt{Proper1Cycle}!for IsToricVariety, IsList}
\label{Proper1Cycle:for IsToricVariety, IsList}
}\hfill{\scriptsize (operation)}}\\
\textbf{\indent Returns:\ }
a proper 1-cycle 



 The arguments are a smooth and complete toric variety $X_\Sigma$ and a list of integers. We then use the integers in this list as 'coordinates'
of the proper 1-cycle with respect to the list of proper 1-cycles produced by
the previous method. We then return the corresponding proper 1-cycle. }

 }

 
\section{\textcolor{Chapter }{Attributes for proper 1-cycles}}\label{Chapter_Irreducible_complete_torus-invariant_curves_and_proper_1-cycles_in_a_toric_variety_Section_Attributes_for_proper_1-cycles}
\logpage{[ 7, 7, 0 ]}
\hyperdef{L}{X849341167EF9F424}{}
{
  

\subsection{\textcolor{Chapter }{AmbientToricVariety (for IsProper1Cycle)}}
\logpage{[ 7, 7, 1 ]}\nobreak
\hyperdef{L}{X848BD3EC8504059F}{}
{\noindent\textcolor{FuncColor}{$\triangleright$\enspace\texttt{AmbientToricVariety({\mdseries\slshape C})\index{AmbientToricVariety@\texttt{AmbientToricVariety}!for IsProper1Cycle}
\label{AmbientToricVariety:for IsProper1Cycle}
}\hfill{\scriptsize (attribute)}}\\
\textbf{\indent Returns:\ }
a toric variety 



 The argument is a proper 1-cycle $C$. The output is the toric variety, in which this cycle $C$ lies. }

 

\subsection{\textcolor{Chapter }{UnderlyingGroupElement (for IsProper1Cycle)}}
\logpage{[ 7, 7, 2 ]}\nobreak
\hyperdef{L}{X7B7E08E2850BE1F8}{}
{\noindent\textcolor{FuncColor}{$\triangleright$\enspace\texttt{UnderlyingGroupElement({\mdseries\slshape C})\index{UnderlyingGroupElement@\texttt{UnderlyingGroupElement}!for IsProper1Cycle}
\label{UnderlyingGroupElement:for IsProper1Cycle}
}\hfill{\scriptsize (attribute)}}\\
\textbf{\indent Returns:\ }
a list 



 The argument is a proper 1-cycle. We then return the underlying group element
(with respect to the generators computed from the method
\texttt{\symbol{92}}emph\texttt{\symbol{123}}GeneratorsOfProper1Cycles\texttt{\symbol{125}}). }

 }

 
\section{\textcolor{Chapter }{Operations with proper 1-cycles}}\label{Chapter_Irreducible_complete_torus-invariant_curves_and_proper_1-cycles_in_a_toric_variety_Section_Operations_with_proper_1-cycles}
\logpage{[ 7, 8, 0 ]}
\hyperdef{L}{X865DAEC07BE0E063}{}
{
  

\subsection{\textcolor{Chapter }{IntersectionProduct (for IsProper1Cycle, IsToricDivisor)}}
\logpage{[ 7, 8, 1 ]}\nobreak
\hyperdef{L}{X87E986CF7BA5A12E}{}
{\noindent\textcolor{FuncColor}{$\triangleright$\enspace\texttt{IntersectionProduct({\mdseries\slshape C, D})\index{IntersectionProduct@\texttt{IntersectionProduct}!for IsProper1Cycle, IsToricDivisor}
\label{IntersectionProduct:for IsProper1Cycle, IsToricDivisor}
}\hfill{\scriptsize (operation)}}\\
\textbf{\indent Returns:\ }
an integer 



 Given a proper 1-cycle $C$ and a divisor $D$ in a smooth and complete toric variety $X_\Sigma$, this method computes their intersection product. }

 

\subsection{\textcolor{Chapter }{IntersectionProduct (for IsToricDivisor, IsProper1Cycle)}}
\logpage{[ 7, 8, 2 ]}\nobreak
\hyperdef{L}{X7E3BAE0182356C57}{}
{\noindent\textcolor{FuncColor}{$\triangleright$\enspace\texttt{IntersectionProduct({\mdseries\slshape arg1, arg2})\index{IntersectionProduct@\texttt{IntersectionProduct}!for IsToricDivisor, IsProper1Cycle}
\label{IntersectionProduct:for IsToricDivisor, IsProper1Cycle}
}\hfill{\scriptsize (operation)}}\\


 

 }

 

\subsection{\textcolor{Chapter }{IntersectionForm (for IsToricVariety)}}
\logpage{[ 7, 8, 3 ]}\nobreak
\hyperdef{L}{X7E9AB7467AAB1B34}{}
{\noindent\textcolor{FuncColor}{$\triangleright$\enspace\texttt{IntersectionForm({\mdseries\slshape vari})\index{IntersectionForm@\texttt{IntersectionForm}!for IsToricVariety}
\label{IntersectionForm:for IsToricVariety}
}\hfill{\scriptsize (attribute)}}\\
\textbf{\indent Returns:\ }
a list of lists 



 Given a simplicial and complete toric variety, we can use proposition 6.4.1 of
Cox-Schenk-Litte to compute the intersection form $N^1 \left( X_\Sigma \right) \times N_1 \left( X_\Sigma \right) \to \mathbb{R} $. We return a list of lists that encodes this mapping. }

 }

 
\section{\textcolor{Chapter }{Examples in projective space}}\label{Chapter_Irreducible_complete_torus-invariant_curves_and_proper_1-cycles_in_a_toric_variety_Section_Examples_in_projective_space}
\logpage{[ 7, 9, 0 ]}
\hyperdef{L}{X7F6099FC7DA7FF99}{}
{
  
\begin{Verbatim}[commandchars=!@|,fontsize=\small,frame=single,label=Example]
  !gapprompt@gap>| !gapinput@P2 := ProjectiveSpace( 2 );|
  <A projective toric variety of dimension 2>
  !gapprompt@gap>| !gapinput@ICTCurves( P2 );|
  [ <An irreducible, complete, torus-invariant curve in a toric variety
     given as V( [ x_3 ] )>,
    <An irreducible, complete, torus-invariant curve in a toric variety
     given as V( [ x_2 ] )>,
    <An irreducible, complete, torus-invariant curve in a toric variety
     given as V( [ x_1 ] )> ]
  !gapprompt@gap>| !gapinput@C1 := ICTCurves( P2 )[ 1 ];|
  <An irreducible, complete, torus-invariant curve in a toric variety
   given as V( [ x_3 ] )>
  !gapprompt@gap>| !gapinput@IntersectionForm( P2 );|
  [ [ 1 ] ]
  !gapprompt@gap>| !gapinput@IntersectionProduct( C1, DivisorOfGivenClass( P2, [ 1 ] ) );|
  1
  !gapprompt@gap>| !gapinput@IntersectionProduct( DivisorOfGivenClass( P2, [ 5 ] ), C1 );|
  5
\end{Verbatim}
 
\begin{Verbatim}[commandchars=!@|,fontsize=\small,frame=single,label=Example]
  !gapprompt@gap>| !gapinput@P3 := ProjectiveSpace( 3 );|
  <A projective toric variety of dimension 3>
  !gapprompt@gap>| !gapinput@C1 := ICTCurves( P3 )[ 1 ];|
  <An irreducible, complete, torus-invariant curve in a toric variety
   given as V( [ x_3, x_4 ] )>
  !gapprompt@gap>| !gapinput@vars := DefiningVariables( C1 );|
  [ x_3, x_4 ]
  !gapprompt@gap>| !gapinput@structureSheaf1 := LeftStructureSheaf( C1 );;|
  !gapprompt@gap>| !gapinput@IsWellDefined( structureSheaf1 );|
  true
  !gapprompt@gap>| !gapinput@structureSheaf2 := RightStructureSheaf( C1 );;|
  !gapprompt@gap>| !gapinput@IsWellDefined( structureSheaf2 );|
  true
\end{Verbatim}
 }

 }

   
\chapter{\textcolor{Chapter }{Nef and Mori Cone}}\label{Chapter_Nef_and_Mori_Cone}
\logpage{[ 8, 0, 0 ]}
\hyperdef{L}{X824D03307C329CE0}{}
{
  
\section{\textcolor{Chapter }{New Properties For Toric Divisors}}\label{Chapter_Nef_and_Mori_Cone_Section_New_Properties_For_Toric_Divisors}
\logpage{[ 8, 1, 0 ]}
\hyperdef{L}{X87741CBD80A44D38}{}
{
  

\subsection{\textcolor{Chapter }{IsNef (for IsToricDivisor)}}
\logpage{[ 8, 1, 1 ]}\nobreak
\hyperdef{L}{X824DED3C7F5F065C}{}
{\noindent\textcolor{FuncColor}{$\triangleright$\enspace\texttt{IsNef({\mdseries\slshape divi})\index{IsNef@\texttt{IsNef}!for IsToricDivisor}
\label{IsNef:for IsToricDivisor}
}\hfill{\scriptsize (property)}}\\
\textbf{\indent Returns:\ }
true or false 



 Checks if the torus invariant Weil divisor \mbox{\texttt{\mdseries\slshape divi}} is nef. }

 

\subsection{\textcolor{Chapter }{IsAmpleViaNefCone (for IsToricDivisor)}}
\logpage{[ 8, 1, 2 ]}\nobreak
\hyperdef{L}{X78DB40598598F631}{}
{\noindent\textcolor{FuncColor}{$\triangleright$\enspace\texttt{IsAmpleViaNefCone({\mdseries\slshape divi})\index{IsAmpleViaNefCone@\texttt{IsAmpleViaNefCone}!for IsToricDivisor}
\label{IsAmpleViaNefCone:for IsToricDivisor}
}\hfill{\scriptsize (property)}}\\
\textbf{\indent Returns:\ }
true or false 



 Checks if the class of the torus invariant Weil divisor \mbox{\texttt{\mdseries\slshape divi}} lies in the interior of the nef cone. Given that the ambient toric variety is
projective this implies that \mbox{\texttt{\mdseries\slshape divi}} is ample. }

 }

 
\section{\textcolor{Chapter }{Attributes}}\label{Chapter_Nef_and_Mori_Cone_Section_Attributes}
\logpage{[ 8, 2, 0 ]}
\hyperdef{L}{X7C701DBF7BAE649A}{}
{
  

\subsection{\textcolor{Chapter }{CartierDataGroup (for IsToricVariety)}}
\logpage{[ 8, 2, 1 ]}\nobreak
\hyperdef{L}{X8776E09C7B0DD45F}{}
{\noindent\textcolor{FuncColor}{$\triangleright$\enspace\texttt{CartierDataGroup({\mdseries\slshape vari})\index{CartierDataGroup@\texttt{CartierDataGroup}!for IsToricVariety}
\label{CartierDataGroup:for IsToricVariety}
}\hfill{\scriptsize (attribute)}}\\
\textbf{\indent Returns:\ }
a list of lists 



 Given a toric variety \mbox{\texttt{\mdseries\slshape vari}}, we compute the integral vectors in $\mathbb{Z}^{n \left| \Sigma_{max} \right|}$, $n$ is the rank of the character lattice which encodes a toric Cartier divisor
according to theorem 4.2.8. in Cox-Schenk-Little. We return a list of lists.
When interpreting this list of lists as a matrix, then the kernel of this
matrix is the set of these vectors. }

 

\subsection{\textcolor{Chapter }{NefConeInCartierDataGroup (for IsToricVariety)}}
\logpage{[ 8, 2, 2 ]}\nobreak
\hyperdef{L}{X79E7C10083F6A23B}{}
{\noindent\textcolor{FuncColor}{$\triangleright$\enspace\texttt{NefConeInCartierDataGroup({\mdseries\slshape vari})\index{NefConeInCartierDataGroup@\texttt{NefConeInCartierDataGroup}!for IsToricVariety}
\label{NefConeInCartierDataGroup:for IsToricVariety}
}\hfill{\scriptsize (attribute)}}\\
\textbf{\indent Returns:\ }
a list of lists 



 Given a smooth and complete toric variety \mbox{\texttt{\mdseries\slshape vari}}, we compute the nef cone within the proper Cartier data in $\mathbb{Z}^{n \left| \Sigma_{max} \right|}$, $n$ is the rank of the character lattice. We return a list of ray generators of
this cone. }

 

\subsection{\textcolor{Chapter }{NefConeInTorusInvariantWeilDivisorGroup (for IsToricVariety)}}
\logpage{[ 8, 2, 3 ]}\nobreak
\hyperdef{L}{X8676891C8068E9BE}{}
{\noindent\textcolor{FuncColor}{$\triangleright$\enspace\texttt{NefConeInTorusInvariantWeilDivisorGroup({\mdseries\slshape vari})\index{NefConeInTorusInvariantWeilDivisorGroup@\texttt{Nef}\-\texttt{Cone}\-\texttt{In}\-\texttt{Torus}\-\texttt{Invariant}\-\texttt{Weil}\-\texttt{Divisor}\-\texttt{Group}!for IsToricVariety}
\label{NefConeInTorusInvariantWeilDivisorGroup:for IsToricVariety}
}\hfill{\scriptsize (attribute)}}\\
\textbf{\indent Returns:\ }
a list of lists 



 Given a smooth and complete toric variety, we compute the nef cone within $ Div_T \left( X_\Sigma \right) $. We return a list of ray generators of this cone. }

 

\subsection{\textcolor{Chapter }{NefConeInClassGroup (for IsToricVariety)}}
\logpage{[ 8, 2, 4 ]}\nobreak
\hyperdef{L}{X8766EFA783BE594C}{}
{\noindent\textcolor{FuncColor}{$\triangleright$\enspace\texttt{NefConeInClassGroup({\mdseries\slshape vari})\index{NefConeInClassGroup@\texttt{NefConeInClassGroup}!for IsToricVariety}
\label{NefConeInClassGroup:for IsToricVariety}
}\hfill{\scriptsize (attribute)}}\\
\textbf{\indent Returns:\ }
a list of lists 



 Given a smooth and complete toric variety, we compute the nef cone within $ Cl \left( X_\Sigma \right) $. We return a list of ray generators of this cone. }

 

\subsection{\textcolor{Chapter }{NefCone (for IsToricVariety)}}
\logpage{[ 8, 2, 5 ]}\nobreak
\hyperdef{L}{X7EDF97988311C7D6}{}
{\noindent\textcolor{FuncColor}{$\triangleright$\enspace\texttt{NefCone({\mdseries\slshape arg})\index{NefCone@\texttt{NefCone}!for IsToricVariety}
\label{NefCone:for IsToricVariety}
}\hfill{\scriptsize (attribute)}}\\


 

 }

 

\subsection{\textcolor{Chapter }{ClassesOfSmallestAmpleDivisors (for IsToricVariety)}}
\logpage{[ 8, 2, 6 ]}\nobreak
\hyperdef{L}{X810BCBEE87FF0BE4}{}
{\noindent\textcolor{FuncColor}{$\triangleright$\enspace\texttt{ClassesOfSmallestAmpleDivisors({\mdseries\slshape vari})\index{ClassesOfSmallestAmpleDivisors@\texttt{ClassesOfSmallestAmpleDivisors}!for IsToricVariety}
\label{ClassesOfSmallestAmpleDivisors:for IsToricVariety}
}\hfill{\scriptsize (attribute)}}\\
\textbf{\indent Returns:\ }
a list of integers 



 Given a smooth and complete toric variety, we compute the smallest divisor
class, such that the associated divisor is ample. This is based on theorem
6.3.22 in Cox-Schenk-Little, which implies that this task is equivalent to
finding the smallest lattice point within the nef cone (in $Cl \left( X_\Sigma \right)$). We return a list of integers encoding this lattice point. }

 

\subsection{\textcolor{Chapter }{GroupOfProper1Cycles (for IsToricVariety)}}
\logpage{[ 8, 2, 7 ]}\nobreak
\hyperdef{L}{X82C6E9307D7230AA}{}
{\noindent\textcolor{FuncColor}{$\triangleright$\enspace\texttt{GroupOfProper1Cycles({\mdseries\slshape vari})\index{GroupOfProper1Cycles@\texttt{GroupOfProper1Cycles}!for IsToricVariety}
\label{GroupOfProper1Cycles:for IsToricVariety}
}\hfill{\scriptsize (attribute)}}\\
\textbf{\indent Returns:\ }
a kernel submodule 



 Given a simplicial and complete toric variety, we use proposition 6.4.1 of
Cox-Schenk-Litte to compute the group of proper 1-cycles. We return the
corresponding kernel submodule. }

 

\subsection{\textcolor{Chapter }{MoriCone (for IsToricVariety)}}
\logpage{[ 8, 2, 8 ]}\nobreak
\hyperdef{L}{X7F2479E07CFF2511}{}
{\noindent\textcolor{FuncColor}{$\triangleright$\enspace\texttt{MoriCone({\mdseries\slshape vari})\index{MoriCone@\texttt{MoriCone}!for IsToricVariety}
\label{MoriCone:for IsToricVariety}
}\hfill{\scriptsize (attribute)}}\\
\textbf{\indent Returns:\ }
an NmzCone6 



 Given a smooth and complete toric variety, we can compute both the
intersection form and the Nef cone in the class group. Then the Mori cone is
the dual cone of the Nef cone with respect to the intersection product. We
compute an H-presentation of this dual cone and use those to produce a cone
with the normaliz interface. }

 }

 
\section{\textcolor{Chapter }{Nef and Mori Cone: Examples}}\label{Chapter_Nef_and_Mori_Cone_Section_Nef_and_Mori_Cone_Examples}
\logpage{[ 8, 3, 0 ]}
\hyperdef{L}{X82B5D6E87F491048}{}
{
  
\subsection{\textcolor{Chapter }{Projective Space}}\label{Chapter_Nef_and_Mori_Cone_Section_Nef_and_Mori_Cone_Examples_Subsection_Projective_Space}
\logpage{[ 8, 3, 1 ]}
\hyperdef{L}{X7D486AA57EE583B9}{}
{
  
\begin{Verbatim}[commandchars=!@|,fontsize=\small,frame=single,label=Example]
  !gapprompt@gap>| !gapinput@P2 := ProjectiveSpace( 2 );|
  <A projective toric variety of dimension 2>
  !gapprompt@gap>| !gapinput@P2xP2 := P2*P2;|
  <A projective toric variety of dimension 4
  which is a product of 2 toric varieties>
  !gapprompt@gap>| !gapinput@NefCone( P2 );|
  [ [ 1 ] ]
  !gapprompt@gap>| !gapinput@NefCone( P2xP2 );|
  [ [ 0, 1 ], [ 1, 0 ] ]
  !gapprompt@gap>| !gapinput@MoriCone( P2 );|
  [ [ 1 ] ]
  !gapprompt@gap>| !gapinput@MoriCone( P2xP2 );|
  [ [ 0, 1 ], [ 1, 0 ] ]
  !gapprompt@gap>| !gapinput@D1 := DivisorOfGivenClass( P2, [ -1 ] );|
  <A Cartier divisor of a toric variety with coordinates ( -1, 0, 0 )>
  !gapprompt@gap>| !gapinput@IsAmpleViaNefCone( D1 );|
  false
  !gapprompt@gap>| !gapinput@D2 := DivisorOfGivenClass( P2, [ 1 ] );|
  <A Cartier divisor of a toric variety with coordinates ( 1, 0, 0 )>
  !gapprompt@gap>| !gapinput@IsAmpleViaNefCone( D2 );|
  true
  !gapprompt@gap>| !gapinput@ClassesOfSmallestAmpleDivisors( P2 );|
  [ [ 1 ] ]
  !gapprompt@gap>| !gapinput@ClassesOfSmallestAmpleDivisors( P2xP2 );|
  [ [ 1, 1 ] ]
\end{Verbatim}
 }

 }

 }

   
\chapter{\textcolor{Chapter }{DegreeXLayerVectorSpaces and morphisms}}\label{Chapter_DegreeXLayerVectorSpaces_and_morphisms}
\logpage{[ 9, 0, 0 ]}
\hyperdef{L}{X8724C78B7A053204}{}
{
  
\section{\textcolor{Chapter }{GAP category of DegreeXLayerVectorSpaces}}\label{Chapter_DegreeXLayerVectorSpaces_and_morphisms_Section_GAP_category_of_DegreeXLayerVectorSpaces}
\logpage{[ 9, 1, 0 ]}
\hyperdef{L}{X7F5C690C79E08D5B}{}
{
  

\subsection{\textcolor{Chapter }{IsDegreeXLayerVectorSpace (for IsObject)}}
\logpage{[ 9, 1, 1 ]}\nobreak
\hyperdef{L}{X7E34BC167FED4370}{}
{\noindent\textcolor{FuncColor}{$\triangleright$\enspace\texttt{IsDegreeXLayerVectorSpace({\mdseries\slshape object})\index{IsDegreeXLayerVectorSpace@\texttt{IsDegreeXLayerVectorSpace}!for IsObject}
\label{IsDegreeXLayerVectorSpace:for IsObject}
}\hfill{\scriptsize (filter)}}\\
\textbf{\indent Returns:\ }
true or false 



 The GAP category for vector spaces that represent a degree layer of a f.p.
graded module }

 

\subsection{\textcolor{Chapter }{IsDegreeXLayerVectorSpaceMorphism (for IsObject)}}
\logpage{[ 9, 1, 2 ]}\nobreak
\hyperdef{L}{X7FBABC4579DA58A6}{}
{\noindent\textcolor{FuncColor}{$\triangleright$\enspace\texttt{IsDegreeXLayerVectorSpaceMorphism({\mdseries\slshape object})\index{IsDegreeXLayerVectorSpaceMorphism@\texttt{IsDegreeXLayerVectorSpaceMorphism}!for IsObject}
\label{IsDegreeXLayerVectorSpaceMorphism:for IsObject}
}\hfill{\scriptsize (filter)}}\\
\textbf{\indent Returns:\ }
true or false 



 The GAP category for morphisms between vector spaces that represent a degree
layer of a f.p. graded module }

 

\subsection{\textcolor{Chapter }{IsDegreeXLayerVectorSpacePresentation (for IsObject)}}
\logpage{[ 9, 1, 3 ]}\nobreak
\hyperdef{L}{X7FA44D177A905726}{}
{\noindent\textcolor{FuncColor}{$\triangleright$\enspace\texttt{IsDegreeXLayerVectorSpacePresentation({\mdseries\slshape object})\index{IsDegreeXLayerVectorSpacePresentation@\texttt{IsDegree}\-\texttt{X}\-\texttt{Layer}\-\texttt{Vector}\-\texttt{Space}\-\texttt{Presentation}!for IsObject}
\label{IsDegreeXLayerVectorSpacePresentation:for IsObject}
}\hfill{\scriptsize (filter)}}\\
\textbf{\indent Returns:\ }
true or false 



 The GAP category for (left) presentations of vector spaces that represent a
degree layer of a f.p. graded module }

 

\subsection{\textcolor{Chapter }{IsDegreeXLayerVectorSpacePresentationMorphism (for IsObject)}}
\logpage{[ 9, 1, 4 ]}\nobreak
\hyperdef{L}{X876AD2787F18E093}{}
{\noindent\textcolor{FuncColor}{$\triangleright$\enspace\texttt{IsDegreeXLayerVectorSpacePresentationMorphism({\mdseries\slshape object})\index{IsDegreeXLayerVectorSpacePresentationMorphism@\texttt{IsDegree}\-\texttt{X}\-\texttt{Layer}\-\texttt{Vector}\-\texttt{Space}\-\texttt{Presentation}\-\texttt{Morphism}!for IsObject}
\label{IsDegreeXLayerVectorSpacePresentationMorphism:for IsObject}
}\hfill{\scriptsize (filter)}}\\
\textbf{\indent Returns:\ }
true or false 



 The GAP category for (left) presentation morphisms of vector spaces that
represent a degree layer of a f.p. graded module }

 }

 
\section{\textcolor{Chapter }{Constructors for DegreeXLayerVectorSpaces}}\label{Chapter_DegreeXLayerVectorSpaces_and_morphisms_Section_Constructors_for_DegreeXLayerVectorSpaces}
\logpage{[ 9, 2, 0 ]}
\hyperdef{L}{X7B8A57FC7EA89289}{}
{
  

\subsection{\textcolor{Chapter }{DegreeXLayerVectorSpace (for IsList, IsHomalgGradedRing, IsVectorSpaceObject, IsInt)}}
\logpage{[ 9, 2, 1 ]}\nobreak
\hyperdef{L}{X7CD2AC7E7DB45030}{}
{\noindent\textcolor{FuncColor}{$\triangleright$\enspace\texttt{DegreeXLayerVectorSpace({\mdseries\slshape L, S, V, n})\index{DegreeXLayerVectorSpace@\texttt{DegreeXLayerVectorSpace}!for IsList, IsHomalgGradedRing, IsVectorSpaceObject, IsInt}
\label{DegreeXLayerVectorSpace:for IsList, IsHomalgGradedRing, IsVectorSpaceObject, IsInt}
}\hfill{\scriptsize (operation)}}\\
\textbf{\indent Returns:\ }
a CAPCategoryObject 



 The arguments are a list of monomials $L$, a homalg graded ring $S$ (the Coxring of the variety in question), a vector space $V$ and a non-negative integer $n$. $V$ is to be given as a vector space defined in the package 'LinearAlgebraForCAP'.
This method then returns the corresponding DegreeXLayerVectorSpace. }

 

\subsection{\textcolor{Chapter }{DegreeXLayerVectorSpaceMorphism (for IsDegreeXLayerVectorSpace, IsVectorSpaceMorphism, IsDegreeXLayerVectorSpace)}}
\logpage{[ 9, 2, 2 ]}\nobreak
\hyperdef{L}{X87BB659C7F01DFCD}{}
{\noindent\textcolor{FuncColor}{$\triangleright$\enspace\texttt{DegreeXLayerVectorSpaceMorphism({\mdseries\slshape L, S, V})\index{DegreeXLayerVectorSpaceMorphism@\texttt{DegreeXLayerVectorSpaceMorphism}!for IsDegreeXLayerVectorSpace, IsVectorSpaceMorphism, IsDegreeXLayerVectorSpace}
\label{DegreeXLayerVectorSpaceMorphism:for IsDegreeXLayerVectorSpace, IsVectorSpaceMorphism, IsDegreeXLayerVectorSpace}
}\hfill{\scriptsize (operation)}}\\
\textbf{\indent Returns:\ }
a DegreeXLayerVectorSpaceMorphism 



 The arguments are a DegreeXLayerVectorSpace \mbox{\texttt{\mdseries\slshape source}}, a vector space morphism $\varphi$ (as defined in 'LinearAlgebraForCAP') and a DegreeXLayerVectorSpace \mbox{\texttt{\mdseries\slshape range}}. If $\varphi$ is a vector space morphism between the underlying vector spaces of \mbox{\texttt{\mdseries\slshape source}} and \mbox{\texttt{\mdseries\slshape range}} this method returns the corresponding DegreeXLayerVectorSpaceMorphism. }

 

\subsection{\textcolor{Chapter }{DegreeXLayerVectorSpacePresentation (for IsDegreeXLayerVectorSpaceMorphism)}}
\logpage{[ 9, 2, 3 ]}\nobreak
\hyperdef{L}{X81C82EB179924A9D}{}
{\noindent\textcolor{FuncColor}{$\triangleright$\enspace\texttt{DegreeXLayerVectorSpacePresentation({\mdseries\slshape a})\index{DegreeXLayerVectorSpacePresentation@\texttt{DegreeXLayerVectorSpacePresentation}!for IsDegreeXLayerVectorSpaceMorphism}
\label{DegreeXLayerVectorSpacePresentation:for IsDegreeXLayerVectorSpaceMorphism}
}\hfill{\scriptsize (operation)}}\\
\textbf{\indent Returns:\ }
a DegreeXLayerVectorSpaceMorphism 



 The arguments is a DegreeXLayerVectorSpaceMorphism \mbox{\texttt{\mdseries\slshape a}}. This method treats this morphism as a presentation, i.e. we are interested
in the cokernel of the underlying morphism of vector spaces. The corresponding
DegreeXLayerVectorSpacePresentation is returned. }

 

\subsection{\textcolor{Chapter }{DegreeXLayerVectorSpacePresentationMorphism (for IsDegreeXLayerVectorSpacePresentation, IsVectorSpaceMorphism, IsDegreeXLayerVectorSpacePresentation)}}
\logpage{[ 9, 2, 4 ]}\nobreak
\hyperdef{L}{X7BE3295D86A402B1}{}
{\noindent\textcolor{FuncColor}{$\triangleright$\enspace\texttt{DegreeXLayerVectorSpacePresentationMorphism({\mdseries\slshape source, \texttt{\symbol{92}}varphi, range})\index{DegreeXLayerVectorSpacePresentationMorphism@\texttt{Degree}\-\texttt{X}\-\texttt{Layer}\-\texttt{Vector}\-\texttt{Space}\-\texttt{Presentation}\-\texttt{Morphism}!for IsDegreeXLayerVectorSpacePresentation, IsVectorSpaceMorphism, IsDegreeXLayerVectorSpacePresentation}
\label{DegreeXLayerVectorSpacePresentationMorphism:for IsDegreeXLayerVectorSpacePresentation, IsVectorSpaceMorphism, IsDegreeXLayerVectorSpacePresentation}
}\hfill{\scriptsize (operation)}}\\
\textbf{\indent Returns:\ }
a DegreeXLayerVectorSpacePresentationMorphism 



 The arguments is a DegreeXLayerVectorSpacePresentation \mbox{\texttt{\mdseries\slshape source}}, a vector space morphism $\varphi$ and a DegreeXLayerVectorSpacePresentation \mbox{\texttt{\mdseries\slshape range}}. The corresponding DegreeXLayerVectorSpacePresentationMorphism is returned. }

 }

 
\section{\textcolor{Chapter }{Attributes for DegreeXLayerVectorSpaces}}\label{Chapter_DegreeXLayerVectorSpaces_and_morphisms_Section_Attributes_for_DegreeXLayerVectorSpaces}
\logpage{[ 9, 3, 0 ]}
\hyperdef{L}{X7E2F356782276C8D}{}
{
  

\subsection{\textcolor{Chapter }{UnderlyingHomalgGradedRing (for IsDegreeXLayerVectorSpace)}}
\logpage{[ 9, 3, 1 ]}\nobreak
\hyperdef{L}{X83D6FCD47A09721C}{}
{\noindent\textcolor{FuncColor}{$\triangleright$\enspace\texttt{UnderlyingHomalgGradedRing({\mdseries\slshape V})\index{UnderlyingHomalgGradedRing@\texttt{UnderlyingHomalgGradedRing}!for IsDegreeXLayerVectorSpace}
\label{UnderlyingHomalgGradedRing:for IsDegreeXLayerVectorSpace}
}\hfill{\scriptsize (attribute)}}\\
\textbf{\indent Returns:\ }
a homalg graded ring 



 The argument is a DegreeXLayerVectorSpace $V$. The output is the Coxring, in which this vector space is embedded via the
generators (specified when installing $V$). }

 

\subsection{\textcolor{Chapter }{Generators (for IsDegreeXLayerVectorSpace)}}
\logpage{[ 9, 3, 2 ]}\nobreak
\hyperdef{L}{X81F703E87C9EB27A}{}
{\noindent\textcolor{FuncColor}{$\triangleright$\enspace\texttt{Generators({\mdseries\slshape V})\index{Generators@\texttt{Generators}!for IsDegreeXLayerVectorSpace}
\label{Generators:for IsDegreeXLayerVectorSpace}
}\hfill{\scriptsize (attribute)}}\\
\textbf{\indent Returns:\ }
a list 



 The argument is a DegreeXLayerVectorSpace $V$. The output is the list of generators, that embed $V$ into the Coxring in question. }

 

\subsection{\textcolor{Chapter }{UnderlyingVectorSpaceObject (for IsDegreeXLayerVectorSpace)}}
\logpage{[ 9, 3, 3 ]}\nobreak
\hyperdef{L}{X7DE2F87E82E41996}{}
{\noindent\textcolor{FuncColor}{$\triangleright$\enspace\texttt{UnderlyingVectorSpaceObject({\mdseries\slshape V})\index{UnderlyingVectorSpaceObject@\texttt{UnderlyingVectorSpaceObject}!for IsDegreeXLayerVectorSpace}
\label{UnderlyingVectorSpaceObject:for IsDegreeXLayerVectorSpace}
}\hfill{\scriptsize (attribute)}}\\
\textbf{\indent Returns:\ }
a VectorSpaceObject 



 The argument is a DegreeXLayerVectorSpace $V$. The output is the underlying vectorspace object (as defined in
'LinearAlgebraForCAP'). }

 

\subsection{\textcolor{Chapter }{EmbeddingDimension (for IsDegreeXLayerVectorSpace)}}
\logpage{[ 9, 3, 4 ]}\nobreak
\hyperdef{L}{X8711AE16835A1ED8}{}
{\noindent\textcolor{FuncColor}{$\triangleright$\enspace\texttt{EmbeddingDimension({\mdseries\slshape V})\index{EmbeddingDimension@\texttt{EmbeddingDimension}!for IsDegreeXLayerVectorSpace}
\label{EmbeddingDimension:for IsDegreeXLayerVectorSpace}
}\hfill{\scriptsize (attribute)}}\\
\textbf{\indent Returns:\ }
a VectorSpaceObject 



 The argument is a DegreeXLayerVectorSpace $V$. For $S$ its 'UnderlyingHomalgGradedRing' this vector space is embedded (via its
generators) into $S^n$. The integer $n$ is the embedding dimension. }

 }

 
\section{\textcolor{Chapter }{Attributes for DegreeXLayerVectorSpaceMorphisms}}\label{Chapter_DegreeXLayerVectorSpaces_and_morphisms_Section_Attributes_for_DegreeXLayerVectorSpaceMorphisms}
\logpage{[ 9, 4, 0 ]}
\hyperdef{L}{X7D83F3C28766FBCB}{}
{
  

\subsection{\textcolor{Chapter }{Source (for IsDegreeXLayerVectorSpaceMorphism)}}
\logpage{[ 9, 4, 1 ]}\nobreak
\hyperdef{L}{X860BF0BC7CAB17D5}{}
{\noindent\textcolor{FuncColor}{$\triangleright$\enspace\texttt{Source({\mdseries\slshape a})\index{Source@\texttt{Source}!for IsDegreeXLayerVectorSpaceMorphism}
\label{Source:for IsDegreeXLayerVectorSpaceMorphism}
}\hfill{\scriptsize (attribute)}}\\
\textbf{\indent Returns:\ }
a DegreeXLayerVectorSpace 



 The argument is a DegreeXLayerVectorSpaceMorphism $a$. The output is its source. }

 

\subsection{\textcolor{Chapter }{Range (for IsDegreeXLayerVectorSpaceMorphism)}}
\logpage{[ 9, 4, 2 ]}\nobreak
\hyperdef{L}{X819C68C07DC49A04}{}
{\noindent\textcolor{FuncColor}{$\triangleright$\enspace\texttt{Range({\mdseries\slshape a})\index{Range@\texttt{Range}!for IsDegreeXLayerVectorSpaceMorphism}
\label{Range:for IsDegreeXLayerVectorSpaceMorphism}
}\hfill{\scriptsize (attribute)}}\\
\textbf{\indent Returns:\ }
a DegreeXLayerVectorSpace 



 The argument is a DegreeXLayerVectorSpaceMorphism $a$. The output is its range. }

 

\subsection{\textcolor{Chapter }{UnderlyingVectorSpaceMorphism (for IsDegreeXLayerVectorSpaceMorphism)}}
\logpage{[ 9, 4, 3 ]}\nobreak
\hyperdef{L}{X7C94E51E7C6185FF}{}
{\noindent\textcolor{FuncColor}{$\triangleright$\enspace\texttt{UnderlyingVectorSpaceMorphism({\mdseries\slshape a})\index{UnderlyingVectorSpaceMorphism@\texttt{UnderlyingVectorSpaceMorphism}!for IsDegreeXLayerVectorSpaceMorphism}
\label{UnderlyingVectorSpaceMorphism:for IsDegreeXLayerVectorSpaceMorphism}
}\hfill{\scriptsize (attribute)}}\\
\textbf{\indent Returns:\ }
a DegreeXLayerVectorSpace 



 The argument is a DegreeXLayerVectorSpaceMorphism $a$. The output is its range. }

 

\subsection{\textcolor{Chapter }{UnderlyingHomalgGradedRing (for IsDegreeXLayerVectorSpaceMorphism)}}
\logpage{[ 9, 4, 4 ]}\nobreak
\hyperdef{L}{X823E920E86BBD823}{}
{\noindent\textcolor{FuncColor}{$\triangleright$\enspace\texttt{UnderlyingHomalgGradedRing({\mdseries\slshape a})\index{UnderlyingHomalgGradedRing@\texttt{UnderlyingHomalgGradedRing}!for IsDegreeXLayerVectorSpaceMorphism}
\label{UnderlyingHomalgGradedRing:for IsDegreeXLayerVectorSpaceMorphism}
}\hfill{\scriptsize (attribute)}}\\
\textbf{\indent Returns:\ }
a homalg graded ring 



 The argument is a DegreeXLayerVectorSpaceMorphism $a$. The output is the Coxring, in which the source and range of this is morphism
are embedded. }

 }

 
\section{\textcolor{Chapter }{Attributes for DegreeXLayerVectorSpacePresentations}}\label{Chapter_DegreeXLayerVectorSpaces_and_morphisms_Section_Attributes_for_DegreeXLayerVectorSpacePresentations}
\logpage{[ 9, 5, 0 ]}
\hyperdef{L}{X7B98042A79573E03}{}
{
  

\subsection{\textcolor{Chapter }{UnderlyingDegreeXLayerVectorSpaceMorphism (for IsDegreeXLayerVectorSpacePresentation)}}
\logpage{[ 9, 5, 1 ]}\nobreak
\hyperdef{L}{X87A2FF337C039DFF}{}
{\noindent\textcolor{FuncColor}{$\triangleright$\enspace\texttt{UnderlyingDegreeXLayerVectorSpaceMorphism({\mdseries\slshape a})\index{UnderlyingDegreeXLayerVectorSpaceMorphism@\texttt{Underlying}\-\texttt{Degree}\-\texttt{X}\-\texttt{Layer}\-\texttt{Vector}\-\texttt{Space}\-\texttt{Morphism}!for IsDegreeXLayerVectorSpacePresentation}
\label{UnderlyingDegreeXLayerVectorSpaceMorphism:for IsDegreeXLayerVectorSpacePresentation}
}\hfill{\scriptsize (attribute)}}\\
\textbf{\indent Returns:\ }
a DegreeXLayerVectorSpaceMorphism 



 The argument is a DegreeXLayerVectorSpacePresentation $a$. The output is the underlying DegreeXLayerVectorSpaceMorphism }

 

\subsection{\textcolor{Chapter }{UnderlyingVectorSpaceObject (for IsDegreeXLayerVectorSpacePresentation)}}
\logpage{[ 9, 5, 2 ]}\nobreak
\hyperdef{L}{X79769F4883DF5646}{}
{\noindent\textcolor{FuncColor}{$\triangleright$\enspace\texttt{UnderlyingVectorSpaceObject({\mdseries\slshape a})\index{UnderlyingVectorSpaceObject@\texttt{UnderlyingVectorSpaceObject}!for IsDegreeXLayerVectorSpacePresentation}
\label{UnderlyingVectorSpaceObject:for IsDegreeXLayerVectorSpacePresentation}
}\hfill{\scriptsize (attribute)}}\\
\textbf{\indent Returns:\ }
a VectorSpaceObject 



 The argument is a DegreeXLayerVectorSpacePresentation $a$. The output is the vector space object which is the cokernel of the
underlying vector space morphism. }

 

\subsection{\textcolor{Chapter }{UnderlyingVectorSpaceMorphism (for IsDegreeXLayerVectorSpacePresentation)}}
\logpage{[ 9, 5, 3 ]}\nobreak
\hyperdef{L}{X7CCDDECA842F4D80}{}
{\noindent\textcolor{FuncColor}{$\triangleright$\enspace\texttt{UnderlyingVectorSpaceMorphism({\mdseries\slshape a})\index{UnderlyingVectorSpaceMorphism@\texttt{UnderlyingVectorSpaceMorphism}!for IsDegreeXLayerVectorSpacePresentation}
\label{UnderlyingVectorSpaceMorphism:for IsDegreeXLayerVectorSpacePresentation}
}\hfill{\scriptsize (attribute)}}\\
\textbf{\indent Returns:\ }
a VectorSpaceMorphism 



 The argument is a DegreeXLayerVectorSpacePresentation $a$. The output is the vector space morphism which defines the underlying
morphism of DegreeXLayerVectorSpaces. }

 

\subsection{\textcolor{Chapter }{UnderlyingHomalgGradedRing (for IsDegreeXLayerVectorSpacePresentation)}}
\logpage{[ 9, 5, 4 ]}\nobreak
\hyperdef{L}{X8457F800878F27BD}{}
{\noindent\textcolor{FuncColor}{$\triangleright$\enspace\texttt{UnderlyingHomalgGradedRing({\mdseries\slshape a})\index{UnderlyingHomalgGradedRing@\texttt{UnderlyingHomalgGradedRing}!for IsDegreeXLayerVectorSpacePresentation}
\label{UnderlyingHomalgGradedRing:for IsDegreeXLayerVectorSpacePresentation}
}\hfill{\scriptsize (attribute)}}\\
\textbf{\indent Returns:\ }
a homalg graded ring 



 The argument is a DegreeXLayerVectorSpacePresentation $a$. The output is the Coxring, in which the source and range of the underlying
morphism of DegreeXLayerVectorSpaces are embedded. }

 

\subsection{\textcolor{Chapter }{UnderlyingVectorSpacePresentation (for IsDegreeXLayerVectorSpacePresentation)}}
\logpage{[ 9, 5, 5 ]}\nobreak
\hyperdef{L}{X7DDC458E855BA14A}{}
{\noindent\textcolor{FuncColor}{$\triangleright$\enspace\texttt{UnderlyingVectorSpacePresentation({\mdseries\slshape a})\index{UnderlyingVectorSpacePresentation@\texttt{UnderlyingVectorSpacePresentation}!for IsDegreeXLayerVectorSpacePresentation}
\label{UnderlyingVectorSpacePresentation:for IsDegreeXLayerVectorSpacePresentation}
}\hfill{\scriptsize (attribute)}}\\
\textbf{\indent Returns:\ }
a CAP presentation category object 



 The argument is a DegreeXLayerVectorSpacePresentation $a$. The output is the underlying vector space presentation. }

 }

 
\section{\textcolor{Chapter }{Attributes for DegreeXLayerVectorSpacePresentationMorphisms}}\label{Chapter_DegreeXLayerVectorSpaces_and_morphisms_Section_Attributes_for_DegreeXLayerVectorSpacePresentationMorphisms}
\logpage{[ 9, 6, 0 ]}
\hyperdef{L}{X80B8C7DA7EEEEC40}{}
{
  

\subsection{\textcolor{Chapter }{Source (for IsDegreeXLayerVectorSpacePresentationMorphism)}}
\logpage{[ 9, 6, 1 ]}\nobreak
\hyperdef{L}{X780C00FE7C59B6E4}{}
{\noindent\textcolor{FuncColor}{$\triangleright$\enspace\texttt{Source({\mdseries\slshape a})\index{Source@\texttt{Source}!for IsDegreeXLayerVectorSpacePresentationMorphism}
\label{Source:for IsDegreeXLayerVectorSpacePresentationMorphism}
}\hfill{\scriptsize (attribute)}}\\
\textbf{\indent Returns:\ }
a DegreeXLayerVectorSpacePresentation 



 The argument is a DegreeXLayerVectorSpacePresentationMorphism $a$. The output is its source. }

 

\subsection{\textcolor{Chapter }{Range (for IsDegreeXLayerVectorSpacePresentationMorphism)}}
\logpage{[ 9, 6, 2 ]}\nobreak
\hyperdef{L}{X7DBE41747F6A3820}{}
{\noindent\textcolor{FuncColor}{$\triangleright$\enspace\texttt{Range({\mdseries\slshape a})\index{Range@\texttt{Range}!for IsDegreeXLayerVectorSpacePresentationMorphism}
\label{Range:for IsDegreeXLayerVectorSpacePresentationMorphism}
}\hfill{\scriptsize (attribute)}}\\
\textbf{\indent Returns:\ }
a DegreeXLayerVectorSpacePresentation 



 The argument is a DegreeXLayerVectorSpacePresentationMorphism $a$. The output is its range. }

 

\subsection{\textcolor{Chapter }{UnderlyingHomalgGradedRing (for IsDegreeXLayerVectorSpacePresentationMorphism)}}
\logpage{[ 9, 6, 3 ]}\nobreak
\hyperdef{L}{X86B5358C8360D470}{}
{\noindent\textcolor{FuncColor}{$\triangleright$\enspace\texttt{UnderlyingHomalgGradedRing({\mdseries\slshape a})\index{UnderlyingHomalgGradedRing@\texttt{UnderlyingHomalgGradedRing}!for IsDegreeXLayerVectorSpacePresentationMorphism}
\label{UnderlyingHomalgGradedRing:for IsDegreeXLayerVectorSpacePresentationMorphism}
}\hfill{\scriptsize (attribute)}}\\
\textbf{\indent Returns:\ }
a homalg graded ring 



 The argument is a DegreeXLayerVectorSpacePresentationMorphism $a$. The output is the underlying graded ring of its source. }

 

\subsection{\textcolor{Chapter }{UnderlyingVectorSpacePresentationMorphism (for IsDegreeXLayerVectorSpacePresentationMorphism)}}
\logpage{[ 9, 6, 4 ]}\nobreak
\hyperdef{L}{X816E0BC286D03BC4}{}
{\noindent\textcolor{FuncColor}{$\triangleright$\enspace\texttt{UnderlyingVectorSpacePresentationMorphism({\mdseries\slshape a})\index{UnderlyingVectorSpacePresentationMorphism@\texttt{Underlying}\-\texttt{Vector}\-\texttt{Space}\-\texttt{Presentation}\-\texttt{Morphism}!for IsDegreeXLayerVectorSpacePresentationMorphism}
\label{UnderlyingVectorSpacePresentationMorphism:for IsDegreeXLayerVectorSpacePresentationMorphism}
}\hfill{\scriptsize (attribute)}}\\
\textbf{\indent Returns:\ }
a CAP presentation category morphism 



 The argument is a DegreeXLayerVectorSpacePresentationMorphism $a$. The output is the underlying vector space presentation morphism. }

 }

 
\section{\textcolor{Chapter }{Convenience methods}}\label{Chapter_DegreeXLayerVectorSpaces_and_morphisms_Section_Convenience_methods}
\logpage{[ 9, 7, 0 ]}
\hyperdef{L}{X7B40ED8B78D067A5}{}
{
  

\subsection{\textcolor{Chapter }{FullInformation (for IsDegreeXLayerVectorSpacePresentation)}}
\logpage{[ 9, 7, 1 ]}\nobreak
\hyperdef{L}{X80B35AB4797BADE5}{}
{\noindent\textcolor{FuncColor}{$\triangleright$\enspace\texttt{FullInformation({\mdseries\slshape p})\index{FullInformation@\texttt{FullInformation}!for IsDegreeXLayerVectorSpacePresentation}
\label{FullInformation:for IsDegreeXLayerVectorSpacePresentation}
}\hfill{\scriptsize (operation)}}\\
\textbf{\indent Returns:\ }
detailed information about p 



 The argument is a DegreeXLayerVectorSpacePresentation $p$. This method displays $p$ in great detail. }

 

\subsection{\textcolor{Chapter }{FullInformation (for IsDegreeXLayerVectorSpacePresentationMorphism)}}
\logpage{[ 9, 7, 2 ]}\nobreak
\hyperdef{L}{X78E9436082ACC06E}{}
{\noindent\textcolor{FuncColor}{$\triangleright$\enspace\texttt{FullInformation({\mdseries\slshape p})\index{FullInformation@\texttt{FullInformation}!for IsDegreeXLayerVectorSpacePresentationMorphism}
\label{FullInformation:for IsDegreeXLayerVectorSpacePresentationMorphism}
}\hfill{\scriptsize (operation)}}\\
\textbf{\indent Returns:\ }
detailed information about p 



 The argument is a DegreeXLayerVectorSpacePresentationMorphism $p$. This method displays $p$ in great detail. }

 }

 
\section{\textcolor{Chapter }{Examples}}\label{Chapter_DegreeXLayerVectorSpaces_and_morphisms_Section_Examples}
\logpage{[ 9, 8, 0 ]}
\hyperdef{L}{X7A489A5D79DA9E5C}{}
{
  
\subsection{\textcolor{Chapter }{DegreeXLayerVectorSpaces}}\label{Chapter_DegreeXLayerVectorSpaces_and_morphisms_Section_Examples_Subsection_DegreeXLayerVectorSpaces}
\logpage{[ 9, 8, 1 ]}
\hyperdef{L}{X7B7D98C17AA4D20E}{}
{
  
\begin{Verbatim}[commandchars=!@|,fontsize=\small,frame=single,label=Example]
  !gapprompt@gap>| !gapinput@mQ := HomalgFieldOfRationals();|
  Q
  !gapprompt@gap>| !gapinput@P1 := ProjectiveSpace( 1 );|
  <A projective toric variety of dimension 1>
  !gapprompt@gap>| !gapinput@cox_ring := CoxRing( P1 );|
  Q[x_1,x_2]
  (weights: [ 1, 1 ])
  !gapprompt@gap>| !gapinput@mons := MonomsOfCoxRingOfDegreeByNormalizAsColumnMatrices|
  !gapprompt@>| !gapinput@        ( P1, [1], 1, 1 );;|
  !gapprompt@gap>| !gapinput@vector_space := VectorSpaceObject( Length( mons ), mQ );|
  <A vector space object over Q of dimension 2>
  !gapprompt@gap>| !gapinput@DXVS := DegreeXLayerVectorSpace( mons, cox_ring, vector_space, 1 );|
  <A vector space embedded into (Q[x_1,x_2] (with weights [ 1, 1 ]))^1>
  !gapprompt@gap>| !gapinput@EmbeddingDimension( DXVS );|
  1
  !gapprompt@gap>| !gapinput@Generators( DXVS );|
  [ <A 1 x 1 matrix over a graded ring>, <A 1 x 1 matrix over a graded ring> ]
\end{Verbatim}
 }

 
\subsection{\textcolor{Chapter }{Morphisms of DegreeXLayerVectorSpaces}}\label{Chapter_DegreeXLayerVectorSpaces_and_morphisms_Section_Examples_Subsection_Morphisms_of_DegreeXLayerVectorSpaces}
\logpage{[ 9, 8, 2 ]}
\hyperdef{L}{X7DA7BC87783B84EE}{}
{
  
\begin{Verbatim}[commandchars=!@|,fontsize=\small,frame=single,label=Example]
  !gapprompt@gap>| !gapinput@mons2 := Concatenation(|
  !gapprompt@>| !gapinput@         MonomsOfCoxRingOfDegreeByNormalizAsColumnMatrices|
  !gapprompt@>| !gapinput@         ( P1, [1], 1, 2 ),|
  !gapprompt@>| !gapinput@         MonomsOfCoxRingOfDegreeByNormalizAsColumnMatrices|
  !gapprompt@>| !gapinput@         ( P1, [1], 2, 2 ) );;|
  !gapprompt@gap>| !gapinput@vector_space2 := VectorSpaceObject( Length( mons2 ), mQ );|
  <A vector space object over Q of dimension 4>
  !gapprompt@gap>| !gapinput@DXVS2 := DegreeXLayerVectorSpace( mons2, cox_ring, vector_space2, 2 );|
  <A vector space embedded into (Q[x_1,x_2] (with weights [ 1, 1 ]))^2>
  !gapprompt@gap>| !gapinput@matrix := HomalgMatrix( [ [ 1, 0, 0, 0 ],|
  !gapprompt@>| !gapinput@                          [ 0, 1, 0, 0 ] ], mQ );|
  <A matrix over an internal ring>
  !gapprompt@gap>| !gapinput@vector_space_morphism := VectorSpaceMorphism( vector_space,|
  !gapprompt@>| !gapinput@                                              matrix,|
  !gapprompt@>| !gapinput@                                              vector_space2 );;|
  !gapprompt@gap>| !gapinput@IsWellDefined( vector_space_morphism );|
  true
  !gapprompt@gap>| !gapinput@morDXVS := DegreeXLayerVectorSpaceMorphism( |
  !gapprompt@>| !gapinput@           DXVS, vector_space_morphism, DXVS2 );|
  <A morphism of two vector spaces embedded into
  (suitable powers of) Q[x_1,x_2] (with weights [ 1, 1 ])>
  !gapprompt@gap>| !gapinput@UnderlyingVectorSpaceMorphism( morDXVS );|
  <A morphism in Category of matrices over Q>
  !gapprompt@gap>| !gapinput@UnderlyingHomalgGradedRing( morDXVS );|
  Q[x_1,x_2]
  (weights: [ 1, 1 ])
\end{Verbatim}
 }

 
\subsection{\textcolor{Chapter }{DegreeXLayerVectorSpacePresentations}}\label{Chapter_DegreeXLayerVectorSpaces_and_morphisms_Section_Examples_Subsection_DegreeXLayerVectorSpacePresentations}
\logpage{[ 9, 8, 3 ]}
\hyperdef{L}{X81F9E32C7EC259D6}{}
{
  
\begin{Verbatim}[commandchars=!@|,fontsize=\small,frame=single,label=Example]
  !gapprompt@gap>| !gapinput@DXVSPresentation := DegreeXLayerVectorSpacePresentation( morDXVS );|
  <A vector space embedded into (a suitable power of)
  Q[x_1,x_2] (with weights [ 1, 1 ]) given as the
  cokernel of a vector space morphism>
  !gapprompt@gap>| !gapinput@UnderlyingVectorSpaceObject( DXVSPresentation );|
  <A vector space object over Q of dimension 2>
  !gapprompt@gap>| !gapinput@relation := RelationMorphism( |
  !gapprompt@>| !gapinput@            UnderlyingVectorSpacePresentation( DXVSPresentation ) );|
  <A morphism in Category of matrices over Q>
  !gapprompt@gap>| !gapinput@m := UnderlyingMatrix( relation );|
  <A 2 x 4 matrix over an internal ring>
  !gapprompt@gap>| !gapinput@m = matrix;|
  true
\end{Verbatim}
 }

 
\subsection{\textcolor{Chapter }{Morphisms of DegreeXLayerVectorSpacePresentations}}\label{Chapter_DegreeXLayerVectorSpaces_and_morphisms_Section_Examples_Subsection_Morphisms_of_DegreeXLayerVectorSpacePresentations}
\logpage{[ 9, 8, 4 ]}
\hyperdef{L}{X7CA79C4D84AC63FD}{}
{
  
\begin{Verbatim}[commandchars=!@|,fontsize=\small,frame=single,label=Example]
  !gapprompt@gap>| !gapinput@zero_space := ZeroObject( CapCategory( vector_space ) );;|
  !gapprompt@gap>| !gapinput@source := DegreeXLayerVectorSpace( [], cox_ring, zero_space, 1 );;|
  !gapprompt@gap>| !gapinput@vector_space_morphism := ZeroMorphism( zero_space, vector_space );;|
  !gapprompt@gap>| !gapinput@morDXVS2 := DegreeXLayerVectorSpaceMorphism(|
  !gapprompt@>| !gapinput@            source, vector_space_morphism, DXVS );;|
  !gapprompt@gap>| !gapinput@DXVSPresentation2 := DegreeXLayerVectorSpacePresentation( morDXVS2 );|
  <A vector space embedded into (a suitable power of)
  Q[x_1,x_2] (with weights [ 1, 1 ]) given as the
  cokernel of a vector space morphism>
  !gapprompt@gap>| !gapinput@matrix := HomalgMatrix( [ [ 0, 0, 1, 0 ],|
  !gapprompt@>| !gapinput@                          [ 0, 0, 0, 1 ] ], mQ );|
  <A matrix over an internal ring>
  !gapprompt@gap>| !gapinput@source := Range( UnderlyingVectorSpaceMorphism( DXVSPresentation2 ) );;|
  !gapprompt@gap>| !gapinput@range := Range( UnderlyingVectorSpaceMorphism( DXVSPresentation ) );;|
  !gapprompt@gap>| !gapinput@vector_space_morphism := VectorSpaceMorphism( source, matrix, range );;|
  !gapprompt@gap>| !gapinput@IsWellDefined( vector_space_morphism );|
  true
  !gapprompt@gap>| !gapinput@DXVSPresentationMorphism := DegreeXLayerVectorSpacePresentationMorphism(|
  !gapprompt@>| !gapinput@                                      DXVSPresentation2,|
  !gapprompt@>| !gapinput@                                      vector_space_morphism,|
  !gapprompt@>| !gapinput@                                      DXVSPresentation );|
  <A vector space presentation morphism of vector spaces embedded into
  (a suitable power of) Q[x_1,x_2] (with weights [ 1, 1 ]) and given as
  cokernels>
  !gapprompt@gap>| !gapinput@uVSMor := UnderlyingVectorSpacePresentationMorphism|
  !gapprompt@>| !gapinput@                                      ( DXVSPresentationMorphism );|
  <A morphism in Freyd( Category of matrices over Q )>
  !gapprompt@gap>| !gapinput@IsWellDefined( uVSMor );|
  true
\end{Verbatim}
 }

 }

 }

   
\chapter{\textcolor{Chapter }{Truncations of graded rows and columns}}\label{Chapter_Truncations_of_graded_rows_and_columns}
\logpage{[ 10, 0, 0 ]}
\hyperdef{L}{X841A9F698680862E}{}
{
  
\section{\textcolor{Chapter }{Truncations of graded rows and columns}}\label{Chapter_Truncations_of_graded_rows_and_columns_Section_Truncations_of_graded_rows_and_columns}
\logpage{[ 10, 1, 0 ]}
\hyperdef{L}{X841A9F698680862E}{}
{
  

\subsection{\textcolor{Chapter }{TruncateGradedRowOrColumn (for IsToricVariety, IsGradedRowOrColumn, IsList, IsFieldForHomalg)}}
\logpage{[ 10, 1, 1 ]}\nobreak
\hyperdef{L}{X783DA2067F0DFFD9}{}
{\noindent\textcolor{FuncColor}{$\triangleright$\enspace\texttt{TruncateGradedRowOrColumn({\mdseries\slshape V, M, degree{\textunderscore}list, field})\index{TruncateGradedRowOrColumn@\texttt{TruncateGradedRowOrColumn}!for IsToricVariety, IsGradedRowOrColumn, IsList, IsFieldForHomalg}
\label{TruncateGradedRowOrColumn:for IsToricVariety, IsGradedRowOrColumn, IsList, IsFieldForHomalg}
}\hfill{\scriptsize (operation)}}\\
\textbf{\indent Returns:\ }
Vector space 



 The arguments are a toric variety $V$, a graded row or column $M$ over the Cox ring of $V$ and a \mbox{\texttt{\mdseries\slshape degree{\textunderscore}list}} specifying an element of the degree group of the toric variety $V$. The latter can either be specified by a list of integers or as a
HomalgModuleElement. Based on this input, the method computes the truncation
of $M$ to the specified degree. We return this finite dimensional vector space.
Optionally, we allow for a field $F$ as fourth input. This field is then used to construct the vector space.
Otherwise, we use the coefficient field of the Cox ring of $V$. }

 

\subsection{\textcolor{Chapter }{TruncateGradedRowOrColumn (for IsToricVariety, IsGradedRowOrColumn, IsHomalgModuleElement, IsFieldForHomalg)}}
\logpage{[ 10, 1, 2 ]}\nobreak
\hyperdef{L}{X7EB2EE3F7F3E6A77}{}
{\noindent\textcolor{FuncColor}{$\triangleright$\enspace\texttt{TruncateGradedRowOrColumn({\mdseries\slshape V, M, m, field})\index{TruncateGradedRowOrColumn@\texttt{TruncateGradedRowOrColumn}!for IsToricVariety, IsGradedRowOrColumn, IsHomalgModuleElement, IsFieldForHomalg}
\label{TruncateGradedRowOrColumn:for IsToricVariety, IsGradedRowOrColumn, IsHomalgModuleElement, IsFieldForHomalg}
}\hfill{\scriptsize (operation)}}\\
\textbf{\indent Returns:\ }
Vector space 



 As above, but with a HomalgModuleElement m specifying the degree. }

 

\subsection{\textcolor{Chapter }{TruncateGradedRowOrColumn (for IsToricVariety, IsGradedRowOrColumn, IsList)}}
\logpage{[ 10, 1, 3 ]}\nobreak
\hyperdef{L}{X7AF643C87EC2BCDD}{}
{\noindent\textcolor{FuncColor}{$\triangleright$\enspace\texttt{TruncateGradedRowOrColumn({\mdseries\slshape V, M, degree})\index{TruncateGradedRowOrColumn@\texttt{TruncateGradedRowOrColumn}!for IsToricVariety, IsGradedRowOrColumn, IsList}
\label{TruncateGradedRowOrColumn:for IsToricVariety, IsGradedRowOrColumn, IsList}
}\hfill{\scriptsize (operation)}}\\
\textbf{\indent Returns:\ }
Vector space 



 As above, but the coefficient ring of the Cox ring will be used as field }

 

\subsection{\textcolor{Chapter }{TruncateGradedRowOrColumn (for IsToricVariety, IsGradedRowOrColumn, IsHomalgModuleElement)}}
\logpage{[ 10, 1, 4 ]}\nobreak
\hyperdef{L}{X790B952987279B54}{}
{\noindent\textcolor{FuncColor}{$\triangleright$\enspace\texttt{TruncateGradedRowOrColumn({\mdseries\slshape V, M, m})\index{TruncateGradedRowOrColumn@\texttt{TruncateGradedRowOrColumn}!for IsToricVariety, IsGradedRowOrColumn, IsHomalgModuleElement}
\label{TruncateGradedRowOrColumn:for IsToricVariety, IsGradedRowOrColumn, IsHomalgModuleElement}
}\hfill{\scriptsize (operation)}}\\
\textbf{\indent Returns:\ }
Vector space 



 As above, but a HomalgModuleElement m specifies the degree and we use the
coefficient ring of the Cox ring as field. }

 

\subsection{\textcolor{Chapter }{DegreeXLayerOfGradedRowOrColumn (for IsToricVariety, IsGradedRowOrColumn, IsList, IsFieldForHomalg)}}
\logpage{[ 10, 1, 5 ]}\nobreak
\hyperdef{L}{X7F50402585C2E6BB}{}
{\noindent\textcolor{FuncColor}{$\triangleright$\enspace\texttt{DegreeXLayerOfGradedRowOrColumn({\mdseries\slshape V, M, degree{\textunderscore}list, field})\index{DegreeXLayerOfGradedRowOrColumn@\texttt{DegreeXLayerOfGradedRowOrColumn}!for IsToricVariety, IsGradedRowOrColumn, IsList, IsFieldForHomalg}
\label{DegreeXLayerOfGradedRowOrColumn:for IsToricVariety, IsGradedRowOrColumn, IsList, IsFieldForHomalg}
}\hfill{\scriptsize (operation)}}\\
\textbf{\indent Returns:\ }
DegreeXLayerVectorSpace 



 The arguments are a toric variety $V$, a graded row or column $M$ over the Cox ring of $V$ and a \mbox{\texttt{\mdseries\slshape degree{\textunderscore}list}} specifying an element of the degree group of the toric variety $V$. The latter can either be specified by a list of integers or as a
HomalgModuleElement. Based on this input, the method computes the truncation
of $M$ to the specified degree. This is a finite dimensional vector space. We return
the corresponding DegreeXLayerVectorSpace. Optionally, we allow for a field $F$ as fourth input. This field is used to construct the DegreeXLayerVectorSpace.
Namely, the wrapper DegreeXLayerVectorSpace contains a representation of the
obtained vector space as $F^n$. In case $F$ is specified, we use this particular field. Otherwise,
HomalgFieldOfRationals() will be used. }

 

\subsection{\textcolor{Chapter }{DegreeXLayerOfGradedRowOrColumn (for IsToricVariety, IsGradedRowOrColumn, IsHomalgModuleElement, IsFieldForHomalg)}}
\logpage{[ 10, 1, 6 ]}\nobreak
\hyperdef{L}{X84D06626846FA1CF}{}
{\noindent\textcolor{FuncColor}{$\triangleright$\enspace\texttt{DegreeXLayerOfGradedRowOrColumn({\mdseries\slshape V, M, m, field})\index{DegreeXLayerOfGradedRowOrColumn@\texttt{DegreeXLayerOfGradedRowOrColumn}!for IsToricVariety, IsGradedRowOrColumn, IsHomalgModuleElement, IsFieldForHomalg}
\label{DegreeXLayerOfGradedRowOrColumn:for IsToricVariety, IsGradedRowOrColumn, IsHomalgModuleElement, IsFieldForHomalg}
}\hfill{\scriptsize (operation)}}\\
\textbf{\indent Returns:\ }
DegreeXLayerVectorSpace 



 As above, but with a HomalgModuleElement m specifying the degree. }

 

\subsection{\textcolor{Chapter }{DegreeXLayerOfGradedRowOrColumn (for IsToricVariety, IsGradedRowOrColumn, IsList)}}
\logpage{[ 10, 1, 7 ]}\nobreak
\hyperdef{L}{X84187E1E7BEA2930}{}
{\noindent\textcolor{FuncColor}{$\triangleright$\enspace\texttt{DegreeXLayerOfGradedRowOrColumn({\mdseries\slshape V, M, degree})\index{DegreeXLayerOfGradedRowOrColumn@\texttt{DegreeXLayerOfGradedRowOrColumn}!for IsToricVariety, IsGradedRowOrColumn, IsList}
\label{DegreeXLayerOfGradedRowOrColumn:for IsToricVariety, IsGradedRowOrColumn, IsList}
}\hfill{\scriptsize (operation)}}\\
\textbf{\indent Returns:\ }
DegreeXLayerVectorSpace 



 As above, but the coefficient ring of the Cox ring will be used as field }

 

\subsection{\textcolor{Chapter }{DegreeXLayerOfGradedRowOrColumn (for IsToricVariety, IsGradedRowOrColumn, IsHomalgModuleElement)}}
\logpage{[ 10, 1, 8 ]}\nobreak
\hyperdef{L}{X79E811857FCEF7C9}{}
{\noindent\textcolor{FuncColor}{$\triangleright$\enspace\texttt{DegreeXLayerOfGradedRowOrColumn({\mdseries\slshape V, M, m})\index{DegreeXLayerOfGradedRowOrColumn@\texttt{DegreeXLayerOfGradedRowOrColumn}!for IsToricVariety, IsGradedRowOrColumn, IsHomalgModuleElement}
\label{DegreeXLayerOfGradedRowOrColumn:for IsToricVariety, IsGradedRowOrColumn, IsHomalgModuleElement}
}\hfill{\scriptsize (operation)}}\\
\textbf{\indent Returns:\ }
DegreeXLayerVectorSpace 



 As above, but a HomalgModuleElement m specifies the degree and we use the
coefficient ring of the Cox ring as field. }

 }

 
\section{\textcolor{Chapter }{Formats for generators of truncations of graded rows and columns}}\label{Chapter_Truncations_of_graded_rows_and_columns_Section_Formats_for_generators_of_truncations_of_graded_rows_and_columns}
\logpage{[ 10, 2, 0 ]}
\hyperdef{L}{X7F5EB67D81E29D83}{}
{
  

\subsection{\textcolor{Chapter }{GeneratorsOfDegreeXLayerOfGradedRowOrColumnAsListOfColumnMatrices (for IsToricVariety, IsGradedRowOrColumn, IsList)}}
\logpage{[ 10, 2, 1 ]}\nobreak
\hyperdef{L}{X828F7AB57A0AFA79}{}
{\noindent\textcolor{FuncColor}{$\triangleright$\enspace\texttt{GeneratorsOfDegreeXLayerOfGradedRowOrColumnAsListOfColumnMatrices({\mdseries\slshape V, M, l})\index{GeneratorsOfDegreeXLayerOfGradedRowOrColumnAsListOfColumnMatrices@\texttt{Generators}\-\texttt{Of}\-\texttt{Degree}\-\texttt{X}\-\texttt{Layer}\-\texttt{Of}\-\texttt{Graded}\-\texttt{Row}\-\texttt{Or}\-\texttt{Column}\-\texttt{As}\-\texttt{List}\-\texttt{Of}\-\texttt{Column}\-\texttt{Matrices}!for IsToricVariety, IsGradedRowOrColumn, IsList}
\label{GeneratorsOfDegreeXLayerOfGradedRowOrColumnAsListOfColumnMatrices:for IsToricVariety, IsGradedRowOrColumn, IsList}
}\hfill{\scriptsize (operation)}}\\
\textbf{\indent Returns:\ }
a list 



 The arguments are a variety V, a graded row or column M and a list l,
specifying a degree in the class group of the Cox ring of $V$. We then compute the truncation of M to the specified degree and return its
generators as list of column matrices. }

 

\subsection{\textcolor{Chapter }{GeneratorsOfDegreeXLayerOfGradedRowOrColumnAsListOfColumnMatrices (for IsToricVariety, IsGradedRowOrColumn, IsHomalgModuleElement)}}
\logpage{[ 10, 2, 2 ]}\nobreak
\hyperdef{L}{X828A6FFD87EBC178}{}
{\noindent\textcolor{FuncColor}{$\triangleright$\enspace\texttt{GeneratorsOfDegreeXLayerOfGradedRowOrColumnAsListOfColumnMatrices({\mdseries\slshape V, M, m})\index{GeneratorsOfDegreeXLayerOfGradedRowOrColumnAsListOfColumnMatrices@\texttt{Generators}\-\texttt{Of}\-\texttt{Degree}\-\texttt{X}\-\texttt{Layer}\-\texttt{Of}\-\texttt{Graded}\-\texttt{Row}\-\texttt{Or}\-\texttt{Column}\-\texttt{As}\-\texttt{List}\-\texttt{Of}\-\texttt{Column}\-\texttt{Matrices}!for IsToricVariety, IsGradedRowOrColumn, IsHomalgModuleElement}
\label{GeneratorsOfDegreeXLayerOfGradedRowOrColumnAsListOfColumnMatrices:for IsToricVariety, IsGradedRowOrColumn, IsHomalgModuleElement}
}\hfill{\scriptsize (operation)}}\\
\textbf{\indent Returns:\ }
a list 



 The arguments are a variety V, a graded row or column M and a
HomalgModuleElement m, specifying a degree in the class group of the Cox ring
of $V$. We then compute the truncation of M to the specified degree and return its
generators as list of column matrices. }

 

\subsection{\textcolor{Chapter }{GeneratorsOfDegreeXLayerOfGradedRowOrColumnAsUnionOfColumnMatrices (for IsToricVariety, IsGradedRowOrColumn, IsList)}}
\logpage{[ 10, 2, 3 ]}\nobreak
\hyperdef{L}{X7843935480BC0942}{}
{\noindent\textcolor{FuncColor}{$\triangleright$\enspace\texttt{GeneratorsOfDegreeXLayerOfGradedRowOrColumnAsUnionOfColumnMatrices({\mdseries\slshape V, M, m})\index{GeneratorsOfDegreeXLayerOfGradedRowOrColumnAsUnionOfColumnMatrices@\texttt{Generators}\-\texttt{Of}\-\texttt{Degree}\-\texttt{X}\-\texttt{Layer}\-\texttt{Of}\-\texttt{Graded}\-\texttt{Row}\-\texttt{Or}\-\texttt{Column}\-\texttt{As}\-\texttt{Union}\-\texttt{Of}\-\texttt{Column}\-\texttt{Matrices}!for IsToricVariety, IsGradedRowOrColumn, IsList}
\label{GeneratorsOfDegreeXLayerOfGradedRowOrColumnAsUnionOfColumnMatrices:for IsToricVariety, IsGradedRowOrColumn, IsList}
}\hfill{\scriptsize (operation)}}\\
\textbf{\indent Returns:\ }
a list 



 The arguments are a variety V, a graded row or column M and a list l,
specifying a degree in the class group of the Cox ring of $V$. We then compute the truncation of M to the specified degree and its
generators as column matrices. The matrix formed from the union of these
column matrices is returned. }

 

\subsection{\textcolor{Chapter }{GeneratorsOfDegreeXLayerOfGradedRowOrColumnAsUnionOfColumnMatrices (for IsToricVariety, IsGradedRowOrColumn, IsHomalgModuleElement)}}
\logpage{[ 10, 2, 4 ]}\nobreak
\hyperdef{L}{X83510BF7866C3637}{}
{\noindent\textcolor{FuncColor}{$\triangleright$\enspace\texttt{GeneratorsOfDegreeXLayerOfGradedRowOrColumnAsUnionOfColumnMatrices({\mdseries\slshape V, M, m})\index{GeneratorsOfDegreeXLayerOfGradedRowOrColumnAsUnionOfColumnMatrices@\texttt{Generators}\-\texttt{Of}\-\texttt{Degree}\-\texttt{X}\-\texttt{Layer}\-\texttt{Of}\-\texttt{Graded}\-\texttt{Row}\-\texttt{Or}\-\texttt{Column}\-\texttt{As}\-\texttt{Union}\-\texttt{Of}\-\texttt{Column}\-\texttt{Matrices}!for IsToricVariety, IsGradedRowOrColumn, IsHomalgModuleElement}
\label{GeneratorsOfDegreeXLayerOfGradedRowOrColumnAsUnionOfColumnMatrices:for IsToricVariety, IsGradedRowOrColumn, IsHomalgModuleElement}
}\hfill{\scriptsize (operation)}}\\
\textbf{\indent Returns:\ }
a list 



 The arguments are a variety V, a graded row or column M and a
HomalgModuleElement m, specifying a degree in the class group of the Cox ring
of $V$. We then compute the truncation of M to the specified degree and its
generators as column matrices. The matrix formed from the union of these
column matrices is returned. }

 

\subsection{\textcolor{Chapter }{GeneratorsOfDegreeXLayerOfGradedRowOrColumnAsListsOfRecords (for IsToricVariety, IsGradedRowOrColumn, IsList)}}
\logpage{[ 10, 2, 5 ]}\nobreak
\hyperdef{L}{X7D78A75879BF12A9}{}
{\noindent\textcolor{FuncColor}{$\triangleright$\enspace\texttt{GeneratorsOfDegreeXLayerOfGradedRowOrColumnAsListsOfRecords({\mdseries\slshape V, M, l})\index{GeneratorsOfDegreeXLayerOfGradedRowOrColumnAsListsOfRecords@\texttt{Generators}\-\texttt{Of}\-\texttt{Degree}\-\texttt{X}\-\texttt{Layer}\-\texttt{Of}\-\texttt{Graded}\-\texttt{Row}\-\texttt{Or}\-\texttt{Column}\-\texttt{As}\-\texttt{Lists}\-\texttt{Of}\-\texttt{Records}!for IsToricVariety, IsGradedRowOrColumn, IsList}
\label{GeneratorsOfDegreeXLayerOfGradedRowOrColumnAsListsOfRecords:for IsToricVariety, IsGradedRowOrColumn, IsList}
}\hfill{\scriptsize (operation)}}\\
\textbf{\indent Returns:\ }
a list 



 The arguments are a variety V, a graded row or column M and a list l,
specifying a degree in the class group of the Cox ring of $V$. We then compute the truncation of M to the specified degree and return its
generators as list [ n, rec{\textunderscore}list ]. n specifies the number of
generators. rec{\textunderscore}list is a list of record. The i-th record
contains the generators of the i-th direct summand of M. }

 The arguments are a variety V, a graded row or column M and a
HomalgModuleElement m, specifying a degree in the class group of the Cox ring
of $V$. We then compute the truncation of M to the specified degree and return its
generators as list [ n, rec{\textunderscore}list ]. n specifies the number of
generators. rec{\textunderscore}list is a list of record. The i-th record
contains the generators of the i-th direct summand of M. 

\subsection{\textcolor{Chapter }{GeneratorsOfDegreeXLayerOfGradedRowOrColumnAsListsOfRecords (for IsToricVariety, IsGradedRowOrColumn, IsHomalgModuleElement)}}
\logpage{[ 10, 2, 6 ]}\nobreak
\hyperdef{L}{X783133BB7B55969C}{}
{\noindent\textcolor{FuncColor}{$\triangleright$\enspace\texttt{GeneratorsOfDegreeXLayerOfGradedRowOrColumnAsListsOfRecords({\mdseries\slshape V, M, m})\index{GeneratorsOfDegreeXLayerOfGradedRowOrColumnAsListsOfRecords@\texttt{Generators}\-\texttt{Of}\-\texttt{Degree}\-\texttt{X}\-\texttt{Layer}\-\texttt{Of}\-\texttt{Graded}\-\texttt{Row}\-\texttt{Or}\-\texttt{Column}\-\texttt{As}\-\texttt{Lists}\-\texttt{Of}\-\texttt{Records}!for IsToricVariety, IsGradedRowOrColumn, IsHomalgModuleElement}
\label{GeneratorsOfDegreeXLayerOfGradedRowOrColumnAsListsOfRecords:for IsToricVariety, IsGradedRowOrColumn, IsHomalgModuleElement}
}\hfill{\scriptsize (operation)}}\\
\textbf{\indent Returns:\ }
a list 



 

 }

 

\subsection{\textcolor{Chapter }{GeneratorsOfDegreeXLayerOfGradedRowOrColumnAsListList (for IsToricVariety, IsGradedRowOrColumn, IsList)}}
\logpage{[ 10, 2, 7 ]}\nobreak
\hyperdef{L}{X82C81C9678844CFC}{}
{\noindent\textcolor{FuncColor}{$\triangleright$\enspace\texttt{GeneratorsOfDegreeXLayerOfGradedRowOrColumnAsListList({\mdseries\slshape V, M, l})\index{GeneratorsOfDegreeXLayerOfGradedRowOrColumnAsListList@\texttt{Generators}\-\texttt{Of}\-\texttt{Degree}\-\texttt{X}\-\texttt{Layer}\-\texttt{Of}\-\texttt{Graded}\-\texttt{Row}\-\texttt{Or}\-\texttt{Column}\-\texttt{As}\-\texttt{List}\-\texttt{List}!for IsToricVariety, IsGradedRowOrColumn, IsList}
\label{GeneratorsOfDegreeXLayerOfGradedRowOrColumnAsListList:for IsToricVariety, IsGradedRowOrColumn, IsList}
}\hfill{\scriptsize (operation)}}\\
\textbf{\indent Returns:\ }
a list 



 The arguments are a variety V, a graded row or column M and a list l,
specifying a degree in the class group of the Cox ring of $V$. We then compute the truncation of M to the specified degree and identify its
generators. We format each generator as list [ n, g ], where g denotes a
generator of the n-th direct summand of M. We return the list of all these
lists [ n, g ]. }

 

\subsection{\textcolor{Chapter }{GeneratorsOfDegreeXLayerOfGradedRowOrColumnAsListList (for IsToricVariety, IsGradedRowOrColumn, IsHomalgModuleElement)}}
\logpage{[ 10, 2, 8 ]}\nobreak
\hyperdef{L}{X7E6D79AC802768E0}{}
{\noindent\textcolor{FuncColor}{$\triangleright$\enspace\texttt{GeneratorsOfDegreeXLayerOfGradedRowOrColumnAsListList({\mdseries\slshape V, M, m})\index{GeneratorsOfDegreeXLayerOfGradedRowOrColumnAsListList@\texttt{Generators}\-\texttt{Of}\-\texttt{Degree}\-\texttt{X}\-\texttt{Layer}\-\texttt{Of}\-\texttt{Graded}\-\texttt{Row}\-\texttt{Or}\-\texttt{Column}\-\texttt{As}\-\texttt{List}\-\texttt{List}!for IsToricVariety, IsGradedRowOrColumn, IsHomalgModuleElement}
\label{GeneratorsOfDegreeXLayerOfGradedRowOrColumnAsListList:for IsToricVariety, IsGradedRowOrColumn, IsHomalgModuleElement}
}\hfill{\scriptsize (operation)}}\\
\textbf{\indent Returns:\ }
a list 



 The arguments are a variety V, a graded row or column M and a
HomalgModuleElement m, specifying a degree in the class group of the Cox ring
of $V$. We then compute the truncation of M to the specified degree and identify its
generators. We format each generator as list [ n, g ], where g denotes a
generator of the n-th direct summand of M. We return the list of all these
lists [ n, g ]. }

 }

 
\section{\textcolor{Chapter }{Truncations of graded row and column morphisms}}\label{Chapter_Truncations_of_graded_rows_and_columns_Section_Truncations_of_graded_row_and_column_morphisms}
\logpage{[ 10, 3, 0 ]}
\hyperdef{L}{X7F779F5C786B932D}{}
{
  

\subsection{\textcolor{Chapter }{TruncateGradedRowOrColumnMorphism (for IsToricVariety, IsGradedRowOrColumnMorphism, IsList, IsBool, IsFieldForHomalg)}}
\logpage{[ 10, 3, 1 ]}\nobreak
\hyperdef{L}{X8099F9B285E69DAD}{}
{\noindent\textcolor{FuncColor}{$\triangleright$\enspace\texttt{TruncateGradedRowOrColumnMorphism({\mdseries\slshape V, a, d, B, F})\index{TruncateGradedRowOrColumnMorphism@\texttt{TruncateGradedRowOrColumnMorphism}!for IsToricVariety, IsGradedRowOrColumnMorphism, IsList, IsBool, IsFieldForHomalg}
\label{TruncateGradedRowOrColumnMorphism:for IsToricVariety, IsGradedRowOrColumnMorphism, IsList, IsBool, IsFieldForHomalg}
}\hfill{\scriptsize (operation)}}\\
\textbf{\indent Returns:\ }
a vector space morphism 



 The arguments are a toric variety $V$, a morphism $a$ of graded rows or columns, a list $d$ specifying a degree in the class group of $V$, a field $F$ for homalg and a boolean $B$. We then truncate $m$ to the specified degree $d$. We express this result as morphism of vector spaces over the field $F$. We return this vector space morphism. If the boolean $B$ is true, we display additional output during the computation, otherwise this
output is surpressed. }

 

\subsection{\textcolor{Chapter }{TruncateGradedRowOrColumnMorphism (for IsToricVariety, IsGradedRowOrColumnMorphism, IsHomalgModuleElement, IsBool, IsHomalgRing)}}
\logpage{[ 10, 3, 2 ]}\nobreak
\hyperdef{L}{X82AC7ACE816136C3}{}
{\noindent\textcolor{FuncColor}{$\triangleright$\enspace\texttt{TruncateGradedRowOrColumnMorphism({\mdseries\slshape V, a, m, B, F})\index{TruncateGradedRowOrColumnMorphism@\texttt{TruncateGradedRowOrColumnMorphism}!for IsToricVariety, IsGradedRowOrColumnMorphism, IsHomalgModuleElement, IsBool, IsHomalgRing}
\label{TruncateGradedRowOrColumnMorphism:for IsToricVariety, IsGradedRowOrColumnMorphism, IsHomalgModuleElement, IsBool, IsHomalgRing}
}\hfill{\scriptsize (operation)}}\\
\textbf{\indent Returns:\ }
a vector space morphism 



 The arguments are a toric variety $V$, a morphism $a$ of graded rows or columns, and a HomalgModuleElement $m$ specifying a degree in the class group of $V$, a field $F$ for homalg and a boolean $B$. We then truncate $m$ to the specified degree $d$. We express this result as morphism of vector spaces over the field $F$. We return this vector space morphism. If the boolean $B$ is true, we display additional output during the computation, otherwise this
output is surpressed. }

 

\subsection{\textcolor{Chapter }{TruncateGradedRowOrColumnMorphism (for IsToricVariety, IsGradedRowOrColumnMorphism, IsList, IsBool)}}
\logpage{[ 10, 3, 3 ]}\nobreak
\hyperdef{L}{X7C450E4C7DA93253}{}
{\noindent\textcolor{FuncColor}{$\triangleright$\enspace\texttt{TruncateGradedRowOrColumnMorphism({\mdseries\slshape V, a, d, B})\index{TruncateGradedRowOrColumnMorphism@\texttt{TruncateGradedRowOrColumnMorphism}!for IsToricVariety, IsGradedRowOrColumnMorphism, IsList, IsBool}
\label{TruncateGradedRowOrColumnMorphism:for IsToricVariety, IsGradedRowOrColumnMorphism, IsList, IsBool}
}\hfill{\scriptsize (operation)}}\\
\textbf{\indent Returns:\ }
a vector space morphism 



 This method operates just as 'TruncateGradedRowOrColumnMorphism' above.
However, here the field F is taken as the field of coefficients of the Cox
ring of the variety $V$. }

 

\subsection{\textcolor{Chapter }{TruncateGradedRowOrColumnMorphism (for IsToricVariety, IsGradedRowOrColumnMorphism, IsHomalgModuleElement, IsBool)}}
\logpage{[ 10, 3, 4 ]}\nobreak
\hyperdef{L}{X81879E1A81638CD6}{}
{\noindent\textcolor{FuncColor}{$\triangleright$\enspace\texttt{TruncateGradedRowOrColumnMorphism({\mdseries\slshape V, a, m, B})\index{TruncateGradedRowOrColumnMorphism@\texttt{TruncateGradedRowOrColumnMorphism}!for IsToricVariety, IsGradedRowOrColumnMorphism, IsHomalgModuleElement, IsBool}
\label{TruncateGradedRowOrColumnMorphism:for IsToricVariety, IsGradedRowOrColumnMorphism, IsHomalgModuleElement, IsBool}
}\hfill{\scriptsize (operation)}}\\
\textbf{\indent Returns:\ }
a vector space morphism 



 This method operates just as 'TruncateGradedRowOrColumnMorphism' above.
However, here the field F is taken as the field of coefficients of the Cox
ring of the variety $V$. }

 

\subsection{\textcolor{Chapter }{TruncateGradedRowOrColumnMorphism (for IsToricVariety, IsGradedRowOrColumnMorphism, IsList)}}
\logpage{[ 10, 3, 5 ]}\nobreak
\hyperdef{L}{X7E50E305806AB496}{}
{\noindent\textcolor{FuncColor}{$\triangleright$\enspace\texttt{TruncateGradedRowOrColumnMorphism({\mdseries\slshape V, a, d})\index{TruncateGradedRowOrColumnMorphism@\texttt{TruncateGradedRowOrColumnMorphism}!for IsToricVariety, IsGradedRowOrColumnMorphism, IsList}
\label{TruncateGradedRowOrColumnMorphism:for IsToricVariety, IsGradedRowOrColumnMorphism, IsList}
}\hfill{\scriptsize (operation)}}\\
\textbf{\indent Returns:\ }
a vector space morphism 



 This method operates just as 'TruncateGradedRowOrColumnMorphism' above.
However, here the field F is taken as the field of coefficients of the Cox
ring of the variety $V$. Also, B is set to false, i.e. no additional information is being displayed. }

 

\subsection{\textcolor{Chapter }{TruncateGradedRowOrColumnMorphism (for IsToricVariety, IsGradedRowOrColumnMorphism, IsHomalgModuleElement)}}
\logpage{[ 10, 3, 6 ]}\nobreak
\hyperdef{L}{X879D92CA85D4A0C6}{}
{\noindent\textcolor{FuncColor}{$\triangleright$\enspace\texttt{TruncateGradedRowOrColumnMorphism({\mdseries\slshape V, a, m})\index{TruncateGradedRowOrColumnMorphism@\texttt{TruncateGradedRowOrColumnMorphism}!for IsToricVariety, IsGradedRowOrColumnMorphism, IsHomalgModuleElement}
\label{TruncateGradedRowOrColumnMorphism:for IsToricVariety, IsGradedRowOrColumnMorphism, IsHomalgModuleElement}
}\hfill{\scriptsize (operation)}}\\
\textbf{\indent Returns:\ }
a vector space morphism 



 This method operates just as 'TruncateGradedRowOrColumnMorphism' above.
However, here the field F is taken as the field of coefficients of the Cox
ring of the variety $V$. Also, B is set to false, i.e. no additional information is being displayed. }

 

\subsection{\textcolor{Chapter }{DegreeXLayerOfGradedRowOrColumnMorphism (for IsToricVariety, IsGradedRowOrColumnMorphism, IsList, IsFieldForHomalg, IsBool)}}
\logpage{[ 10, 3, 7 ]}\nobreak
\hyperdef{L}{X798284867B18B7CD}{}
{\noindent\textcolor{FuncColor}{$\triangleright$\enspace\texttt{DegreeXLayerOfGradedRowOrColumnMorphism({\mdseries\slshape V, a, d, F, B})\index{DegreeXLayerOfGradedRowOrColumnMorphism@\texttt{Degree}\-\texttt{X}\-\texttt{Layer}\-\texttt{Of}\-\texttt{Graded}\-\texttt{Row}\-\texttt{Or}\-\texttt{Column}\-\texttt{Morphism}!for IsToricVariety, IsGradedRowOrColumnMorphism, IsList, IsFieldForHomalg, IsBool}
\label{DegreeXLayerOfGradedRowOrColumnMorphism:for IsToricVariety, IsGradedRowOrColumnMorphism, IsList, IsFieldForHomalg, IsBool}
}\hfill{\scriptsize (operation)}}\\
\textbf{\indent Returns:\ }
a DegreeXLayerVectorSpaceMorphism 



 The arguments are a toric variety $V$, a morphism $a$ of graded rows or columns, a list $d$ specifying a degree in the class group of $V$, a field $F$ for homalg and a boolean $B$. We then truncate $m$ to the specified degree $d$. We express this result as morphism of vector spaces over the field $F$. We return the corresponding DegreeXLayerVectorSpaceMorphism. If the boolean $B$ is true, we display additional output during the computation, otherwise this
output is surpressed. }

 

\subsection{\textcolor{Chapter }{DegreeXLayerOfGradedRowOrColumnMorphism (for IsToricVariety, IsGradedRowOrColumnMorphism, IsHomalgModuleElement, IsHomalgRing, IsBool)}}
\logpage{[ 10, 3, 8 ]}\nobreak
\hyperdef{L}{X78DC1C42791EA99B}{}
{\noindent\textcolor{FuncColor}{$\triangleright$\enspace\texttt{DegreeXLayerOfGradedRowOrColumnMorphism({\mdseries\slshape V, a, m, F, B})\index{DegreeXLayerOfGradedRowOrColumnMorphism@\texttt{Degree}\-\texttt{X}\-\texttt{Layer}\-\texttt{Of}\-\texttt{Graded}\-\texttt{Row}\-\texttt{Or}\-\texttt{Column}\-\texttt{Morphism}!for IsToricVariety, IsGradedRowOrColumnMorphism, IsHomalgModuleElement, IsHomalgRing, IsBool}
\label{DegreeXLayerOfGradedRowOrColumnMorphism:for IsToricVariety, IsGradedRowOrColumnMorphism, IsHomalgModuleElement, IsHomalgRing, IsBool}
}\hfill{\scriptsize (operation)}}\\
\textbf{\indent Returns:\ }
a DegreeXLayerVectorSpaceMorphism 



 The arguments are a toric variety $V$, a morphism $a$ of graded rows or columns, a HomalgModuleElement $m$ specifying a degree in the class group of $V$, a field $F$ for homalg and a boolean $B$. We then truncate $m$ to the specified degree $d$. We express this result as morphism of vector spaces over the field $F$. We return the corresponding DegreeXLayerVectorSpaceMorphism. If the boolean $B$ is true, we display additional output during the computation, otherwise this
output is surpressed. }

 

\subsection{\textcolor{Chapter }{DegreeXLayerOfGradedRowOrColumnMorphism (for IsToricVariety, IsGradedRowOrColumnMorphism, IsList, IsBool)}}
\logpage{[ 10, 3, 9 ]}\nobreak
\hyperdef{L}{X7871BB5886F6DAA0}{}
{\noindent\textcolor{FuncColor}{$\triangleright$\enspace\texttt{DegreeXLayerOfGradedRowOrColumnMorphism({\mdseries\slshape V, a, d, B})\index{DegreeXLayerOfGradedRowOrColumnMorphism@\texttt{Degree}\-\texttt{X}\-\texttt{Layer}\-\texttt{Of}\-\texttt{Graded}\-\texttt{Row}\-\texttt{Or}\-\texttt{Column}\-\texttt{Morphism}!for IsToricVariety, IsGradedRowOrColumnMorphism, IsList, IsBool}
\label{DegreeXLayerOfGradedRowOrColumnMorphism:for IsToricVariety, IsGradedRowOrColumnMorphism, IsList, IsBool}
}\hfill{\scriptsize (operation)}}\\
\textbf{\indent Returns:\ }
a vector space morphism 



 This method operates just as 'DegreeXLayerOfGradedRowOrColumnMorphism' above.
However, here the field F is taken as the field of coefficients of the Cox
ring of the variety $V$. }

 

\subsection{\textcolor{Chapter }{DegreeXLayerOfGradedRowOrColumnMorphism (for IsToricVariety, IsGradedRowOrColumnMorphism, IsHomalgModuleElement, IsBool)}}
\logpage{[ 10, 3, 10 ]}\nobreak
\hyperdef{L}{X7DDB94AA85E978E0}{}
{\noindent\textcolor{FuncColor}{$\triangleright$\enspace\texttt{DegreeXLayerOfGradedRowOrColumnMorphism({\mdseries\slshape V, a, m, B})\index{DegreeXLayerOfGradedRowOrColumnMorphism@\texttt{Degree}\-\texttt{X}\-\texttt{Layer}\-\texttt{Of}\-\texttt{Graded}\-\texttt{Row}\-\texttt{Or}\-\texttt{Column}\-\texttt{Morphism}!for IsToricVariety, IsGradedRowOrColumnMorphism, IsHomalgModuleElement, IsBool}
\label{DegreeXLayerOfGradedRowOrColumnMorphism:for IsToricVariety, IsGradedRowOrColumnMorphism, IsHomalgModuleElement, IsBool}
}\hfill{\scriptsize (operation)}}\\
\textbf{\indent Returns:\ }
a vector space morphism 



 This method operates just as 'DegreeXLayerOfGradedRowOrColumnMorphism' above.
However, here the field F is taken as the field of coefficients of the Cox
ring of the variety $V$. }

 

\subsection{\textcolor{Chapter }{DegreeXLayerOfGradedRowOrColumnMorphism (for IsToricVariety, IsGradedRowOrColumnMorphism, IsList)}}
\logpage{[ 10, 3, 11 ]}\nobreak
\hyperdef{L}{X7E76D9B182FB607C}{}
{\noindent\textcolor{FuncColor}{$\triangleright$\enspace\texttt{DegreeXLayerOfGradedRowOrColumnMorphism({\mdseries\slshape V, a, d})\index{DegreeXLayerOfGradedRowOrColumnMorphism@\texttt{Degree}\-\texttt{X}\-\texttt{Layer}\-\texttt{Of}\-\texttt{Graded}\-\texttt{Row}\-\texttt{Or}\-\texttt{Column}\-\texttt{Morphism}!for IsToricVariety, IsGradedRowOrColumnMorphism, IsList}
\label{DegreeXLayerOfGradedRowOrColumnMorphism:for IsToricVariety, IsGradedRowOrColumnMorphism, IsList}
}\hfill{\scriptsize (operation)}}\\
\textbf{\indent Returns:\ }
a vector space morphism 



 This method operates just as 'DegreeXLayerOfGradedRowOrColumnMorphism' above.
However, here the field F is taken as the field of coefficients of the Cox
ring of the variety $V$. Also, B is set to false, i.e. no additional information is being displayed. }

 

\subsection{\textcolor{Chapter }{DegreeXLayerOfGradedRowOrColumnMorphism (for IsToricVariety, IsGradedRowOrColumnMorphism, IsHomalgModuleElement)}}
\logpage{[ 10, 3, 12 ]}\nobreak
\hyperdef{L}{X7ED0FD307C64DF78}{}
{\noindent\textcolor{FuncColor}{$\triangleright$\enspace\texttt{DegreeXLayerOfGradedRowOrColumnMorphism({\mdseries\slshape V, a, m})\index{DegreeXLayerOfGradedRowOrColumnMorphism@\texttt{Degree}\-\texttt{X}\-\texttt{Layer}\-\texttt{Of}\-\texttt{Graded}\-\texttt{Row}\-\texttt{Or}\-\texttt{Column}\-\texttt{Morphism}!for IsToricVariety, IsGradedRowOrColumnMorphism, IsHomalgModuleElement}
\label{DegreeXLayerOfGradedRowOrColumnMorphism:for IsToricVariety, IsGradedRowOrColumnMorphism, IsHomalgModuleElement}
}\hfill{\scriptsize (operation)}}\\
\textbf{\indent Returns:\ }
a vector space morphism 



 This method operates just as 'DegreeXLayerOfGradedRowOrColumnMorphism' above.
However, here the field F is taken as the field of coefficients of the Cox
ring of the variety $V$. Also, B is set to false, i.e. no additional information is being displayed. }

 }

 
\section{\textcolor{Chapter }{Truncations of morphisms of graded rows and columns in parallel}}\label{Chapter_Truncations_of_graded_rows_and_columns_Section_Truncations_of_morphisms_of_graded_rows_and_columns_in_parallel}
\logpage{[ 10, 4, 0 ]}
\hyperdef{L}{X82E4B7EB87C036A6}{}
{
  

\subsection{\textcolor{Chapter }{TruncateGradedRowOrColumnMorphismInParallel (for IsToricVariety, IsGradedRowOrColumnMorphism, IsList, IsPosInt, IsBool, IsFieldForHomalg)}}
\logpage{[ 10, 4, 1 ]}\nobreak
\hyperdef{L}{X815BB3017B73C6D4}{}
{\noindent\textcolor{FuncColor}{$\triangleright$\enspace\texttt{TruncateGradedRowOrColumnMorphismInParallel({\mdseries\slshape V, a, d, N, B, F})\index{TruncateGradedRowOrColumnMorphismInParallel@\texttt{Truncate}\-\texttt{Graded}\-\texttt{Row}\-\texttt{Or}\-\texttt{Column}\-\texttt{Morphism}\-\texttt{In}\-\texttt{Parallel}!for IsToricVariety, IsGradedRowOrColumnMorphism, IsList, IsPosInt, IsBool, IsFieldForHomalg}
\label{TruncateGradedRowOrColumnMorphismInParallel:for IsToricVariety, IsGradedRowOrColumnMorphism, IsList, IsPosInt, IsBool, IsFieldForHomalg}
}\hfill{\scriptsize (operation)}}\\
\textbf{\indent Returns:\ }
a vector space morphism 



 This method operates just as 'TruncateGradedRowOrColumnMorphism' above.
However, as fourth argument an integer $N$ is to be specified. The computation of the truncation will then be performed
in parallel in $N$ child processes. }

 

\subsection{\textcolor{Chapter }{TruncateGradedRowOrColumnMorphismInParallel (for IsToricVariety, IsGradedRowOrColumnMorphism, IsHomalgModuleElement, IsPosInt, IsBool, IsFieldForHomalg)}}
\logpage{[ 10, 4, 2 ]}\nobreak
\hyperdef{L}{X8237C3037A1FA5B1}{}
{\noindent\textcolor{FuncColor}{$\triangleright$\enspace\texttt{TruncateGradedRowOrColumnMorphismInParallel({\mdseries\slshape V, a, m, N, B, F})\index{TruncateGradedRowOrColumnMorphismInParallel@\texttt{Truncate}\-\texttt{Graded}\-\texttt{Row}\-\texttt{Or}\-\texttt{Column}\-\texttt{Morphism}\-\texttt{In}\-\texttt{Parallel}!for IsToricVariety, IsGradedRowOrColumnMorphism, IsHomalgModuleElement, IsPosInt, IsBool, IsFieldForHomalg}
\label{TruncateGradedRowOrColumnMorphismInParallel:for IsToricVariety, IsGradedRowOrColumnMorphism, IsHomalgModuleElement, IsPosInt, IsBool, IsFieldForHomalg}
}\hfill{\scriptsize (operation)}}\\
\textbf{\indent Returns:\ }
a vector space morphism 



 This method operates just as 'TruncateGradedRowOrColumnMorphism' above.
However, as fourth argument an integer $N$ is to be specified. The computation of the truncation will then be performed
in parallel in $N$ child processes. }

 

\subsection{\textcolor{Chapter }{TruncateGradedRowOrColumnMorphismInParallel (for IsToricVariety, IsGradedRowOrColumnMorphism, IsList, IsPosInt, IsBool)}}
\logpage{[ 10, 4, 3 ]}\nobreak
\hyperdef{L}{X81E3907A82ED9892}{}
{\noindent\textcolor{FuncColor}{$\triangleright$\enspace\texttt{TruncateGradedRowOrColumnMorphismInParallel({\mdseries\slshape V, a, d, N, B})\index{TruncateGradedRowOrColumnMorphismInParallel@\texttt{Truncate}\-\texttt{Graded}\-\texttt{Row}\-\texttt{Or}\-\texttt{Column}\-\texttt{Morphism}\-\texttt{In}\-\texttt{Parallel}!for IsToricVariety, IsGradedRowOrColumnMorphism, IsList, IsPosInt, IsBool}
\label{TruncateGradedRowOrColumnMorphismInParallel:for IsToricVariety, IsGradedRowOrColumnMorphism, IsList, IsPosInt, IsBool}
}\hfill{\scriptsize (operation)}}\\
\textbf{\indent Returns:\ }
a vector space morphism 



 This method operates just as 'TruncateGradedRowOrColumnMorphism' above.
However, as fourth argument an integer $N$ is to be specified. The computation of the truncation will then be performed
in parallel in $N$ child processes. }

 

\subsection{\textcolor{Chapter }{TruncateGradedRowOrColumnMorphismInParallel (for IsToricVariety, IsGradedRowOrColumnMorphism, IsHomalgModuleElement, IsPosInt, IsBool)}}
\logpage{[ 10, 4, 4 ]}\nobreak
\hyperdef{L}{X84E303037E38EC66}{}
{\noindent\textcolor{FuncColor}{$\triangleright$\enspace\texttt{TruncateGradedRowOrColumnMorphismInParallel({\mdseries\slshape V, a, m, N, B})\index{TruncateGradedRowOrColumnMorphismInParallel@\texttt{Truncate}\-\texttt{Graded}\-\texttt{Row}\-\texttt{Or}\-\texttt{Column}\-\texttt{Morphism}\-\texttt{In}\-\texttt{Parallel}!for IsToricVariety, IsGradedRowOrColumnMorphism, IsHomalgModuleElement, IsPosInt, IsBool}
\label{TruncateGradedRowOrColumnMorphismInParallel:for IsToricVariety, IsGradedRowOrColumnMorphism, IsHomalgModuleElement, IsPosInt, IsBool}
}\hfill{\scriptsize (operation)}}\\
\textbf{\indent Returns:\ }
a vector space morphism 



 This method operates just as 'TruncateGradedRowOrColumnMorphism' above.
However, as fourth argument an integer $N$ is to be specified. The computation of the truncation will then be performed
in parallel in $N$ child processes. }

 

\subsection{\textcolor{Chapter }{TruncateGradedRowOrColumnMorphismInParallel (for IsToricVariety, IsGradedRowOrColumnMorphism, IsList, IsPosInt)}}
\logpage{[ 10, 4, 5 ]}\nobreak
\hyperdef{L}{X78AF960D81AD0E66}{}
{\noindent\textcolor{FuncColor}{$\triangleright$\enspace\texttt{TruncateGradedRowOrColumnMorphismInParallel({\mdseries\slshape V, a, d, N})\index{TruncateGradedRowOrColumnMorphismInParallel@\texttt{Truncate}\-\texttt{Graded}\-\texttt{Row}\-\texttt{Or}\-\texttt{Column}\-\texttt{Morphism}\-\texttt{In}\-\texttt{Parallel}!for IsToricVariety, IsGradedRowOrColumnMorphism, IsList, IsPosInt}
\label{TruncateGradedRowOrColumnMorphismInParallel:for IsToricVariety, IsGradedRowOrColumnMorphism, IsList, IsPosInt}
}\hfill{\scriptsize (operation)}}\\
\textbf{\indent Returns:\ }
a vector space morphism 



 This method operates just as 'TruncateGradedRowOrColumnMorphism' above.
However, as fourth argument an integer $N$ is to be specified. The computation of the truncation will then be performed
in parallel in $N$ child processes. }

 

\subsection{\textcolor{Chapter }{TruncateGradedRowOrColumnMorphismInParallel (for IsToricVariety, IsGradedRowOrColumnMorphism, IsHomalgModuleElement, IsPosInt)}}
\logpage{[ 10, 4, 6 ]}\nobreak
\hyperdef{L}{X8705406479F88F35}{}
{\noindent\textcolor{FuncColor}{$\triangleright$\enspace\texttt{TruncateGradedRowOrColumnMorphismInParallel({\mdseries\slshape V, a, m, N})\index{TruncateGradedRowOrColumnMorphismInParallel@\texttt{Truncate}\-\texttt{Graded}\-\texttt{Row}\-\texttt{Or}\-\texttt{Column}\-\texttt{Morphism}\-\texttt{In}\-\texttt{Parallel}!for IsToricVariety, IsGradedRowOrColumnMorphism, IsHomalgModuleElement, IsPosInt}
\label{TruncateGradedRowOrColumnMorphismInParallel:for IsToricVariety, IsGradedRowOrColumnMorphism, IsHomalgModuleElement, IsPosInt}
}\hfill{\scriptsize (operation)}}\\
\textbf{\indent Returns:\ }
a vector space morphism 



 This method operates just as 'TruncateGradedRowOrColumnMorphism' above.
However, as fourth argument an integer $N$ is to be specified. The computation of the truncation will then be performed
in parallel in $N$ child processes. }

 }

 
\section{\textcolor{Chapter }{Examples}}\label{Chapter_Truncations_of_graded_rows_and_columns_Section_Examples}
\logpage{[ 10, 5, 0 ]}
\hyperdef{L}{X7A489A5D79DA9E5C}{}
{
  
\subsection{\textcolor{Chapter }{Truncations of graded rows and columns}}\label{Chapter_Truncations_of_graded_rows_and_columns_Section_Examples_Subsection_Truncations_of_graded_rows_and_columns}
\logpage{[ 10, 5, 1 ]}
\hyperdef{L}{X841A9F698680862E}{}
{
  
\begin{Verbatim}[commandchars=!@|,fontsize=\small,frame=single,label=Example]
  !gapprompt@gap>| !gapinput@P2 := ProjectiveSpace( 2 );|
  <A projective toric variety of dimension 2>
  !gapprompt@gap>| !gapinput@cox_ring := CoxRing( P2 );|
  Q[x_1,x_2,x_3]
  (weights: [ 1, 1, 1 ])
  !gapprompt@gap>| !gapinput@row := GradedRow( [[[2],1]], cox_ring );|
  <A graded row of rank 1>
  !gapprompt@gap>| !gapinput@trunc1 := DegreeXLayerOfGradedRowOrColumn( P2, row, [ -3 ] );|
  <A vector space embedded into (Q[x_1,x_2,x_3] (with weights [ 1, 1, 1 ]))^1>
  !gapprompt@gap>| !gapinput@Length( Generators( trunc1 ) );|
  0
  !gapprompt@gap>| !gapinput@trunc2 := DegreeXLayerOfGradedRowOrColumn( P2, row, [ -1 ] );|
  <A vector space embedded into (Q[x_1,x_2,x_3] (with weights [ 1, 1, 1 ]))^1>
  !gapprompt@gap>| !gapinput@Length( Generators( trunc2 ) );|
  3
\end{Verbatim}
 }

 
\subsection{\textcolor{Chapter }{Formats for generators of truncations of graded rows and columns}}\label{Chapter_Truncations_of_graded_rows_and_columns_Section_Examples_Subsection_Formats_for_generators_of_truncations_of_graded_rows_and_columns}
\logpage{[ 10, 5, 2 ]}
\hyperdef{L}{X7F5EB67D81E29D83}{}
{
  
\begin{Verbatim}[commandchars=!@|,fontsize=\small,frame=single,label=Example]
  !gapprompt@gap>| !gapinput@row2 := GradedRow( [[[2],2]], cox_ring );|
  <A graded row of rank 2>
  !gapprompt@gap>| !gapinput@gens1 := GeneratorsOfDegreeXLayerOfGradedRowOrColumnAsListOfColumnMatrices|
  !gapprompt@>| !gapinput@            (P2, row2, [ -1 ] );;|
  !gapprompt@gap>| !gapinput@Length( gens1 );|
  6
  !gapprompt@gap>| !gapinput@gens1[ 1 ];|
  <A 2 x 1 matrix over a graded ring>
  !gapprompt@gap>| !gapinput@Display( gens1[ 1 ] );|
  x_1,
  0
  (over a graded ring)
  !gapprompt@gap>| !gapinput@Display( gens1[ 4 ] );|
  0,
  x_1
  (over a graded ring)
  !gapprompt@gap>| !gapinput@gens2 := GeneratorsOfDegreeXLayerOfGradedRowOrColumnAsListsOfRecords|
  !gapprompt@>| !gapinput@            (P2, row2, [ -1 ] );|
  [ 6, [ rec( x_1 := 1, x_2 := 2, x_3 := 3 ),
         rec( x_1 := 4, x_2 := 5, x_3 := 6 ) ] ]
  !gapprompt@gap>| !gapinput@gens3 := GeneratorsOfDegreeXLayerOfGradedRowOrColumnAsUnionOfColumnMatrices|
  !gapprompt@>| !gapinput@            (P2, row2, [ -1 ] );|
  <A 2 x 6 mutable matrix over a graded ring>
  !gapprompt@gap>| !gapinput@Display( gens3 );|
  x_1,x_2,x_3,0,  0,  0, 
  0,  0,  0,  x_1,x_2,x_3
  (over a graded ring)
  !gapprompt@gap>| !gapinput@gens4 := GeneratorsOfDegreeXLayerOfGradedRowOrColumnAsListList|
  !gapprompt@>| !gapinput@            (P2, row2, [ -1 ] );|
  [ [ 1, x_1 ], [ 1, x_2 ], [ 1, x_3 ], [ 2, x_1 ], [ 2, x_2 ], [ 2, x_3 ] ]
\end{Verbatim}
 }

 
\subsection{\textcolor{Chapter }{Truncatons of morphisms of graded rows and columns}}\label{Chapter_Truncations_of_graded_rows_and_columns_Section_Examples_Subsection_Truncatons_of_morphisms_of_graded_rows_and_columns}
\logpage{[ 10, 5, 3 ]}
\hyperdef{L}{X8732695B8122C256}{}
{
  
\begin{Verbatim}[commandchars=!@|,fontsize=\small,frame=single,label=Example]
  !gapprompt@gap>| !gapinput@source := GradedRow( [[[-1],1]], cox_ring );|
  <A graded row of rank 1>
  !gapprompt@gap>| !gapinput@range := GradedRow( [[[0],1]], cox_ring );|
  <A graded row of rank 1>
  !gapprompt@gap>| !gapinput@trunc_generators := GeneratorsOfDegreeXLayerOfGradedRowOrColumnAsListsOfRecords|
  !gapprompt@>| !gapinput@                     (P2, range, [ 2 ] );|
  [ 6, [ rec( ("x_1*x_2") := 2, ("x_1*x_3") := 4, ("x_1^2") := 1,
              ("x_2*x_3") := 5, ("x_2^2") := 3, ("x_3^2") := 6 ) ] ]
  !gapprompt@gap>| !gapinput@vars := IndeterminatesOfPolynomialRing( cox_ring );;|
  !gapprompt@gap>| !gapinput@matrix := HomalgMatrix( [[ vars[ 1 ] ]], cox_ring );|
  <A 1 x 1 matrix over a graded ring>
  !gapprompt@gap>| !gapinput@mor := GradedRowOrColumnMorphism( source, matrix, range );|
  <A morphism in Category of graded rows over
  Q[x_1,x_2,x_3] (with weights [ 1, 1, 1 ])>
  !gapprompt@gap>| !gapinput@IsWellDefined( mor );|
  true
  !gapprompt@gap>| !gapinput@trunc_mor := TruncateGradedRowOrColumnMorphism( P2, mor, [ 2 ] );|
  <A morphism in Category of matrices over Q (with weights [ 1 ])>
  !gapprompt@gap>| !gapinput@Display( UnderlyingMatrix( trunc_mor ) );|
  1,0,0,0,0,0,
  0,1,0,0,0,0,
  0,0,0,1,0,0 
  (over a graded ring)
  !gapprompt@gap>| !gapinput@matrix2 := HomalgMatrix( [[ 1/2*vars[ 1 ] ]], cox_ring );|
  <A 1 x 1 matrix over a graded ring>
  !gapprompt@gap>| !gapinput@mor2 := GradedRowOrColumnMorphism( source, matrix2, range );|
  <A morphism in Category of graded rows over
  Q[x_1,x_2,x_3] (with weights [ 1, 1, 1 ])>
  !gapprompt@gap>| !gapinput@IsWellDefined( mor2 );|
  true
  !gapprompt@gap>| !gapinput@trunc_mor2 := TruncateGradedRowOrColumnMorphism( P2, mor2, [ 2 ] );|
  <A morphism in Category of matrices over Q (with weights [ 1 ])>
  !gapprompt@gap>| !gapinput@Display( UnderlyingMatrix( trunc_mor2 ) );|
  1/2,0,0,0,0,0,
  0,1/2,0,0,0,0,
  0,0,0,1/2,0,0 
  (over a graded ring)
\end{Verbatim}
 }

 
\subsection{\textcolor{Chapter }{Truncatons of morphisms of graded rows and columns in parallel}}\label{Chapter_Truncations_of_graded_rows_and_columns_Section_Examples_Subsection_Truncatons_of_morphisms_of_graded_rows_and_columns_in_parallel}
\logpage{[ 10, 5, 4 ]}
\hyperdef{L}{X7BB879B57B46C454}{}
{
  
\begin{Verbatim}[commandchars=!@|,fontsize=\small,frame=single,label=Example]
  !gapprompt@gap>| !gapinput@trunc_mor_parallel := TruncateGradedRowOrColumnMorphismInParallel|
  !gapprompt@>| !gapinput@                                              ( P2, mor, [ 2 ], 2 );|
  <A morphism in Category of matrices over Q (with weights [ 1 ])>
  !gapprompt@gap>| !gapinput@Display( UnderlyingMatrix( trunc_mor_parallel ) );|
  1,0,0,0,0,0,
  0,1,0,0,0,0,
  0,0,0,1,0,0
  (over a graded ring)
  !gapprompt@gap>| !gapinput@trunc_mor2_parallel := TruncateGradedRowOrColumnMorphismInParallel|
  !gapprompt@>| !gapinput@                                              ( P2, mor2, [ 2 ], 2 );|
  <A morphism in Category of matrices over Q (with weights [ 1 ])>
  !gapprompt@gap>| !gapinput@Display( UnderlyingMatrix( trunc_mor2_parallel ) );|
  1/2,0,0,0,0,0,
  0,1/2,0,0,0,0,
  0,0,0,1/2,0,0
  (over a graded ring)
  !gapprompt@gap>| !gapinput@trunc_mor2_parallel2 := TruncateGradedRowOrColumnMorphismInParallel|
  !gapprompt@>| !gapinput@                                              ( P2, mor2, [ 10 ], 3 );;|
  !gapprompt@gap>| !gapinput@IsWellDefined( trunc_mor2_parallel2 );|
  true
  !gapprompt@gap>| !gapinput@NrRows( UnderlyingMatrix( trunc_mor2_parallel2 ) );|
  55
  !gapprompt@gap>| !gapinput@NrColumns( UnderlyingMatrix( trunc_mor2_parallel2 ) );|
  66
\end{Verbatim}
 }

 }

 }

   
\chapter{\textcolor{Chapter }{Truncations of f.p. graded modules}}\label{Chapter_Truncations_of_fp_graded_modules}
\logpage{[ 11, 0, 0 ]}
\hyperdef{L}{X86906B587977B284}{}
{
  
\section{\textcolor{Chapter }{Truncations of fp graded modules}}\label{Chapter_Truncations_of_fp_graded_modules_Section_Truncations_of_fp_graded_modules}
\logpage{[ 11, 1, 0 ]}
\hyperdef{L}{X7C6111177B5624BF}{}
{
  

\subsection{\textcolor{Chapter }{TruncateFPGradedModule (for IsToricVariety, IsFpGradedLeftOrRightModulesObject, IsList, IsBool, IsFieldForHomalg)}}
\logpage{[ 11, 1, 1 ]}\nobreak
\hyperdef{L}{X7CC5EF407D90CFC2}{}
{\noindent\textcolor{FuncColor}{$\triangleright$\enspace\texttt{TruncateFPGradedModule({\mdseries\slshape V, M, d, B, F})\index{TruncateFPGradedModule@\texttt{TruncateFPGradedModule}!for IsToricVariety, IsFpGradedLeftOrRightModulesObject, IsList, IsBool, IsFieldForHomalg}
\label{TruncateFPGradedModule:for IsToricVariety, IsFpGradedLeftOrRightModulesObject, IsList, IsBool, IsFieldForHomalg}
}\hfill{\scriptsize (operation)}}\\
\textbf{\indent Returns:\ }
a FreydCategoryObject 



 The arguments are a toric variety $V$, an f.p. graded module $M$, a list $d$ (specifying a element of the class group of $V$) a boolean $B$ and a field $F$. We then compute the truncation of $M$ to the degree $d$ and return the corresponding vector space presentation as a
FreydCategoryObject. If $B$ is true, we display additional information during the computation. The latter
may be useful for longer computations. }

 
\begin{Verbatim}[commandchars=!@|,fontsize=\small,frame=single,label=Example]
  !gapprompt@gap>| !gapinput@P2 := ProjectiveSpace( 2 );|
  <A projective toric variety of dimension 2>
  !gapprompt@gap>| !gapinput@cox_ring := CoxRing( P2 );|
  Q[x_1,x_2,x_3]
  (weights: [ 1, 1, 1 ])
  !gapprompt@gap>| !gapinput@source := GradedRow( [[[-1],1]], cox_ring );|
  <A graded row of rank 1>
  !gapprompt@gap>| !gapinput@range := GradedRow( [[[0],1]], cox_ring );|
  <A graded row of rank 1>
  !gapprompt@gap>| !gapinput@vars := IndeterminatesOfPolynomialRing( cox_ring );;|
  !gapprompt@gap>| !gapinput@matrix := HomalgMatrix( [[ vars[ 1 ] ]], cox_ring );|
  <A 1 x 1 matrix over a graded ring>
  !gapprompt@gap>| !gapinput@obj1 := FreydCategoryObject(|
  !gapprompt@>| !gapinput@         GradedRowOrColumnMorphism( source, matrix, range ) );|
  <An object in Category of f.p. graded
  left modules over Q[x_1,x_2,x_3]
  (with weights [ 1, 1, 1 ])>
  !gapprompt@gap>| !gapinput@IsWellDefined( obj1 );|
  true
  !gapprompt@gap>| !gapinput@trunc_obj1 := TruncateFPGradedModule( P2, obj1, [ 2 ] );|
  <An object in Freyd( Category of matrices
  over Q (with weights [ 1 ]) )>
  !gapprompt@gap>| !gapinput@IsWellDefined( trunc_obj1 );|
  true
  !gapprompt@gap>| !gapinput@Display( UnderlyingMatrix( RelationMorphism( trunc_obj1 ) ) );|
  1,0,0,0,0,0,
  0,1,0,0,0,0,
  0,0,0,1,0,0
  (over a graded ring)
  !gapprompt@gap>| !gapinput@trunc_obj2 := TruncateFPGradedModuleInParallel( P2, obj1, [ 2 ], 2 );|
  <An object in Freyd( Category of matrices
  over Q (with weights [ 1 ]) )>
  !gapprompt@gap>| !gapinput@IsWellDefined( trunc_obj2 );|
  true
  !gapprompt@gap>| !gapinput@Display( UnderlyingMatrix( RelationMorphism( trunc_obj2 ) ) );|
  1,0,0,0,0,0,
  0,1,0,0,0,0,
  0,0,0,1,0,0
  (over a graded ring)
\end{Verbatim}
 }

 
\section{\textcolor{Chapter }{Truncations of fp graded modules in parallel}}\label{Chapter_Truncations_of_fp_graded_modules_Section_Truncations_of_fp_graded_modules_in_parallel}
\logpage{[ 11, 2, 0 ]}
\hyperdef{L}{X83A681827D70197E}{}
{
  

\subsection{\textcolor{Chapter }{TruncateFPGradedModuleInParallel (for IsToricVariety, IsFpGradedLeftOrRightModulesObject, IsList, IsPosInt, IsBool, IsFieldForHomalg)}}
\logpage{[ 11, 2, 1 ]}\nobreak
\hyperdef{L}{X7EAE93E37897ED43}{}
{\noindent\textcolor{FuncColor}{$\triangleright$\enspace\texttt{TruncateFPGradedModuleInParallel({\mdseries\slshape V, M, d, N, B., F})\index{TruncateFPGradedModuleInParallel@\texttt{TruncateFPGradedModuleInParallel}!for IsToricVariety, IsFpGradedLeftOrRightModulesObject, IsList, IsPosInt, IsBool, IsFieldForHomalg}
\label{TruncateFPGradedModuleInParallel:for IsToricVariety, IsFpGradedLeftOrRightModulesObject, IsList, IsPosInt, IsBool, IsFieldForHomalg}
}\hfill{\scriptsize (operation)}}\\
\textbf{\indent Returns:\ }
a FreydCategoryObject 



 The arguments are a toric variety $V$, an f.p. graded module $M$, a list $d$ (specifying a element of the class group of $V$), an integer $N$, a boolean $B$ and a field $F$. We then compute the truncation of $M$ to the degree $d$ and return the corresponding vector space presentation encoded as a
FreydCategoryObject. This is performed in $N$ child processes in parallel. If $B$ is true, we display additional information during the computation. The latter
may be useful for longer computations. }

 }

 
\section{\textcolor{Chapter }{Truncations of fp graded modules morphisms}}\label{Chapter_Truncations_of_fp_graded_modules_Section_Truncations_of_fp_graded_modules_morphisms}
\logpage{[ 11, 3, 0 ]}
\hyperdef{L}{X8602E944801C9369}{}
{
  

\subsection{\textcolor{Chapter }{TruncateFPGradedModuleMorphism (for IsToricVariety, IsFpGradedLeftOrRightModulesMorphism, IsList, IsBool, IsFieldForHomalg)}}
\logpage{[ 11, 3, 1 ]}\nobreak
\hyperdef{L}{X7D9865697CAC9B47}{}
{\noindent\textcolor{FuncColor}{$\triangleright$\enspace\texttt{TruncateFPGradedModuleMorphism({\mdseries\slshape V, M, d, B, F})\index{TruncateFPGradedModuleMorphism@\texttt{TruncateFPGradedModuleMorphism}!for IsToricVariety, IsFpGradedLeftOrRightModulesMorphism, IsList, IsBool, IsFieldForHomalg}
\label{TruncateFPGradedModuleMorphism:for IsToricVariety, IsFpGradedLeftOrRightModulesMorphism, IsList, IsBool, IsFieldForHomalg}
}\hfill{\scriptsize (operation)}}\\
\textbf{\indent Returns:\ }
a FreydCategoryMorphism 



 The arguments are a toric variety $V$, an f.p. graded module morphism $M$, a list $d$ (specifying a element of the class group of $V$), a boolean $B$ and a field F. We then compute the truncation of $M$ to the degree $d$ and return the corresponding morphism of vector space presentations encoded as
a FreydCategoryMorphism. If $B$ is true, we display additional information during the computation. The latter
may be useful for longer computations. }

 }

 
\section{\textcolor{Chapter }{Truncations of fp graded modules morphisms in parallel}}\label{Chapter_Truncations_of_fp_graded_modules_Section_Truncations_of_fp_graded_modules_morphisms_in_parallel}
\logpage{[ 11, 4, 0 ]}
\hyperdef{L}{X852297CA785CEBF3}{}
{
  

\subsection{\textcolor{Chapter }{TruncateFPGradedModuleMorphismInParallel (for IsToricVariety, IsFpGradedLeftOrRightModulesMorphism, IsList, IsList, IsBool, IsFieldForHomalg)}}
\logpage{[ 11, 4, 1 ]}\nobreak
\hyperdef{L}{X87EE0A93869F26E3}{}
{\noindent\textcolor{FuncColor}{$\triangleright$\enspace\texttt{TruncateFPGradedModuleMorphismInParallel({\mdseries\slshape V, M, d[, N1, N2, N3], B, F})\index{TruncateFPGradedModuleMorphismInParallel@\texttt{Truncate}\-\texttt{F}\-\texttt{P}\-\texttt{Graded}\-\texttt{Module}\-\texttt{Morphism}\-\texttt{In}\-\texttt{Parallel}!for IsToricVariety, IsFpGradedLeftOrRightModulesMorphism, IsList, IsList, IsBool, IsFieldForHomalg}
\label{TruncateFPGradedModuleMorphismInParallel:for IsToricVariety, IsFpGradedLeftOrRightModulesMorphism, IsList, IsList, IsBool, IsFieldForHomalg}
}\hfill{\scriptsize (operation)}}\\
\textbf{\indent Returns:\ }
a FreydCategoryMorphism 



 The arguments are a toric variety $V$, an f.p. graded module morphism $M$, a list $d$ (specifying a element of the class group of $V$), a list of 3 non-negative integers [ $N_1$, $N_2$, $N_3$ ], a boolean $B$ and a field F. We then compute the truncation of $M$ to the degree $d$ and return the corresponding morphism of vector space presentations encoded as
a FreydCategoryMorphism. This is done in parallel: the truncation of the
source is done by $N_1$ child processes in parallel, the truncation of the morphism datum is done by $N_2$ child processes and the truncation of the range of $M$ by $N_3$ processes. If the boolean $B$ is set to true, we display additional information during the computation. The
latter may be useful for longer computations. }

 }

 
\section{\textcolor{Chapter }{Truncations of f.p. graded module morphisms}}\label{Chapter_Truncations_of_fp_graded_modules_Section_Truncations_of_fp_graded_module_morphisms}
\logpage{[ 11, 5, 0 ]}
\hyperdef{L}{X7F69C9CE7AA8D0BD}{}
{
  
\begin{Verbatim}[commandchars=!@|,fontsize=\small,frame=single,label=Example]
  !gapprompt@gap>| !gapinput@source := GradedRow( [[[-1],1]], cox_ring );|
  <A graded row of rank 1>
  !gapprompt@gap>| !gapinput@range := GradedRow( [[[1],2]], cox_ring );|
  <A graded row of rank 2>
  !gapprompt@gap>| !gapinput@matrix := HomalgMatrix( [[ vars[ 1 ] * vars[ 2 ],|
  !gapprompt@>| !gapinput@                           vars[ 1 ] * vars[ 3 ] ]], cox_ring );|
  <A 1 x 2 matrix over a graded ring>
  !gapprompt@gap>| !gapinput@obj2 := FreydCategoryObject(|
  !gapprompt@>| !gapinput@         GradedRowOrColumnMorphism( source, matrix, range ) );|
  <An object in Category of f.p. graded
  left modules over Q[x_1,x_2,x_3]
  (with weights [ 1, 1, 1 ])>
  !gapprompt@gap>| !gapinput@source := GradedRow( [[[0],1]], cox_ring );|
  <A graded row of rank 1>
  !gapprompt@gap>| !gapinput@range := GradedRow( [[[1],2]], cox_ring );|
  <A graded row of rank 2>
  !gapprompt@gap>| !gapinput@matrix := HomalgMatrix( [[ vars[ 2 ], vars[ 3 ] ]], cox_ring );|
  <A 1 x 2 matrix over a graded ring>
  !gapprompt@gap>| !gapinput@mor := GradedRowOrColumnMorphism( source, matrix, range );|
  <A morphism in Category of graded rows
  over Q[x_1,x_2,x_3] (with weights [ 1, 1, 1 ])>
  !gapprompt@gap>| !gapinput@pres_mor := FreydCategoryMorphism( obj1, mor, obj2 );|
  <A morphism in Category of f.p. graded
  left modules over Q[x_1,x_2,x_3]
  (with weights [ 1, 1, 1 ])>
  !gapprompt@gap>| !gapinput@IsWellDefined( pres_mor );|
  true
  !gapprompt@gap>| !gapinput@trunc_pres_mor1 := TruncateFPGradedModuleMorphism( P2, pres_mor, [ 2 ] );|
  <A morphism in Freyd( Category of
  matrices over Q (with weights [ 1 ]) )>
  !gapprompt@gap>| !gapinput@IsWellDefined( trunc_pres_mor1 );|
  true
  !gapprompt@gap>| !gapinput@trunc_pres_mor2 := TruncateFPGradedModuleMorphismInParallel|
  !gapprompt@>| !gapinput@                            ( P2, pres_mor, [ 2 ], [ 2, 2, 2 ] );|
  <A morphism in Freyd( Category of
  matrices over Q (with weights [ 1 ]))>
  !gapprompt@gap>| !gapinput@IsWellDefined( trunc_pres_mor2 );|
  true
\end{Verbatim}
 }

 }

   
\chapter{\textcolor{Chapter }{Truncation functors for f.p. graded modules}}\label{Chapter_Truncation_functors_for_fp_graded_modules}
\logpage{[ 12, 0, 0 ]}
\hyperdef{L}{X7ECAA0B7861D95F5}{}
{
  
\section{\textcolor{Chapter }{Truncation functor for graded rows and columns}}\label{Chapter_Truncation_functors_for_fp_graded_modules_Section_Truncation_functor_for_graded_rows_and_columns}
\logpage{[ 12, 1, 0 ]}
\hyperdef{L}{X79DBF71F87487D00}{}
{
  

\subsection{\textcolor{Chapter }{TruncationFunctorForGradedRows (for IsToricVariety, IsList)}}
\logpage{[ 12, 1, 1 ]}\nobreak
\hyperdef{L}{X819184807EEFF3FB}{}
{\noindent\textcolor{FuncColor}{$\triangleright$\enspace\texttt{TruncationFunctorForGradedRows({\mdseries\slshape V, d})\index{TruncationFunctorForGradedRows@\texttt{TruncationFunctorForGradedRows}!for IsToricVariety, IsList}
\label{TruncationFunctorForGradedRows:for IsToricVariety, IsList}
}\hfill{\scriptsize (operation)}}\\
\textbf{\indent Returns:\ }
a functor 



 The arguments are a toric variety $V$ and degree{\textunderscore}list $d$ specifying an element of the degree group of the toric variety $V$. The latter can either be a list of integers or a HomalgModuleElement. Based
on this input, this method returns the functor for the truncation of graded
rows over the Cox ring of $V$ to degree $d$. }

 

\subsection{\textcolor{Chapter }{TruncationFunctorForGradedColumns (for IsToricVariety, IsList)}}
\logpage{[ 12, 1, 2 ]}\nobreak
\hyperdef{L}{X8388D66E7C7E3289}{}
{\noindent\textcolor{FuncColor}{$\triangleright$\enspace\texttt{TruncationFunctorForGradedColumns({\mdseries\slshape V, d})\index{TruncationFunctorForGradedColumns@\texttt{TruncationFunctorForGradedColumns}!for IsToricVariety, IsList}
\label{TruncationFunctorForGradedColumns:for IsToricVariety, IsList}
}\hfill{\scriptsize (operation)}}\\
\textbf{\indent Returns:\ }
a functor 



 The arguments are a toric variety $V$ and degree{\textunderscore}list $d$ specifying an element of the degree group of the toric variety $V$. The latter can either be a list of integers or a HomalgModuleElement. Based
on this input, this method returns the functor for the truncation of graded
columns over the Cox ring of $V$ to degree $d$. }

 }

 
\section{\textcolor{Chapter }{Truncation functor for f.p. graded modules}}\label{Chapter_Truncation_functors_for_fp_graded_modules_Section_Truncation_functor_for_fp_graded_modules}
\logpage{[ 12, 2, 0 ]}
\hyperdef{L}{X8120EEB07A043B7F}{}
{
  

\subsection{\textcolor{Chapter }{TruncationFunctorForFpGradedLeftModules (for IsToricVariety, IsList)}}
\logpage{[ 12, 2, 1 ]}\nobreak
\hyperdef{L}{X8705C7548213FB25}{}
{\noindent\textcolor{FuncColor}{$\triangleright$\enspace\texttt{TruncationFunctorForFpGradedLeftModules({\mdseries\slshape V, d})\index{TruncationFunctorForFpGradedLeftModules@\texttt{Truncation}\-\texttt{Functor}\-\texttt{For}\-\texttt{Fp}\-\texttt{Graded}\-\texttt{Left}\-\texttt{Modules}!for IsToricVariety, IsList}
\label{TruncationFunctorForFpGradedLeftModules:for IsToricVariety, IsList}
}\hfill{\scriptsize (operation)}}\\
\textbf{\indent Returns:\ }
a functor 



 The arguments are a toric variety $V$ and degree list $d$, which specifies an element of the degree group of the toric variety $V$. $d$ can either be a list of integers or a HomalgModuleElement. Based on this
input, this method returns the functor for the truncation of f.p. graded right
modules to degree $d$. }

 

\subsection{\textcolor{Chapter }{TruncationFunctorForFpGradedRightModules (for IsToricVariety, IsList)}}
\logpage{[ 12, 2, 2 ]}\nobreak
\hyperdef{L}{X7A8E18B97DDF01B2}{}
{\noindent\textcolor{FuncColor}{$\triangleright$\enspace\texttt{TruncationFunctorForFpGradedRightModules({\mdseries\slshape V, d})\index{TruncationFunctorForFpGradedRightModules@\texttt{Truncation}\-\texttt{Functor}\-\texttt{For}\-\texttt{Fp}\-\texttt{Graded}\-\texttt{Right}\-\texttt{Modules}!for IsToricVariety, IsList}
\label{TruncationFunctorForFpGradedRightModules:for IsToricVariety, IsList}
}\hfill{\scriptsize (operation)}}\\
\textbf{\indent Returns:\ }
a functor 



 The arguments are a toric variety $V$ and degree list $d$, which specifies an element of the degree group of the toric variety $V$. $d$ can either be a list of integers or a HomalgModuleElement. Based on this
input, this method returns the functor for the truncation of f.p. graded right
modules to degree $d$. }

 }

 
\section{\textcolor{Chapter }{Examples}}\label{Chapter_Truncation_functors_for_fp_graded_modules_Section_Examples}
\logpage{[ 12, 3, 0 ]}
\hyperdef{L}{X7A489A5D79DA9E5C}{}
{
  
\begin{Verbatim}[commandchars=!@|,fontsize=\small,frame=single,label=Example]
  !gapprompt@gap>| !gapinput@P2 := ProjectiveSpace( 2 );|
  <A projective toric variety of dimension 2>
  !gapprompt@gap>| !gapinput@P1 := ProjectiveSpace( 1 );|
  <A projective toric variety of dimension 1>
  !gapprompt@gap>| !gapinput@tor := P2 * P1;|
  <A projective toric variety of dimension 3
  which is a product of 2 toric varieties>
  !gapprompt@gap>| !gapinput@TruncationFunctorForGradedRows( tor, [ 2, 3 ] );|
  Trunction functor for Category of graded rows
  over Q[x_1,x_2,x_3,x_4,x_5] (with weights
  [ [ 0, 1 ], [ 1, 0 ], [ 1, 0 ],
  [ 0, 1 ], [ 0, 1 ] ] ) to the degree [ 2, 3 ]
  !gapprompt@gap>| !gapinput@TruncationFunctorForFpGradedLeftModules( tor, [ 4, 5 ] );|
  Truncation functor for Category of f.p.
  graded left modules over Q[x_1,x_2,x_3,x_4,x_5]
  (with weights [ [ 0, 1 ], [ 1, 0 ], [ 1, 0 ],
  [ 0, 1 ], [ 0, 1 ] ]) to the degree [ 4, 5 ]
\end{Verbatim}
 }

 }

   
\chapter{\textcolor{Chapter }{Truncations of GradedExt for f.p. graded modules}}\label{Chapter_Truncations_of_GradedExt_for_fp_graded_modules}
\logpage{[ 13, 0, 0 ]}
\hyperdef{L}{X8738F1CE80A26E3D}{}
{
  
\section{\textcolor{Chapter }{Truncations of InternalHoms of FpGradedModules}}\label{Chapter_Truncations_of_GradedExt_for_fp_graded_modules_Section_Truncations_of_InternalHoms_of_FpGradedModules}
\logpage{[ 13, 1, 0 ]}
\hyperdef{L}{X7BB04C7E7EA81C34}{}
{
  

 

\subsection{\textcolor{Chapter }{TruncateInternalHom (for IsToricVariety, IsFpGradedLeftOrRightModulesObject, IsFpGradedLeftOrRightModulesObject, IsList, IsBool, IsFieldForHomalg)}}
\logpage{[ 13, 1, 1 ]}\nobreak
\hyperdef{L}{X84F7CA76824A31AA}{}
{\noindent\textcolor{FuncColor}{$\triangleright$\enspace\texttt{TruncateInternalHom({\mdseries\slshape arg1, arg2, arg3, arg4, arg5, arg6})\index{TruncateInternalHom@\texttt{TruncateInternalHom}!for IsToricVariety, IsFpGradedLeftOrRightModulesObject, IsFpGradedLeftOrRightModulesObject, IsList, IsBool, IsFieldForHomalg}
\label{TruncateInternalHom:for IsToricVariety, IsFpGradedLeftOrRightModulesObject, IsFpGradedLeftOrRightModulesObject, IsList, IsBool, IsFieldForHomalg}
}\hfill{\scriptsize (operation)}}\\


 

 }

 

 

\subsection{\textcolor{Chapter }{TruncateInternalHomEmbedding (for IsToricVariety, IsFpGradedLeftOrRightModulesObject, IsFpGradedLeftOrRightModulesObject, IsList, IsBool, IsFieldForHomalg)}}
\logpage{[ 13, 1, 2 ]}\nobreak
\hyperdef{L}{X7E3B5D647DE54CD6}{}
{\noindent\textcolor{FuncColor}{$\triangleright$\enspace\texttt{TruncateInternalHomEmbedding({\mdseries\slshape arg1, arg2, arg3, arg4, arg5, arg6})\index{TruncateInternalHomEmbedding@\texttt{TruncateInternalHomEmbedding}!for IsToricVariety, IsFpGradedLeftOrRightModulesObject, IsFpGradedLeftOrRightModulesObject, IsList, IsBool, IsFieldForHomalg}
\label{TruncateInternalHomEmbedding:for IsToricVariety, IsFpGradedLeftOrRightModulesObject, IsFpGradedLeftOrRightModulesObject, IsList, IsBool, IsFieldForHomalg}
}\hfill{\scriptsize (operation)}}\\


 

 }

 

 

\subsection{\textcolor{Chapter }{TruncateInternalHom (for IsToricVariety, IsFpGradedLeftOrRightModulesMorphism, IsFpGradedLeftOrRightModulesMorphism, IsList, IsBool, IsFieldForHomalg)}}
\logpage{[ 13, 1, 3 ]}\nobreak
\hyperdef{L}{X7F42CA2283FB80FE}{}
{\noindent\textcolor{FuncColor}{$\triangleright$\enspace\texttt{TruncateInternalHom({\mdseries\slshape arg1, arg2, arg3, arg4, arg5, arg6})\index{TruncateInternalHom@\texttt{TruncateInternalHom}!for IsToricVariety, IsFpGradedLeftOrRightModulesMorphism, IsFpGradedLeftOrRightModulesMorphism, IsList, IsBool, IsFieldForHomalg}
\label{TruncateInternalHom:for IsToricVariety, IsFpGradedLeftOrRightModulesMorphism, IsFpGradedLeftOrRightModulesMorphism, IsList, IsBool, IsFieldForHomalg}
}\hfill{\scriptsize (operation)}}\\


 

 }

 }

 
\section{\textcolor{Chapter }{Truncations of InternalHoms of FpGradedModules to degree zero}}\label{Chapter_Truncations_of_GradedExt_for_fp_graded_modules_Section_Truncations_of_InternalHoms_of_FpGradedModules_to_degree_zero}
\logpage{[ 13, 2, 0 ]}
\hyperdef{L}{X831634C77CA0EAFE}{}
{
  

 

\subsection{\textcolor{Chapter }{TruncateInternalHomToZero (for IsToricVariety, IsFpGradedLeftOrRightModulesObject, IsFpGradedLeftOrRightModulesObject, IsBool, IsFieldForHomalg)}}
\logpage{[ 13, 2, 1 ]}\nobreak
\hyperdef{L}{X7D33B521840A65A7}{}
{\noindent\textcolor{FuncColor}{$\triangleright$\enspace\texttt{TruncateInternalHomToZero({\mdseries\slshape arg1, arg2, arg3, arg4, arg5})\index{TruncateInternalHomToZero@\texttt{TruncateInternalHomToZero}!for IsToricVariety, IsFpGradedLeftOrRightModulesObject, IsFpGradedLeftOrRightModulesObject, IsBool, IsFieldForHomalg}
\label{TruncateInternalHomToZero:for IsToricVariety, IsFpGradedLeftOrRightModulesObject, IsFpGradedLeftOrRightModulesObject, IsBool, IsFieldForHomalg}
}\hfill{\scriptsize (operation)}}\\


 

 }

 

 

\subsection{\textcolor{Chapter }{TruncateInternalHomEmbeddingToZero (for IsToricVariety, IsFpGradedLeftOrRightModulesObject, IsFpGradedLeftOrRightModulesObject, IsBool, IsFieldForHomalg)}}
\logpage{[ 13, 2, 2 ]}\nobreak
\hyperdef{L}{X7A6B33438518955D}{}
{\noindent\textcolor{FuncColor}{$\triangleright$\enspace\texttt{TruncateInternalHomEmbeddingToZero({\mdseries\slshape arg1, arg2, arg3, arg4, arg5})\index{TruncateInternalHomEmbeddingToZero@\texttt{TruncateInternalHomEmbeddingToZero}!for IsToricVariety, IsFpGradedLeftOrRightModulesObject, IsFpGradedLeftOrRightModulesObject, IsBool, IsFieldForHomalg}
\label{TruncateInternalHomEmbeddingToZero:for IsToricVariety, IsFpGradedLeftOrRightModulesObject, IsFpGradedLeftOrRightModulesObject, IsBool, IsFieldForHomalg}
}\hfill{\scriptsize (operation)}}\\


 

 }

 

 

\subsection{\textcolor{Chapter }{TruncateInternalHomToZero (for IsToricVariety, IsFpGradedLeftOrRightModulesMorphism, IsFpGradedLeftOrRightModulesMorphism, IsBool, IsFieldForHomalg)}}
\logpage{[ 13, 2, 3 ]}\nobreak
\hyperdef{L}{X7CB0A87783CB2CDB}{}
{\noindent\textcolor{FuncColor}{$\triangleright$\enspace\texttt{TruncateInternalHomToZero({\mdseries\slshape arg1, arg2, arg3, arg4, arg5})\index{TruncateInternalHomToZero@\texttt{TruncateInternalHomToZero}!for IsToricVariety, IsFpGradedLeftOrRightModulesMorphism, IsFpGradedLeftOrRightModulesMorphism, IsBool, IsFieldForHomalg}
\label{TruncateInternalHomToZero:for IsToricVariety, IsFpGradedLeftOrRightModulesMorphism, IsFpGradedLeftOrRightModulesMorphism, IsBool, IsFieldForHomalg}
}\hfill{\scriptsize (operation)}}\\


 

 }

 }

 
\section{\textcolor{Chapter }{Truncations of InternalHoms of FpGradedModules in parallel}}\label{Chapter_Truncations_of_GradedExt_for_fp_graded_modules_Section_Truncations_of_InternalHoms_of_FpGradedModules_in_parallel}
\logpage{[ 13, 3, 0 ]}
\hyperdef{L}{X803A24997DB44DB8}{}
{
  

 

\subsection{\textcolor{Chapter }{TruncateInternalHomInParallel (for IsToricVariety, IsFpGradedLeftOrRightModulesObject, IsFpGradedLeftOrRightModulesObject, IsList, IsBool, IsFieldForHomalg)}}
\logpage{[ 13, 3, 1 ]}\nobreak
\hyperdef{L}{X7E1745C381B13084}{}
{\noindent\textcolor{FuncColor}{$\triangleright$\enspace\texttt{TruncateInternalHomInParallel({\mdseries\slshape arg1, arg2, arg3, arg4, arg5, arg6})\index{TruncateInternalHomInParallel@\texttt{TruncateInternalHomInParallel}!for IsToricVariety, IsFpGradedLeftOrRightModulesObject, IsFpGradedLeftOrRightModulesObject, IsList, IsBool, IsFieldForHomalg}
\label{TruncateInternalHomInParallel:for IsToricVariety, IsFpGradedLeftOrRightModulesObject, IsFpGradedLeftOrRightModulesObject, IsList, IsBool, IsFieldForHomalg}
}\hfill{\scriptsize (operation)}}\\


 

 }

 

 

\subsection{\textcolor{Chapter }{TruncateInternalHomEmbeddingInParallel (for IsToricVariety, IsFpGradedLeftOrRightModulesObject, IsFpGradedLeftOrRightModulesObject, IsList, IsBool, IsFieldForHomalg)}}
\logpage{[ 13, 3, 2 ]}\nobreak
\hyperdef{L}{X7A10D1D37A306365}{}
{\noindent\textcolor{FuncColor}{$\triangleright$\enspace\texttt{TruncateInternalHomEmbeddingInParallel({\mdseries\slshape arg1, arg2, arg3, arg4, arg5, arg6})\index{TruncateInternalHomEmbeddingInParallel@\texttt{Truncate}\-\texttt{Internal}\-\texttt{Hom}\-\texttt{Embedding}\-\texttt{In}\-\texttt{Parallel}!for IsToricVariety, IsFpGradedLeftOrRightModulesObject, IsFpGradedLeftOrRightModulesObject, IsList, IsBool, IsFieldForHomalg}
\label{TruncateInternalHomEmbeddingInParallel:for IsToricVariety, IsFpGradedLeftOrRightModulesObject, IsFpGradedLeftOrRightModulesObject, IsList, IsBool, IsFieldForHomalg}
}\hfill{\scriptsize (operation)}}\\


 

 }

 

 

\subsection{\textcolor{Chapter }{TruncateInternalHomInParallel (for IsToricVariety, IsFpGradedLeftOrRightModulesMorphism, IsFpGradedLeftOrRightModulesMorphism, IsList, IsBool, IsFieldForHomalg)}}
\logpage{[ 13, 3, 3 ]}\nobreak
\hyperdef{L}{X810BAD537C83486A}{}
{\noindent\textcolor{FuncColor}{$\triangleright$\enspace\texttt{TruncateInternalHomInParallel({\mdseries\slshape arg1, arg2, arg3, arg4, arg5, arg6})\index{TruncateInternalHomInParallel@\texttt{TruncateInternalHomInParallel}!for IsToricVariety, IsFpGradedLeftOrRightModulesMorphism, IsFpGradedLeftOrRightModulesMorphism, IsList, IsBool, IsFieldForHomalg}
\label{TruncateInternalHomInParallel:for IsToricVariety, IsFpGradedLeftOrRightModulesMorphism, IsFpGradedLeftOrRightModulesMorphism, IsList, IsBool, IsFieldForHomalg}
}\hfill{\scriptsize (operation)}}\\


 

 }

 }

 
\section{\textcolor{Chapter }{Truncations of InternalHoms of FpGradedModules to degree zero in parallel}}\label{Chapter_Truncations_of_GradedExt_for_fp_graded_modules_Section_Truncations_of_InternalHoms_of_FpGradedModules_to_degree_zero_in_parallel}
\logpage{[ 13, 4, 0 ]}
\hyperdef{L}{X812CC3F97B627AF4}{}
{
  

 

\subsection{\textcolor{Chapter }{TruncateInternalHomToZeroInParallel (for IsToricVariety, IsFpGradedLeftOrRightModulesObject, IsFpGradedLeftOrRightModulesObject, IsBool, IsFieldForHomalg)}}
\logpage{[ 13, 4, 1 ]}\nobreak
\hyperdef{L}{X867EB9577A89A2B7}{}
{\noindent\textcolor{FuncColor}{$\triangleright$\enspace\texttt{TruncateInternalHomToZeroInParallel({\mdseries\slshape arg1, arg2, arg3, arg4, arg5})\index{TruncateInternalHomToZeroInParallel@\texttt{TruncateInternalHomToZeroInParallel}!for IsToricVariety, IsFpGradedLeftOrRightModulesObject, IsFpGradedLeftOrRightModulesObject, IsBool, IsFieldForHomalg}
\label{TruncateInternalHomToZeroInParallel:for IsToricVariety, IsFpGradedLeftOrRightModulesObject, IsFpGradedLeftOrRightModulesObject, IsBool, IsFieldForHomalg}
}\hfill{\scriptsize (operation)}}\\


 

 }

 

 

\subsection{\textcolor{Chapter }{TruncateInternalHomEmbeddingToZeroInParallel (for IsToricVariety, IsFpGradedLeftOrRightModulesObject, IsFpGradedLeftOrRightModulesObject, IsBool, IsFieldForHomalg)}}
\logpage{[ 13, 4, 2 ]}\nobreak
\hyperdef{L}{X7C1A1F4987B1E5DC}{}
{\noindent\textcolor{FuncColor}{$\triangleright$\enspace\texttt{TruncateInternalHomEmbeddingToZeroInParallel({\mdseries\slshape arg1, arg2, arg3, arg4, arg5})\index{TruncateInternalHomEmbeddingToZeroInParallel@\texttt{Truncate}\-\texttt{Internal}\-\texttt{Hom}\-\texttt{Embedding}\-\texttt{To}\-\texttt{Zero}\-\texttt{In}\-\texttt{Parallel}!for IsToricVariety, IsFpGradedLeftOrRightModulesObject, IsFpGradedLeftOrRightModulesObject, IsBool, IsFieldForHomalg}
\label{TruncateInternalHomEmbeddingToZeroInParallel:for IsToricVariety, IsFpGradedLeftOrRightModulesObject, IsFpGradedLeftOrRightModulesObject, IsBool, IsFieldForHomalg}
}\hfill{\scriptsize (operation)}}\\


 

 }

 

 

\subsection{\textcolor{Chapter }{TruncateInternalHomToZeroInParallel (for IsToricVariety, IsFpGradedLeftOrRightModulesMorphism, IsFpGradedLeftOrRightModulesMorphism, IsBool, IsFieldForHomalg)}}
\logpage{[ 13, 4, 3 ]}\nobreak
\hyperdef{L}{X7DB929AE8422AE49}{}
{\noindent\textcolor{FuncColor}{$\triangleright$\enspace\texttt{TruncateInternalHomToZeroInParallel({\mdseries\slshape arg1, arg2, arg3, arg4, arg5})\index{TruncateInternalHomToZeroInParallel@\texttt{TruncateInternalHomToZeroInParallel}!for IsToricVariety, IsFpGradedLeftOrRightModulesMorphism, IsFpGradedLeftOrRightModulesMorphism, IsBool, IsFieldForHomalg}
\label{TruncateInternalHomToZeroInParallel:for IsToricVariety, IsFpGradedLeftOrRightModulesMorphism, IsFpGradedLeftOrRightModulesMorphism, IsBool, IsFieldForHomalg}
}\hfill{\scriptsize (operation)}}\\


 

 }

 

 

\subsection{\textcolor{Chapter }{TruncateGradedExt (for IsInt, IsToricVariety, IsFpGradedLeftOrRightModulesObject, IsFpGradedLeftOrRightModulesObject, IsList, IsList)}}
\logpage{[ 13, 4, 4 ]}\nobreak
\hyperdef{L}{X7E3AA3B885FDD3EE}{}
{\noindent\textcolor{FuncColor}{$\triangleright$\enspace\texttt{TruncateGradedExt({\mdseries\slshape arg1, arg2, arg3, arg4, arg5, arg6})\index{TruncateGradedExt@\texttt{TruncateGradedExt}!for IsInt, IsToricVariety, IsFpGradedLeftOrRightModulesObject, IsFpGradedLeftOrRightModulesObject, IsList, IsList}
\label{TruncateGradedExt:for IsInt, IsToricVariety, IsFpGradedLeftOrRightModulesObject, IsFpGradedLeftOrRightModulesObject, IsList, IsList}
}\hfill{\scriptsize (operation)}}\\


 

 }

 

 

\subsection{\textcolor{Chapter }{TruncateGradedExtToZero (for IsInt, IsToricVariety, IsFpGradedLeftOrRightModulesObject, IsFpGradedLeftOrRightModulesObject, IsBool, IsFieldForHomalg)}}
\logpage{[ 13, 4, 5 ]}\nobreak
\hyperdef{L}{X7EB2C68E856747B6}{}
{\noindent\textcolor{FuncColor}{$\triangleright$\enspace\texttt{TruncateGradedExtToZero({\mdseries\slshape arg1, arg2, arg3, arg4, arg5, arg6})\index{TruncateGradedExtToZero@\texttt{TruncateGradedExtToZero}!for IsInt, IsToricVariety, IsFpGradedLeftOrRightModulesObject, IsFpGradedLeftOrRightModulesObject, IsBool, IsFieldForHomalg}
\label{TruncateGradedExtToZero:for IsInt, IsToricVariety, IsFpGradedLeftOrRightModulesObject, IsFpGradedLeftOrRightModulesObject, IsBool, IsFieldForHomalg}
}\hfill{\scriptsize (operation)}}\\


 

 }

 

 

\subsection{\textcolor{Chapter }{TruncateGradedExtInParallel (for IsInt, IsToricVariety, IsFpGradedLeftOrRightModulesObject, IsFpGradedLeftOrRightModulesObject, IsList, IsList)}}
\logpage{[ 13, 4, 6 ]}\nobreak
\hyperdef{L}{X7E6BF7497831C7E0}{}
{\noindent\textcolor{FuncColor}{$\triangleright$\enspace\texttt{TruncateGradedExtInParallel({\mdseries\slshape arg1, arg2, arg3, arg4, arg5, arg6})\index{TruncateGradedExtInParallel@\texttt{TruncateGradedExtInParallel}!for IsInt, IsToricVariety, IsFpGradedLeftOrRightModulesObject, IsFpGradedLeftOrRightModulesObject, IsList, IsList}
\label{TruncateGradedExtInParallel:for IsInt, IsToricVariety, IsFpGradedLeftOrRightModulesObject, IsFpGradedLeftOrRightModulesObject, IsList, IsList}
}\hfill{\scriptsize (operation)}}\\


 

 }

 

 

\subsection{\textcolor{Chapter }{TruncateGradedExtToZeroInParallel (for IsInt, IsToricVariety, IsFpGradedLeftOrRightModulesObject, IsFpGradedLeftOrRightModulesObject, IsBool, IsFieldForHomalg)}}
\logpage{[ 13, 4, 7 ]}\nobreak
\hyperdef{L}{X787B0B6F7E89060E}{}
{\noindent\textcolor{FuncColor}{$\triangleright$\enspace\texttt{TruncateGradedExtToZeroInParallel({\mdseries\slshape arg1, arg2, arg3, arg4, arg5, arg6})\index{TruncateGradedExtToZeroInParallel@\texttt{TruncateGradedExtToZeroInParallel}!for IsInt, IsToricVariety, IsFpGradedLeftOrRightModulesObject, IsFpGradedLeftOrRightModulesObject, IsBool, IsFieldForHomalg}
\label{TruncateGradedExtToZeroInParallel:for IsInt, IsToricVariety, IsFpGradedLeftOrRightModulesObject, IsFpGradedLeftOrRightModulesObject, IsBool, IsFieldForHomalg}
}\hfill{\scriptsize (operation)}}\\


 

 }

 }

 
\section{\textcolor{Chapter }{Examples}}\label{Chapter_Truncations_of_GradedExt_for_fp_graded_modules_Section_Examples}
\logpage{[ 13, 5, 0 ]}
\hyperdef{L}{X7A489A5D79DA9E5C}{}
{
  
\subsection{\textcolor{Chapter }{Truncation of IntHom}}\label{Chapter_Truncations_of_GradedExt_for_fp_graded_modules_Section_Examples_Subsection_Truncation_of_IntHom}
\logpage{[ 13, 5, 1 ]}
\hyperdef{L}{X81E9CAE07AB87567}{}
{
  
\begin{Verbatim}[commandchars=!@|,fontsize=\small,frame=single,label=Example]
  !gapprompt@gap>| !gapinput@P2 := ProjectiveSpace( 2 );|
  <A projective toric variety of dimension 2>
  !gapprompt@gap>| !gapinput@cox_ring := CoxRing( P2 );|
  Q[x_1,x_2,x_3]
  (weights: [ 1, 1, 1 ])
  !gapprompt@gap>| !gapinput@source := GradedRow( [[[-1],1]], cox_ring );|
  <A graded row of rank 1>
  !gapprompt@gap>| !gapinput@range := GradedRow( [[[0],1]], cox_ring );|
  <A graded row of rank 1>
  !gapprompt@gap>| !gapinput@vars := IndeterminatesOfPolynomialRing( cox_ring );;|
  !gapprompt@gap>| !gapinput@matrix := HomalgMatrix( [[ vars[ 1 ] ]], cox_ring );|
  <A 1 x 1 matrix over a graded ring>
  !gapprompt@gap>| !gapinput@obj1 := FreydCategoryObject(|
  !gapprompt@>| !gapinput@         GradedRowOrColumnMorphism( source, matrix, range ) );|
  <An object in Category of f.p. graded
  left modules over Q[x_1,x_2,x_3]
  (with weights [ 1, 1, 1 ])>
  !gapprompt@gap>| !gapinput@IsWellDefined( obj1 );|
  true
  !gapprompt@gap>| !gapinput@source := GradedRow( [[[-1],1]], cox_ring );|
  <A graded row of rank 1>
  !gapprompt@gap>| !gapinput@range := GradedRow( [[[1],2]], cox_ring );|
  <A graded row of rank 2>
  !gapprompt@gap>| !gapinput@matrix := HomalgMatrix( [[ vars[ 1 ] * vars[ 2 ],|
  !gapprompt@>| !gapinput@                           vars[ 1 ] * vars[ 3 ] ]], cox_ring );|
  <A 1 x 2 matrix over a graded ring>
  !gapprompt@gap>| !gapinput@obj2 := FreydCategoryObject(|
  !gapprompt@>| !gapinput@         GradedRowOrColumnMorphism( source, matrix, range ) );|
  <An object in Category of f.p. graded
  left modules over Q[x_1,x_2,x_3]
  (with weights [ 1, 1, 1 ])>
  !gapprompt@gap>| !gapinput@IsWellDefined( obj2 );|
  true
  !gapprompt@gap>| !gapinput@source := GradedRow( [[[0],1]], cox_ring );|
  <A graded row of rank 1>
  !gapprompt@gap>| !gapinput@range := GradedRow( [[[1],2]], cox_ring );|
  <A graded row of rank 2>
  !gapprompt@gap>| !gapinput@matrix := HomalgMatrix( [[ vars[ 2 ], vars[ 3 ] ]], cox_ring );|
  <A 1 x 2 matrix over a graded ring>
  !gapprompt@gap>| !gapinput@mor := GradedRowOrColumnMorphism( source, matrix, range );|
  <A morphism in Category of graded rows
  over Q[x_1,x_2,x_3] (with weights [ 1, 1, 1 ])>
  !gapprompt@gap>| !gapinput@pres_mor := FreydCategoryMorphism( obj1, mor, obj2 );|
  <A morphism in Category of f.p. graded
  left modules over Q[x_1,x_2,x_3]
  (with weights [ 1, 1, 1 ])>
  !gapprompt@gap>| !gapinput@IsWellDefined( pres_mor );|
  true
  !gapprompt@gap>| !gapinput@Q := HomalgFieldOfRationalsInSingular();|
  Q
  !gapprompt@gap>| !gapinput@m1 := TruncateInternalHom( P2, obj1, obj2, [ 4 ], false, Q );|
  <An object in Freyd( Category of matrices over Q )>
  !gapprompt@gap>| !gapinput@IsWellDefined( m1 );|
  true
  !gapprompt@gap>| !gapinput@m2 := TruncateInternalHomEmbedding( P2, obj1, obj2, [ 4 ], false, Q );|
  <A monomorphism in Freyd( Category of matrices over Q )>
  !gapprompt@gap>| !gapinput@IsWellDefined( m2 );|
  true
  !gapprompt@gap>| !gapinput@m3 := TruncateInternalHom( P2, pres_mor, IdentityMorphism( obj2 ), [ 4 ], false, Q );|
  <A morphism in Freyd( Category of matrices over Q )>
  !gapprompt@gap>| !gapinput@IsWellDefined( m3 );|
  true
\end{Verbatim}
 }

 
\subsection{\textcolor{Chapter }{Truncation of IntHom to degree zero}}\label{Chapter_Truncations_of_GradedExt_for_fp_graded_modules_Section_Examples_Subsection_Truncation_of_IntHom_to_degree_zero}
\logpage{[ 13, 5, 2 ]}
\hyperdef{L}{X80891C1087761B58}{}
{
  
\begin{Verbatim}[commandchars=!@|,fontsize=\small,frame=single,label=Example]
  !gapprompt@gap>| !gapinput@m4 := TruncateInternalHomToZero( P2, obj1, obj2, false, Q );|
  <An object in Freyd( Category of matrices over Q )>
  !gapprompt@gap>| !gapinput@IsWellDefined( m4 );|
  true
  !gapprompt@gap>| !gapinput@m5 := TruncateInternalHomEmbeddingToZero( P2, obj1, obj2, false, Q );|
  <A monomorphism in Freyd( Category of matrices over Q )>
  !gapprompt@gap>| !gapinput@IsWellDefined( m5 );|
  true
  !gapprompt@gap>| !gapinput@m6 := TruncateInternalHomToZero( P2, pres_mor, IdentityMorphism( obj2 ), false, Q );|
  <A morphism in Freyd( Category of matrices over Q )>
  !gapprompt@gap>| !gapinput@IsWellDefined( m6 );|
  true
\end{Verbatim}
 }

 
\subsection{\textcolor{Chapter }{Truncation of IntHom in parallel}}\label{Chapter_Truncations_of_GradedExt_for_fp_graded_modules_Section_Examples_Subsection_Truncation_of_IntHom_in_parallel}
\logpage{[ 13, 5, 3 ]}
\hyperdef{L}{X78744A11809F3793}{}
{
  
\begin{Verbatim}[commandchars=!@|,fontsize=\small,frame=single,label=Example]
  !gapprompt@gap>| !gapinput@m7 := TruncateInternalHomInParallel( P2, obj1, obj2, [ 4 ], false, Q );|
  <An object in Freyd( Category of matrices over Q )>
  !gapprompt@gap>| !gapinput@m1 = m7;|
  true
  !gapprompt@gap>| !gapinput@m8 := TruncateInternalHomEmbeddingInParallel( P2, obj1, obj2, [ 4 ], false, Q );|
  <A monomorphism in Freyd( Category of matrices over Q )>
  !gapprompt@gap>| !gapinput@m8 = m2;|
  true
  !gapprompt@gap>| !gapinput@m9 := TruncateInternalHomInParallel( P2, pres_mor, IdentityMorphism( obj2 ), [ 4 ], false, Q );|
  <A morphism in Freyd( Category of matrices over Q )>
  !gapprompt@gap>| !gapinput@m9 = m3;|
  true
\end{Verbatim}
 }

 
\subsection{\textcolor{Chapter }{Truncation of IntHom to degree zero in parallel}}\label{Chapter_Truncations_of_GradedExt_for_fp_graded_modules_Section_Examples_Subsection_Truncation_of_IntHom_to_degree_zero_in_parallel}
\logpage{[ 13, 5, 4 ]}
\hyperdef{L}{X7877DA0B79C62FDD}{}
{
  
\begin{Verbatim}[commandchars=!@|,fontsize=\small,frame=single,label=Example]
  !gapprompt@gap>| !gapinput@m10 := TruncateInternalHomToZeroInParallel( P2, obj1, obj2, false, Q );|
  <An object in Freyd( Category of matrices over Q )>
  !gapprompt@gap>| !gapinput@m10 = m4;|
  true
  !gapprompt@gap>| !gapinput@m11 := TruncateInternalHomEmbeddingToZeroInParallel( P2, obj1, obj2, false, Q );|
  <A monomorphism in Freyd( Category of matrices over Q )>
  !gapprompt@gap>| !gapinput@m11 = m5;|
  true
  !gapprompt@gap>| !gapinput@m12 := TruncateInternalHomToZeroInParallel( P2, pres_mor, IdentityMorphism( obj2 ), false, Q );|
  <A morphism in Freyd( Category of matrices over Q )>
  !gapprompt@gap>| !gapinput@m12 = m6;|
  true
\end{Verbatim}
 }

 
\subsection{\textcolor{Chapter }{Truncation of GradedExt}}\label{Chapter_Truncations_of_GradedExt_for_fp_graded_modules_Section_Examples_Subsection_Truncation_of_GradedExt}
\logpage{[ 13, 5, 5 ]}
\hyperdef{L}{X7AC725F181EEE3FD}{}
{
  
\begin{Verbatim}[commandchars=!@|,fontsize=\small,frame=single,label=Example]
  !gapprompt@gap>| !gapinput@v1 := TruncateGradedExt( 1, P2, obj1, obj2, [ 4 ], [ false, Q ] );|
  <An object in Freyd( Category of matrices over Q )>
  !gapprompt@gap>| !gapinput@IsWellDefined( v1 );|
  true
  !gapprompt@gap>| !gapinput@v2 := TruncateGradedExt( 1, P2, obj1, obj2, [ 0 ], [ false, Q ] );|
  <An object in Freyd( Category of matrices over Q )>
  !gapprompt@gap>| !gapinput@IsWellDefined( v2 );|
  true
  !gapprompt@gap>| !gapinput@v3 := TruncateGradedExtToZero( 1, P2, obj1, obj2, false, Q );|
  <An object in Freyd( Category of matrices over Q )>
  !gapprompt@gap>| !gapinput@v3 = v2;|
  true
  !gapprompt@gap>| !gapinput@v4 := TruncateGradedExtInParallel( 1, P2, obj1, obj2, [ 4 ], [ false, Q ] );|
  <An object in Freyd( Category of matrices over Q )>
  !gapprompt@gap>| !gapinput@IsWellDefined( v4 );|
  true
  !gapprompt@gap>| !gapinput@v5 := TruncateGradedExtInParallel( 1, P2, obj1, obj2, [ 0 ], [ false, Q ] );|
  <An object in Freyd( Category of matrices over Q )>
  !gapprompt@gap>| !gapinput@IsWellDefined( v5 );|
  true
  !gapprompt@gap>| !gapinput@v6 := TruncateGradedExtToZeroInParallel( 1, P2, obj1, obj2, false, Q );|
  <An object in Freyd( Category of matrices over Q )>
  !gapprompt@gap>| !gapinput@v6 = v5;|
  true
\end{Verbatim}
 }

 }

 }

   
\chapter{\textcolor{Chapter }{Sheaf cohomology by use of https://arxiv.org/abs/1802.08860}}\label{Chapter_Sheaf_cohomology_by_use_of_httpsarxivorgabs180208860}
\logpage{[ 14, 0, 0 ]}
\hyperdef{L}{X83B465EC7B2050EA}{}
{
  
\section{\textcolor{Chapter }{Preliminaries}}\label{Chapter_Sheaf_cohomology_by_use_of_httpsarxivorgabs180208860_Section_Preliminaries}
\logpage{[ 14, 1, 0 ]}
\hyperdef{L}{X8749E1888244CC3D}{}
{
  

\subsection{\textcolor{Chapter }{ParameterCheck (for IsToricVariety, IsFpGradedLeftOrRightModulesObject, IsFpGradedLeftOrRightModulesObject, IsInt)}}
\logpage{[ 14, 1, 1 ]}\nobreak
\hyperdef{L}{X85E7494381B5C3A8}{}
{\noindent\textcolor{FuncColor}{$\triangleright$\enspace\texttt{ParameterCheck({\mdseries\slshape V, M1, M2, i})\index{ParameterCheck@\texttt{ParameterCheck}!for IsToricVariety, IsFpGradedLeftOrRightModulesObject, IsFpGradedLeftOrRightModulesObject, IsInt}
\label{ParameterCheck:for IsToricVariety, IsFpGradedLeftOrRightModulesObject, IsFpGradedLeftOrRightModulesObject, IsInt}
}\hfill{\scriptsize (operation)}}\\
\textbf{\indent Returns:\ }
true or false 



 Given a toric variety $V$, we eventually wish to compute the i-th sheaf cohomology of the
sheafification of the f.p. graded S-module $M_2$ (S being the Cox ring of vari). To this end we use modules $M_1$ which sheafify to the structure sheaf of vari. This method tests if the
truncation to degree zero of $Ext^i_S( M_1, M_2 )$ is isomorphic to $H^i( V, \widetilde{M_2} )$. }

 

\subsection{\textcolor{Chapter }{FindIdeal (for IsToricVariety, IsFpGradedLeftOrRightModulesObject, IsInt)}}
\logpage{[ 14, 1, 2 ]}\nobreak
\hyperdef{L}{X7C916ADD876A18A8}{}
{\noindent\textcolor{FuncColor}{$\triangleright$\enspace\texttt{FindIdeal({\mdseries\slshape V, M, i})\index{FindIdeal@\texttt{FindIdeal}!for IsToricVariety, IsFpGradedLeftOrRightModulesObject, IsInt}
\label{FindIdeal:for IsToricVariety, IsFpGradedLeftOrRightModulesObject, IsInt}
}\hfill{\scriptsize (operation)}}\\
\textbf{\indent Returns:\ }
a list 



 Given a toric variety $V$ and an f.p. graded S-module $M$ (S being the Cox ring of vari), we wish to compute the i-th sheaf cohomology
of $\tilde{M}$. To this end, this method identifies an ideal $I$ of $S$ such that $\tilde{I}$ is the structure sheaf of $V$ and such that $Ext^i_S( I, M )$ is isomorphic to $H^i( V, \tilde{M} )$. We identify $I$ by determining an ample degree $d \in \mathrm{Cl} ( V )$. Then, for a suitable non-negative integer $e$, the generators of $I$ are the $e$-th power of all monomials of degree $d$ in the Cox ring of $S$. We return the list [ e, d, I ]. }

 }

 
\section{\textcolor{Chapter }{Computation of global sections}}\label{Chapter_Sheaf_cohomology_by_use_of_httpsarxivorgabs180208860_Section_Computation_of_global_sections}
\logpage{[ 14, 2, 0 ]}
\hyperdef{L}{X7F66EA1B863A9AD9}{}
{
  

\subsection{\textcolor{Chapter }{H0 (for IsToricVariety, IsFpGradedLeftOrRightModulesObject)}}
\logpage{[ 14, 2, 1 ]}\nobreak
\hyperdef{L}{X806638F97CECBC02}{}
{\noindent\textcolor{FuncColor}{$\triangleright$\enspace\texttt{H0({\mdseries\slshape V, M})\index{H0@\texttt{H0}!for IsToricVariety, IsFpGradedLeftOrRightModulesObject}
\label{H0:for IsToricVariety, IsFpGradedLeftOrRightModulesObject}
}\hfill{\scriptsize (operation)}}\\
\textbf{\indent Returns:\ }
a vector space 



 Given a variety $V$ and an f.p. graded $S$-module $M$ ($S$ being the Cox ring of $V$), this method computes $H^0( V, \tilde{M} )$. }

 

\subsection{\textcolor{Chapter }{H0Parallel (for IsToricVariety, IsFpGradedLeftOrRightModulesObject)}}
\logpage{[ 14, 2, 2 ]}\nobreak
\hyperdef{L}{X82239D1584CC5169}{}
{\noindent\textcolor{FuncColor}{$\triangleright$\enspace\texttt{H0Parallel({\mdseries\slshape V, M})\index{H0Parallel@\texttt{H0Parallel}!for IsToricVariety, IsFpGradedLeftOrRightModulesObject}
\label{H0Parallel:for IsToricVariety, IsFpGradedLeftOrRightModulesObject}
}\hfill{\scriptsize (operation)}}\\
\textbf{\indent Returns:\ }
a vector space 



 Given a variety $V$ and an f.p. graded $S$-module $M$ ($S$ being the Cox ring of $V$), this method computes $H^0( V, \tilde{M} )$. This method is parallelized and is thus best suited for long and complicated
computations. }

 }

 
\section{\textcolor{Chapter }{Computation of the i-th sheaf cohomologies}}\label{Chapter_Sheaf_cohomology_by_use_of_httpsarxivorgabs180208860_Section_Computation_of_the_i-th_sheaf_cohomologies}
\logpage{[ 14, 3, 0 ]}
\hyperdef{L}{X7E0A3B137BCC127F}{}
{
  

\subsection{\textcolor{Chapter }{Hi (for IsToricVariety, IsFpGradedLeftOrRightModulesObject, IsInt)}}
\logpage{[ 14, 3, 1 ]}\nobreak
\hyperdef{L}{X7B9CBB888756B239}{}
{\noindent\textcolor{FuncColor}{$\triangleright$\enspace\texttt{Hi({\mdseries\slshape V, M, i})\index{Hi@\texttt{Hi}!for IsToricVariety, IsFpGradedLeftOrRightModulesObject, IsInt}
\label{Hi:for IsToricVariety, IsFpGradedLeftOrRightModulesObject, IsInt}
}\hfill{\scriptsize (operation)}}\\
\textbf{\indent Returns:\ }
a vector space 



 Given a variety $V$ and an f.p. graded $S$-module $M$ ($S$ being the Cox ring of $V$), this method computes $H^i( V, \tilde{M} )$. }

 

\subsection{\textcolor{Chapter }{HiParallel (for IsToricVariety, IsFpGradedLeftOrRightModulesObject, IsInt)}}
\logpage{[ 14, 3, 2 ]}\nobreak
\hyperdef{L}{X80DB4FBD869C8030}{}
{\noindent\textcolor{FuncColor}{$\triangleright$\enspace\texttt{HiParallel({\mdseries\slshape V, M, i})\index{HiParallel@\texttt{HiParallel}!for IsToricVariety, IsFpGradedLeftOrRightModulesObject, IsInt}
\label{HiParallel:for IsToricVariety, IsFpGradedLeftOrRightModulesObject, IsInt}
}\hfill{\scriptsize (operation)}}\\
\textbf{\indent Returns:\ }
a vector space 



 Given a variety $V$ and an f.p. graded $S$-module $M$ ($S$ being the Cox ring of $V$), this method computes $H^i( V, \tilde{M} )$. This method is parallelized and is thus best suited for long and complicated
computations. }

 }

 
\section{\textcolor{Chapter }{Computation of all sheaf cohomologies}}\label{Chapter_Sheaf_cohomology_by_use_of_httpsarxivorgabs180208860_Section_Computation_of_all_sheaf_cohomologies}
\logpage{[ 14, 4, 0 ]}
\hyperdef{L}{X7ED9C2657DDABA20}{}
{
  

\subsection{\textcolor{Chapter }{AllHi (for IsToricVariety, IsFpGradedLeftOrRightModulesObject)}}
\logpage{[ 14, 4, 1 ]}\nobreak
\hyperdef{L}{X8252E52187CBF1C7}{}
{\noindent\textcolor{FuncColor}{$\triangleright$\enspace\texttt{AllHi({\mdseries\slshape V, M})\index{AllHi@\texttt{AllHi}!for IsToricVariety, IsFpGradedLeftOrRightModulesObject}
\label{AllHi:for IsToricVariety, IsFpGradedLeftOrRightModulesObject}
}\hfill{\scriptsize (operation)}}\\
\textbf{\indent Returns:\ }
a list of vector spaces 



 Given a variety $V$ and an f.p. graded $S$-module $M$ ($S$ being the Cox ring of $V$), this method computes all sheaf cohomologies $H^\ast( V, \tilde{M} )$. }

 

\subsection{\textcolor{Chapter }{AllHiParallel (for IsToricVariety, IsFpGradedLeftOrRightModulesObject)}}
\logpage{[ 14, 4, 2 ]}\nobreak
\hyperdef{L}{X7CC947D1784AC28A}{}
{\noindent\textcolor{FuncColor}{$\triangleright$\enspace\texttt{AllHiParallel({\mdseries\slshape V, M})\index{AllHiParallel@\texttt{AllHiParallel}!for IsToricVariety, IsFpGradedLeftOrRightModulesObject}
\label{AllHiParallel:for IsToricVariety, IsFpGradedLeftOrRightModulesObject}
}\hfill{\scriptsize (operation)}}\\
\textbf{\indent Returns:\ }
a list of vector spaces 



 Given a variety $V$ and an f.p. graded $S$-module $M$ ($S$ being the Cox ring of $V$), this method computes all sheaf cohomologies $H^\ast( V, \tilde{M} )$. This method is parallelized and is thus best suited for long and complicated
computations. }

 }

 
\section{\textcolor{Chapter }{Examples}}\label{Chapter_Sheaf_cohomology_by_use_of_httpsarxivorgabs180208860_Section_Examples}
\logpage{[ 14, 5, 0 ]}
\hyperdef{L}{X7A489A5D79DA9E5C}{}
{
  
\subsection{\textcolor{Chapter }{Sheaf cohomology of toric vector bundles}}\label{Chapter_Sheaf_cohomology_by_use_of_httpsarxivorgabs180208860_Section_Examples_Subsection_Sheaf_cohomology_of_toric_vector_bundles}
\logpage{[ 14, 5, 1 ]}
\hyperdef{L}{X85DC50617ED0EBF9}{}
{
  
\begin{Verbatim}[commandchars=!@B,fontsize=\small,frame=single,label=Example]
  !gapprompt@gap>B !gapinput@F1 := Fan( [[1],[-1]],[[1],[2]] );B
  <A fan in |R^1>
  !gapprompt@gap>B !gapinput@P1 := ToricVariety( F1 );B
  <A toric variety of dimension 1>
  !gapprompt@gap>B !gapinput@P1xP1 := P1 * P1;B
  <A toric variety of dimension 2 which is a product of 2 toric varieties>
  !gapprompt@gap>B !gapinput@VForCAP := AsFreydCategoryObject( GradedRow( [[[1,1],1],[[-2,0],1]],B
  !gapprompt@>B !gapinput@                                                         CoxRing( P1xP1 ) ) );B
  <A projective object in Category of f.p. graded
  left modules over Q[x_1,x_2,x_3,x_4] (with weights
  [ [ 0, 1 ], [ 1, 0 ], [ 1, 0 ], [ 0, 1 ] ])>
  !gapprompt@gap>B !gapinput@V2ForCAP := AsFreydCategoryObject( GradedRow( [[[-2,0],1]],B
  !gapprompt@>B !gapinput@                                                         CoxRing( P1xP1 ) ) );B
  <A projective object in Category of f.p. graded
  left modules over Q[x_1,x_2,x_3,x_4] (with weights
  [ [ 0, 1 ], [ 1, 0 ], [ 1, 0 ], [ 0, 1 ] ])>
  !gapprompt@gap>B !gapinput@AllHi( P1xP1, VForCAP, false, false );B
  Computing h^0
  ----------------------------------------------
  
  Computing h^1
  ----------------------------------------------
  
  Computing h^2
  ----------------------------------------------
  
  [ [ 0, <A vector space object over Q of dimension 4> ],
    [ 1, <A vector space object over Q of dimension 1> ],
    [ 1, <A vector space object over Q of dimension 0> ] ]
  !gapprompt@gap>B !gapinput@AllHiParallel( P1xP1, VForCAP, false, false );B
  Computing h^0
  ----------------------------------------------
  
  Computing h^1
  ----------------------------------------------
  
  Computing h^2
  ----------------------------------------------
  
  [ [ 0, <A vector space object over Q of dimension 4> ],
    [ 1, <A vector space object over Q of dimension 1> ],
    [ 1, <A vector space object over Q of dimension 0> ] ]
  !gapprompt@gap>B !gapinput@AllHi( P1xP1, V2ForCAP, false, false );B
  Computing h^0
  ----------------------------------------------
  
  Computing h^1
  ----------------------------------------------
  
  Computing h^2
  ----------------------------------------------
  
  [ [ 0, <A vector space object over Q of dimension 0> ],
    [ 1, <A vector space object over Q of dimension 1> ],
    [ 1, <A vector space object over Q of dimension 0> ] ]
  !gapprompt@gap>B !gapinput@AllHiParallel( P1xP1, V2ForCAP, false, false );B
  Computing h^0
  ----------------------------------------------
  
  Computing h^1
  ----------------------------------------------
  
  Computing h^2
  ----------------------------------------------
  
  [ [ 0, <A vector space object over Q of dimension 0> ],
    [ 1, <A vector space object over Q of dimension 1> ],
    [ 1, <A vector space object over Q of dimension 0> ] ]
\end{Verbatim}
 }

 
\subsection{\textcolor{Chapter }{Sheaf cohomologies of the irrelevant ideal of P1xP1}}\label{Chapter_Sheaf_cohomology_by_use_of_httpsarxivorgabs180208860_Section_Examples_Subsection_Sheaf_cohomologies_of_the_irrelevant_ideal_of_P1xP1}
\logpage{[ 14, 5, 2 ]}
\hyperdef{L}{X8161BC5279B1F45E}{}
{
  
\begin{Verbatim}[commandchars=!@|,fontsize=\small,frame=single,label=Example]
  !gapprompt@gap>| !gapinput@irP1xP1 := IrrelevantLeftIdealForCAP( P1xP1 );|
  <An object in Category of f.p. graded left
  modules over Q[x_1,x_2,x_3,x_4] (with weights
  [ [ 0, 1 ], [ 1, 0 ], [ 1, 0 ], [ 0, 1 ] ])>
  !gapprompt@gap>| !gapinput@AllHi( P1xP1, irP1xP1, false, false );|
  Computing h^0
  ----------------------------------------------
  
  Computing h^1
  ----------------------------------------------
  
  Computing h^2
  ----------------------------------------------
  
  [ [ 1, <A vector space object over Q of dimension 1> ],
    [ 1, <A vector space object over Q of dimension 0> ],
    [ 0, <A vector space object over Q of dimension 0> ] ]
  !gapprompt@gap>| !gapinput@AllHiParallel( P1xP1, irP1xP1, false, false );|
  Computing h^0
  ----------------------------------------------
  
  Computing h^1
  ----------------------------------------------
  
  Computing h^2
  ----------------------------------------------
  
  [ [ 1, <A vector space object over Q of dimension 1> ],
    [ 1, <A vector space object over Q of dimension 0> ],
    [ 0, <A vector space object over Q of dimension 0> ] ]
\end{Verbatim}
 }

 }

 }

   
\chapter{\textcolor{Chapter }{Tools for cohomology computations}}\label{Chapter_Tools_for_cohomology_computations}
\logpage{[ 15, 0, 0 ]}
\hyperdef{L}{X857DAA9B83253C53}{}
{
  
\section{\textcolor{Chapter }{Turn CAP Graded Modules into old graded modules and vice versa}}\label{Chapter_Tools_for_cohomology_computations_Section_Turn_CAP_Graded_Modules_into_old_graded_modules_and_vice_versa}
\logpage{[ 15, 1, 0 ]}
\hyperdef{L}{X814BA5C37D9161B5}{}
{
  

\subsection{\textcolor{Chapter }{TurnIntoOldGradedModule (for IsFpGradedLeftOrRightModulesObject)}}
\logpage{[ 15, 1, 1 ]}\nobreak
\hyperdef{L}{X7B552B1D7BE572F2}{}
{\noindent\textcolor{FuncColor}{$\triangleright$\enspace\texttt{TurnIntoOldGradedModule({\mdseries\slshape M})\index{TurnIntoOldGradedModule@\texttt{TurnIntoOldGradedModule}!for IsFpGradedLeftOrRightModulesObject}
\label{TurnIntoOldGradedModule:for IsFpGradedLeftOrRightModulesObject}
}\hfill{\scriptsize (operation)}}\\
\textbf{\indent Returns:\ }
the corresponding graded modules in terms of the 'old' packages GradedModules 



 The argument is a graded left or right module presentation M for CAP }

 }

 
\section{\textcolor{Chapter }{Save CAP f.p. graded module to file}}\label{Chapter_Tools_for_cohomology_computations_Section_Save_CAP_fp_graded_module_to_file}
\logpage{[ 15, 2, 0 ]}
\hyperdef{L}{X7E2D26D17EB8826A}{}
{
  

\subsection{\textcolor{Chapter }{SaveToFileAsOldGradedModule (for IsString, IsFpGradedLeftOrRightModulesObject)}}
\logpage{[ 15, 2, 1 ]}\nobreak
\hyperdef{L}{X80D646137C4C2787}{}
{\noindent\textcolor{FuncColor}{$\triangleright$\enspace\texttt{SaveToFileAsOldGradedModule({\mdseries\slshape M})\index{SaveToFileAsOldGradedModule@\texttt{SaveToFileAsOldGradedModule}!for IsString, IsFpGradedLeftOrRightModulesObject}
\label{SaveToFileAsOldGradedModule:for IsString, IsFpGradedLeftOrRightModulesObject}
}\hfill{\scriptsize (operation)}}\\
\textbf{\indent Returns:\ }
true (in case of success) or raises error in case the file could not be
written 



 The argument is a graded left or right module presentation M for CAP and saves
this module to file as 'old' graded module presentation. By default, the files
are saved in the main directory of the package
'SheafCohomologyOnToricVarieties'. }

 

\subsection{\textcolor{Chapter }{SaveToFileAsCAPGradedModule (for IsString, IsFpGradedLeftOrRightModulesObject)}}
\logpage{[ 15, 2, 2 ]}\nobreak
\hyperdef{L}{X852EB1358279C4BF}{}
{\noindent\textcolor{FuncColor}{$\triangleright$\enspace\texttt{SaveToFileAsCAPGradedModule({\mdseries\slshape M})\index{SaveToFileAsCAPGradedModule@\texttt{SaveToFileAsCAPGradedModule}!for IsString, IsFpGradedLeftOrRightModulesObject}
\label{SaveToFileAsCAPGradedModule:for IsString, IsFpGradedLeftOrRightModulesObject}
}\hfill{\scriptsize (operation)}}\\
\textbf{\indent Returns:\ }
true (in case of success) or raises error in case the file could not be
written 



 The argument is a graded left or right module presentation M for CAP and saves
this module to file as CAP graded module presentation. By default, the files
are saved in the main directory of the package
'SheafCohomologyOnToricVarieties'. }

 }

 
\section{\textcolor{Chapter }{Approximation Of Sheaf Cohomologies}}\label{Chapter_Tools_for_cohomology_computations_Section_Approximation_Of_Sheaf_Cohomologies}
\logpage{[ 15, 3, 0 ]}
\hyperdef{L}{X7F1DCE007A619814}{}
{
  

\subsection{\textcolor{Chapter }{BPowerLeft (for IsToricVariety, IsInt)}}
\logpage{[ 15, 3, 1 ]}\nobreak
\hyperdef{L}{X879C9BF681F92847}{}
{\noindent\textcolor{FuncColor}{$\triangleright$\enspace\texttt{BPowerLeft({\mdseries\slshape V, e})\index{BPowerLeft@\texttt{BPowerLeft}!for IsToricVariety, IsInt}
\label{BPowerLeft:for IsToricVariety, IsInt}
}\hfill{\scriptsize (operation)}}\\
\textbf{\indent Returns:\ }
a CAP graded left module 



 The argument is a toric variety V and a non-negative integer e. The method
computes the e-th Frobenius power of the irrelevant left ideal of V. }

 

\subsection{\textcolor{Chapter }{BPowerRight (for IsToricVariety, IsInt)}}
\logpage{[ 15, 3, 2 ]}\nobreak
\hyperdef{L}{X7DE7517B87B72608}{}
{\noindent\textcolor{FuncColor}{$\triangleright$\enspace\texttt{BPowerRight({\mdseries\slshape V, e})\index{BPowerRight@\texttt{BPowerRight}!for IsToricVariety, IsInt}
\label{BPowerRight:for IsToricVariety, IsInt}
}\hfill{\scriptsize (operation)}}\\
\textbf{\indent Returns:\ }
a CAP graded right module 



 The argument is a toric variety V and a non-negative integer e. The method
computes the e-th Frobenius power of the irrelevant right ideal of V. }

 

\subsection{\textcolor{Chapter }{ApproxH0 (for IsToricVariety, IsInt, IsFpGradedLeftOrRightModulesObject)}}
\logpage{[ 15, 3, 3 ]}\nobreak
\hyperdef{L}{X782C40FD789477A0}{}
{\noindent\textcolor{FuncColor}{$\triangleright$\enspace\texttt{ApproxH0({\mdseries\slshape V, e, M})\index{ApproxH0@\texttt{ApproxH0}!for IsToricVariety, IsInt, IsFpGradedLeftOrRightModulesObject}
\label{ApproxH0:for IsToricVariety, IsInt, IsFpGradedLeftOrRightModulesObject}
}\hfill{\scriptsize (operation)}}\\
\textbf{\indent Returns:\ }
a non-negative integer 



 The argument is a toric variety V, a non-negative integer e and a graded CAP
module M. The method computes the degree zero layer of Hom( B(e), M ) and
returns its vector space dimension. }

 

\subsection{\textcolor{Chapter }{ApproxH0Parallel (for IsToricVariety, IsInt, IsFpGradedLeftOrRightModulesObject)}}
\logpage{[ 15, 3, 4 ]}\nobreak
\hyperdef{L}{X7A20D76A87500E66}{}
{\noindent\textcolor{FuncColor}{$\triangleright$\enspace\texttt{ApproxH0Parallel({\mdseries\slshape V, e, M})\index{ApproxH0Parallel@\texttt{ApproxH0Parallel}!for IsToricVariety, IsInt, IsFpGradedLeftOrRightModulesObject}
\label{ApproxH0Parallel:for IsToricVariety, IsInt, IsFpGradedLeftOrRightModulesObject}
}\hfill{\scriptsize (operation)}}\\
\textbf{\indent Returns:\ }
a non-negative integer 



 The argument is a toric variety V, a non-negative integer e and a graded CAP
module M. The method computes the degree zero layer of Hom( B(e), M ) by use
of parallelisation and returns its vector space dimension. }

 

\subsection{\textcolor{Chapter }{ApproxHi (for IsToricVariety, IsInt, IsInt, IsFpGradedLeftOrRightModulesObject)}}
\logpage{[ 15, 3, 5 ]}\nobreak
\hyperdef{L}{X7DC2C6307B02302C}{}
{\noindent\textcolor{FuncColor}{$\triangleright$\enspace\texttt{ApproxHi({\mdseries\slshape V, i, e, M})\index{ApproxHi@\texttt{ApproxHi}!for IsToricVariety, IsInt, IsInt, IsFpGradedLeftOrRightModulesObject}
\label{ApproxHi:for IsToricVariety, IsInt, IsInt, IsFpGradedLeftOrRightModulesObject}
}\hfill{\scriptsize (operation)}}\\
\textbf{\indent Returns:\ }
a non-negative integer 



 The argument is a toric variety V, non-negative integers i, e and a graded CAP
module M. The method computes the degree zero layer of
Ext\texttt{\symbol{94}}i( B(e), M ) and returns its vector space dimension. }

 

\subsection{\textcolor{Chapter }{ApproxHiParallel (for IsToricVariety, IsInt, IsInt, IsFpGradedLeftOrRightModulesObject)}}
\logpage{[ 15, 3, 6 ]}\nobreak
\hyperdef{L}{X82F50DDF83474880}{}
{\noindent\textcolor{FuncColor}{$\triangleright$\enspace\texttt{ApproxHiParallel({\mdseries\slshape V, i, e, M})\index{ApproxHiParallel@\texttt{ApproxHiParallel}!for IsToricVariety, IsInt, IsInt, IsFpGradedLeftOrRightModulesObject}
\label{ApproxHiParallel:for IsToricVariety, IsInt, IsInt, IsFpGradedLeftOrRightModulesObject}
}\hfill{\scriptsize (operation)}}\\
\textbf{\indent Returns:\ }
a non-negative integer 



 The argument is a toric variety V, non-negative integer i, e and a graded CAP
module M. The method computes the degree zero layer of
Ext\texttt{\symbol{94}}i( B(e), M ) by use of parallelisation and returns its
vector space dimension. }

 }

 
\section{\textcolor{Chapter }{Examples}}\label{Chapter_Tools_for_cohomology_computations_Section_Examples}
\logpage{[ 15, 4, 0 ]}
\hyperdef{L}{X7A489A5D79DA9E5C}{}
{
  
\subsection{\textcolor{Chapter }{Conversion of modules}}\label{Chapter_Tools_for_cohomology_computations_Section_Examples_Subsection_Conversion_of_modules}
\logpage{[ 15, 4, 1 ]}
\hyperdef{L}{X819D1D7679F04980}{}
{
  
\begin{Verbatim}[commandchars=!@|,fontsize=\small,frame=single,label=Example]
  !gapprompt@gap>| !gapinput@P1 := ProjectiveSpace( 1 );|
  <A projective toric variety of dimension 1>
  !gapprompt@gap>| !gapinput@P1xP1 := P1 * P1;|
  <A projective toric variety of dimension 2
  which is a product of 2 toric varieties>
  !gapprompt@gap>| !gapinput@irP1xP1 := IrrelevantLeftIdealForCAP( P1xP1 );|
  <An object in Category of f.p. graded left
  modules over Q[x_1,x_2,x_3,x_4] (with weights
  [ [ 0, 1 ], [ 1, 0 ], [ 1, 0 ], [ 0, 1 ] ])>
  !gapprompt@gap>| !gapinput@module2 := TurnIntoOldGradedModule( irP1xP1 );|
  <A graded left module presented by 4 relations for 4 generators>
  !gapprompt@gap>| !gapinput@module3 := TurnIntoCAPGradedModule( module2 );|
  <An object in Category of f.p. graded left
  modules over Q[x_1,x_2,x_3,x_4] (with weights
  [ [ 0, 1 ], [ 1, 0 ], [ 1, 0 ], [ 0, 1 ] ])>
  !gapprompt@gap>| !gapinput@module3 = irP1xP1;|
  true
\end{Verbatim}
 }

 
\subsection{\textcolor{Chapter }{Approximation of 0-th sheaf cohomology}}\label{Chapter_Tools_for_cohomology_computations_Section_Examples_Subsection_Approximation_of_0-th_sheaf_cohomology}
\logpage{[ 15, 4, 2 ]}
\hyperdef{L}{X7EB5685C81D410CF}{}
{
  
\begin{Verbatim}[commandchars=!@|,fontsize=\small,frame=single,label=Example]
  !gapprompt@gap>| !gapinput@ApproxH0( P1xP1, 0, irP1xP1 );|
  <A vector space object over Q of dimension 0>
  !gapprompt@gap>| !gapinput@ApproxH0( P1xP1, 1, irP1xP1 );|
  <A vector space object over Q of dimension 1>
  !gapprompt@gap>| !gapinput@ApproxH0( P1xP1, 2, irP1xP1 );|
  <A vector space object over Q of dimension 1>
  !gapprompt@gap>| !gapinput@ApproxH0Parallel( P1xP1, 0, irP1xP1 );|
  <A vector space object over Q of dimension 0>
  !gapprompt@gap>| !gapinput@ApproxH0Parallel( P1xP1, 1, irP1xP1 );|
  <A vector space object over Q of dimension 1>
  !gapprompt@gap>| !gapinput@ApproxH0Parallel( P1xP1, 2, irP1xP1 );|
  <A vector space object over Q of dimension 1>
\end{Verbatim}
 }

 
\subsection{\textcolor{Chapter }{Approximation of 1-st sheaf cohomology}}\label{Chapter_Tools_for_cohomology_computations_Section_Examples_Subsection_Approximation_of_1-st_sheaf_cohomology}
\logpage{[ 15, 4, 3 ]}
\hyperdef{L}{X80CE8A8086FB803A}{}
{
  
\begin{Verbatim}[commandchars=!@B,fontsize=\small,frame=single,label=Example]
  !gapprompt@gap>B !gapinput@F1 := Fan( [[1],[-1]],[[1],[2]] );B
  <A fan in |R^1>
  !gapprompt@gap>B !gapinput@P1 := ToricVariety( F1 );B
  <A toric variety of dimension 1>
  !gapprompt@gap>B !gapinput@P1xP1 := P1 * P1;B
  <A toric variety of dimension 2 which is a product of 2 toric varieties>
  !gapprompt@gap>B !gapinput@VForCAP := AsFreydCategoryObject( GradedRow( [[[1,1],1],[[-2,0],1]],B
  !gapprompt@>B !gapinput@                                                         CoxRing( P1xP1 ) ) );B
  <A projective object in Category of f.p. graded
  left modules over Q[x_1,x_2,x_3,x_4] (with weights
  [ [ 0, 1 ], [ 1, 0 ], [ 1, 0 ], [ 0, 1 ] ])>
  !gapprompt@gap>B !gapinput@ApproxHi( P1xP1, 1, 0, VForCAP );B
  <A vector space object over Q of dimension 0>
  !gapprompt@gap>B !gapinput@ApproxHi( P1xP1, 1, 1, VForCAP );B
  <A vector space object over Q of dimension 1>
  !gapprompt@gap>B !gapinput@ApproxHi( P1xP1, 1, 2, VForCAP );B
  <A vector space object over Q of dimension 1>
  !gapprompt@gap>B !gapinput@ApproxHiParallel( P1xP1, 1, 0, VForCAP );B
  <A vector space object over Q of dimension 0>
  !gapprompt@gap>B !gapinput@ApproxHiParallel( P1xP1, 1, 1, VForCAP );B
  <A vector space object over Q of dimension 1>
  !gapprompt@gap>B !gapinput@ApproxHiParallel( P1xP1, 1, 2, VForCAP );B
  <A vector space object over Q of dimension 1>
\end{Verbatim}
 }

 }

 }

 \def\indexname{Index\logpage{[ "Ind", 0, 0 ]}
\hyperdef{L}{X83A0356F839C696F}{}
}

\cleardoublepage
\phantomsection
\addcontentsline{toc}{chapter}{Index}


\printindex

\immediate\write\pagenrlog{["Ind", 0, 0], \arabic{page},}
\newpage
\immediate\write\pagenrlog{["End"], \arabic{page}];}
\immediate\closeout\pagenrlog
\end{document}
